\documentclass[11pt,dvipsnames]{article}

    \usepackage[breakable]{tcolorbox}
    \usepackage{parskip} % Stop auto-indenting (to mimic markdown behaviour)
    

    % Basic figure setup, for now with no caption control since it's done
    % automatically by Pandoc (which extracts ![](path) syntax from Markdown).
    \usepackage{graphicx}
    % Maintain compatibility with old templates. Remove in nbconvert 6.0
    \let\Oldincludegraphics\includegraphics
    % Ensure that by default, figures have no caption (until we provide a
    % proper Figure object with a Caption API and a way to capture that
    % in the conversion process - todo).
    \usepackage{caption}
    \DeclareCaptionFormat{nocaption}{}
    \captionsetup{format=nocaption,aboveskip=0pt,belowskip=0pt}

    \usepackage{float}
    \floatplacement{figure}{H} % forces figures to be placed at the correct location
    \usepackage{xcolor} % Allow colors to be defined
    \usepackage{enumerate} % Needed for markdown enumerations to work
    \usepackage{geometry} % Used to adjust the document margins
    \usepackage{amsmath} % Equations
    \usepackage{amssymb} % Equations
    \usepackage{textcomp} % defines textquotesingle
    % Hack from http://tex.stackexchange.com/a/47451/13684:
    \AtBeginDocument{%
        \def\PYZsq{\textquotesingle}% Upright quotes in Pygmentized code
    }
    \usepackage{upquote} % Upright quotes for verbatim code
    \usepackage{eurosym} % defines \euro

    \usepackage{iftex}
    \ifPDFTeX
        \usepackage[T1]{fontenc}
        \IfFileExists{alphabeta.sty}{
              \usepackage{alphabeta}
          }{
              \usepackage[mathletters]{ucs}
              \usepackage[utf8x]{inputenc}
          }
    \else
        \usepackage{fontspec}
        \usepackage{unicode-math}
    \fi

    \usepackage{fancyvrb} % verbatim replacement that allows latex
    \usepackage{grffile} % extends the file name processing of package graphics
                         % to support a larger range
    \makeatletter % fix for old versions of grffile with XeLaTeX
    \@ifpackagelater{grffile}{2019/11/01}
    {
      % Do nothing on new versions
    }
    {
      \def\Gread@@xetex#1{%
        \IfFileExists{"\Gin@base".bb}%
        {\Gread@eps{\Gin@base.bb}}%
        {\Gread@@xetex@aux#1}%
      }
    }
    \makeatother
    \usepackage[Export]{adjustbox} % Used to constrain images to a maximum size
    \adjustboxset{max size={0.9\linewidth}{0.9\paperheight}}

    % The hyperref package gives us a pdf with properly built
    % internal navigation ('pdf bookmarks' for the table of contents,
    % internal cross-reference links, web links for URLs, etc.)
    \usepackage{hyperref}
    % The default LaTeX title has an obnoxious amount of whitespace. By default,
    % titling removes some of it. It also provides customization options.
    \usepackage{titling}
    \usepackage{longtable} % longtable support required by pandoc >1.10
    \usepackage{booktabs}  % table support for pandoc > 1.12.2
    \usepackage{array}     % table support for pandoc >= 2.11.3
    \usepackage{calc}      % table minipage width calculation for pandoc >= 2.11.1
    \usepackage[inline]{enumitem} % IRkernel/repr support (it uses the enumerate* environment)
    \usepackage[normalem]{ulem} % ulem is needed to support strikethroughs (\sout)
                                % normalem makes italics be italics, not underlines
    \usepackage{soul}      % strikethrough (\st) support for pandoc >= 3.0.0
    \usepackage{mathrsfs}
    

    
    % Colors for the hyperref package
    \definecolor{urlcolor}{rgb}{0,.145,.698}
    \definecolor{linkcolor}{rgb}{.71,0.21,0.01}
    \definecolor{citecolor}{rgb}{.12,.54,.11}

    % ANSI colors
    \definecolor{ansi-black}{HTML}{3E424D}
    \definecolor{ansi-black-intense}{HTML}{282C36}
    \definecolor{ansi-red}{HTML}{E75C58}
    \definecolor{ansi-red-intense}{HTML}{B22B31}
    \definecolor{ansi-green}{HTML}{00A250}
    \definecolor{ansi-green-intense}{HTML}{007427}
    \definecolor{ansi-yellow}{HTML}{DDB62B}
    \definecolor{ansi-yellow-intense}{HTML}{B27D12}
    \definecolor{ansi-blue}{HTML}{208FFB}
    \definecolor{ansi-blue-intense}{HTML}{0065CA}
    \definecolor{ansi-magenta}{HTML}{D160C4}
    \definecolor{ansi-magenta-intense}{HTML}{A03196}
    \definecolor{ansi-cyan}{HTML}{60C6C8}
    \definecolor{ansi-cyan-intense}{HTML}{258F8F}
    \definecolor{ansi-white}{HTML}{C5C1B4}
    \definecolor{ansi-white-intense}{HTML}{A1A6B2}
    \definecolor{ansi-default-inverse-fg}{HTML}{FFFFFF}
    \definecolor{ansi-default-inverse-bg}{HTML}{000000}

    % common color for the border for error outputs.
    \definecolor{outerrorbackground}{HTML}{FFDFDF}

    % commands and environments needed by pandoc snippets
    % extracted from the output of `pandoc -s`
    \providecommand{\tightlist}{%
      \setlength{\itemsep}{0pt}\setlength{\parskip}{0pt}}
    \DefineVerbatimEnvironment{Highlighting}{Verbatim}{commandchars=\\\{\}}
    % Add ',fontsize=\small' for more characters per line
    \newenvironment{Shaded}{}{}
    \newcommand{\KeywordTok}[1]{\textcolor[rgb]{0.00,0.44,0.13}{\textbf{{#1}}}}
    \newcommand{\DataTypeTok}[1]{\textcolor[rgb]{0.56,0.13,0.00}{{#1}}}
    \newcommand{\DecValTok}[1]{\textcolor[rgb]{0.25,0.63,0.44}{{#1}}}
    \newcommand{\BaseNTok}[1]{\textcolor[rgb]{0.25,0.63,0.44}{{#1}}}
    \newcommand{\FloatTok}[1]{\textcolor[rgb]{0.25,0.63,0.44}{{#1}}}
    \newcommand{\CharTok}[1]{\textcolor[rgb]{0.25,0.44,0.63}{{#1}}}
    \newcommand{\StringTok}[1]{\textcolor[rgb]{0.25,0.44,0.63}{{#1}}}
    \newcommand{\CommentTok}[1]{\textcolor[rgb]{0.38,0.63,0.69}{\textit{{#1}}}}
    \newcommand{\OtherTok}[1]{\textcolor[rgb]{0.00,0.44,0.13}{{#1}}}
    \newcommand{\AlertTok}[1]{\textcolor[rgb]{1.00,0.00,0.00}{\textbf{{#1}}}}
    \newcommand{\FunctionTok}[1]{\textcolor[rgb]{0.02,0.16,0.49}{{#1}}}
    \newcommand{\RegionMarkerTok}[1]{{#1}}
    \newcommand{\ErrorTok}[1]{\textcolor[rgb]{1.00,0.00,0.00}{\textbf{{#1}}}}
    \newcommand{\NormalTok}[1]{{#1}}

    % Additional commands for more recent versions of Pandoc
    \newcommand{\ConstantTok}[1]{\textcolor[rgb]{0.53,0.00,0.00}{{#1}}}
    \newcommand{\SpecialCharTok}[1]{\textcolor[rgb]{0.25,0.44,0.63}{{#1}}}
    \newcommand{\VerbatimStringTok}[1]{\textcolor[rgb]{0.25,0.44,0.63}{{#1}}}
    \newcommand{\SpecialStringTok}[1]{\textcolor[rgb]{0.73,0.40,0.53}{{#1}}}
    \newcommand{\ImportTok}[1]{{#1}}
    \newcommand{\DocumentationTok}[1]{\textcolor[rgb]{0.73,0.13,0.13}{\textit{{#1}}}}
    \newcommand{\AnnotationTok}[1]{\textcolor[rgb]{0.38,0.63,0.69}{\textbf{\textit{{#1}}}}}
    \newcommand{\CommentVarTok}[1]{\textcolor[rgb]{0.38,0.63,0.69}{\textbf{\textit{{#1}}}}}
    \newcommand{\VariableTok}[1]{\textcolor[rgb]{0.10,0.09,0.49}{{#1}}}
    \newcommand{\ControlFlowTok}[1]{\textcolor[rgb]{0.00,0.44,0.13}{\textbf{{#1}}}}
    \newcommand{\OperatorTok}[1]{\textcolor[rgb]{0.40,0.40,0.40}{{#1}}}
    \newcommand{\BuiltInTok}[1]{{#1}}
    \newcommand{\ExtensionTok}[1]{{#1}}
    \newcommand{\PreprocessorTok}[1]{\textcolor[rgb]{0.74,0.48,0.00}{{#1}}}
    \newcommand{\AttributeTok}[1]{\textcolor[rgb]{0.49,0.56,0.16}{{#1}}}
    \newcommand{\InformationTok}[1]{\textcolor[rgb]{0.38,0.63,0.69}{\textbf{\textit{{#1}}}}}
    \newcommand{\WarningTok}[1]{\textcolor[rgb]{0.38,0.63,0.69}{\textbf{\textit{{#1}}}}}


    % Define a nice break command that doesn't care if a line doesn't already
    % exist.
    \def\br{\hspace*{\fill} \\* }
    % Math Jax compatibility definitions
    \def\gt{>}
    \def\lt{<}
    \let\Oldtex\TeX
    \let\Oldlatex\LaTeX
    \renewcommand{\TeX}{\textrm{\Oldtex}}
    \renewcommand{\LaTeX}{\textrm{\Oldlatex}}
    % Document parameters
    % Document title
    \title{ZapataFergusonM\_Session2}
    
    
    
    
    
    
    
% Pygments definitions
\makeatletter
\def\PY@reset{\let\PY@it=\relax \let\PY@bf=\relax%
    \let\PY@ul=\relax \let\PY@tc=\relax%
    \let\PY@bc=\relax \let\PY@ff=\relax}
\def\PY@tok#1{\csname PY@tok@#1\endcsname}
\def\PY@toks#1+{\ifx\relax#1\empty\else%
    \PY@tok{#1}\expandafter\PY@toks\fi}
\def\PY@do#1{\PY@bc{\PY@tc{\PY@ul{%
    \PY@it{\PY@bf{\PY@ff{#1}}}}}}}
\def\PY#1#2{\PY@reset\PY@toks#1+\relax+\PY@do{#2}}

\@namedef{PY@tok@w}{\def\PY@tc##1{\textcolor[rgb]{0.73,0.73,0.73}{##1}}}
\@namedef{PY@tok@c}{\let\PY@it=\textit\def\PY@tc##1{\textcolor[rgb]{0.24,0.48,0.48}{##1}}}
\@namedef{PY@tok@cp}{\def\PY@tc##1{\textcolor[rgb]{0.61,0.40,0.00}{##1}}}
\@namedef{PY@tok@k}{\let\PY@bf=\textbf\def\PY@tc##1{\textcolor[rgb]{0.00,0.50,0.00}{##1}}}
\@namedef{PY@tok@kp}{\def\PY@tc##1{\textcolor[rgb]{0.00,0.50,0.00}{##1}}}
\@namedef{PY@tok@kt}{\def\PY@tc##1{\textcolor[rgb]{0.69,0.00,0.25}{##1}}}
\@namedef{PY@tok@o}{\def\PY@tc##1{\textcolor[rgb]{0.40,0.40,0.40}{##1}}}
\@namedef{PY@tok@ow}{\let\PY@bf=\textbf\def\PY@tc##1{\textcolor[rgb]{0.67,0.13,1.00}{##1}}}
\@namedef{PY@tok@nb}{\def\PY@tc##1{\textcolor[rgb]{0.00,0.50,0.00}{##1}}}
\@namedef{PY@tok@nf}{\def\PY@tc##1{\textcolor[rgb]{0.00,0.00,1.00}{##1}}}
\@namedef{PY@tok@nc}{\let\PY@bf=\textbf\def\PY@tc##1{\textcolor[rgb]{0.00,0.00,1.00}{##1}}}
\@namedef{PY@tok@nn}{\let\PY@bf=\textbf\def\PY@tc##1{\textcolor[rgb]{0.00,0.00,1.00}{##1}}}
\@namedef{PY@tok@ne}{\let\PY@bf=\textbf\def\PY@tc##1{\textcolor[rgb]{0.80,0.25,0.22}{##1}}}
\@namedef{PY@tok@nv}{\def\PY@tc##1{\textcolor[rgb]{0.10,0.09,0.49}{##1}}}
\@namedef{PY@tok@no}{\def\PY@tc##1{\textcolor[rgb]{0.53,0.00,0.00}{##1}}}
\@namedef{PY@tok@nl}{\def\PY@tc##1{\textcolor[rgb]{0.46,0.46,0.00}{##1}}}
\@namedef{PY@tok@ni}{\let\PY@bf=\textbf\def\PY@tc##1{\textcolor[rgb]{0.44,0.44,0.44}{##1}}}
\@namedef{PY@tok@na}{\def\PY@tc##1{\textcolor[rgb]{0.41,0.47,0.13}{##1}}}
\@namedef{PY@tok@nt}{\let\PY@bf=\textbf\def\PY@tc##1{\textcolor[rgb]{0.00,0.50,0.00}{##1}}}
\@namedef{PY@tok@nd}{\def\PY@tc##1{\textcolor[rgb]{0.67,0.13,1.00}{##1}}}
\@namedef{PY@tok@s}{\def\PY@tc##1{\textcolor[rgb]{0.73,0.13,0.13}{##1}}}
\@namedef{PY@tok@sd}{\let\PY@it=\textit\def\PY@tc##1{\textcolor[rgb]{0.73,0.13,0.13}{##1}}}
\@namedef{PY@tok@si}{\let\PY@bf=\textbf\def\PY@tc##1{\textcolor[rgb]{0.64,0.35,0.47}{##1}}}
\@namedef{PY@tok@se}{\let\PY@bf=\textbf\def\PY@tc##1{\textcolor[rgb]{0.67,0.36,0.12}{##1}}}
\@namedef{PY@tok@sr}{\def\PY@tc##1{\textcolor[rgb]{0.64,0.35,0.47}{##1}}}
\@namedef{PY@tok@ss}{\def\PY@tc##1{\textcolor[rgb]{0.10,0.09,0.49}{##1}}}
\@namedef{PY@tok@sx}{\def\PY@tc##1{\textcolor[rgb]{0.00,0.50,0.00}{##1}}}
\@namedef{PY@tok@m}{\def\PY@tc##1{\textcolor[rgb]{0.40,0.40,0.40}{##1}}}
\@namedef{PY@tok@gh}{\let\PY@bf=\textbf\def\PY@tc##1{\textcolor[rgb]{0.00,0.00,0.50}{##1}}}
\@namedef{PY@tok@gu}{\let\PY@bf=\textbf\def\PY@tc##1{\textcolor[rgb]{0.50,0.00,0.50}{##1}}}
\@namedef{PY@tok@gd}{\def\PY@tc##1{\textcolor[rgb]{0.63,0.00,0.00}{##1}}}
\@namedef{PY@tok@gi}{\def\PY@tc##1{\textcolor[rgb]{0.00,0.52,0.00}{##1}}}
\@namedef{PY@tok@gr}{\def\PY@tc##1{\textcolor[rgb]{0.89,0.00,0.00}{##1}}}
\@namedef{PY@tok@ge}{\let\PY@it=\textit}
\@namedef{PY@tok@gs}{\let\PY@bf=\textbf}
\@namedef{PY@tok@ges}{\let\PY@bf=\textbf\let\PY@it=\textit}
\@namedef{PY@tok@gp}{\let\PY@bf=\textbf\def\PY@tc##1{\textcolor[rgb]{0.00,0.00,0.50}{##1}}}
\@namedef{PY@tok@go}{\def\PY@tc##1{\textcolor[rgb]{0.44,0.44,0.44}{##1}}}
\@namedef{PY@tok@gt}{\def\PY@tc##1{\textcolor[rgb]{0.00,0.27,0.87}{##1}}}
\@namedef{PY@tok@err}{\def\PY@bc##1{{\setlength{\fboxsep}{\string -\fboxrule}\fcolorbox[rgb]{1.00,0.00,0.00}{1,1,1}{\strut ##1}}}}
\@namedef{PY@tok@kc}{\let\PY@bf=\textbf\def\PY@tc##1{\textcolor[rgb]{0.00,0.50,0.00}{##1}}}
\@namedef{PY@tok@kd}{\let\PY@bf=\textbf\def\PY@tc##1{\textcolor[rgb]{0.00,0.50,0.00}{##1}}}
\@namedef{PY@tok@kn}{\let\PY@bf=\textbf\def\PY@tc##1{\textcolor[rgb]{0.00,0.50,0.00}{##1}}}
\@namedef{PY@tok@kr}{\let\PY@bf=\textbf\def\PY@tc##1{\textcolor[rgb]{0.00,0.50,0.00}{##1}}}
\@namedef{PY@tok@bp}{\def\PY@tc##1{\textcolor[rgb]{0.00,0.50,0.00}{##1}}}
\@namedef{PY@tok@fm}{\def\PY@tc##1{\textcolor[rgb]{0.00,0.00,1.00}{##1}}}
\@namedef{PY@tok@vc}{\def\PY@tc##1{\textcolor[rgb]{0.10,0.09,0.49}{##1}}}
\@namedef{PY@tok@vg}{\def\PY@tc##1{\textcolor[rgb]{0.10,0.09,0.49}{##1}}}
\@namedef{PY@tok@vi}{\def\PY@tc##1{\textcolor[rgb]{0.10,0.09,0.49}{##1}}}
\@namedef{PY@tok@vm}{\def\PY@tc##1{\textcolor[rgb]{0.10,0.09,0.49}{##1}}}
\@namedef{PY@tok@sa}{\def\PY@tc##1{\textcolor[rgb]{0.73,0.13,0.13}{##1}}}
\@namedef{PY@tok@sb}{\def\PY@tc##1{\textcolor[rgb]{0.73,0.13,0.13}{##1}}}
\@namedef{PY@tok@sc}{\def\PY@tc##1{\textcolor[rgb]{0.73,0.13,0.13}{##1}}}
\@namedef{PY@tok@dl}{\def\PY@tc##1{\textcolor[rgb]{0.73,0.13,0.13}{##1}}}
\@namedef{PY@tok@s2}{\def\PY@tc##1{\textcolor[rgb]{0.73,0.13,0.13}{##1}}}
\@namedef{PY@tok@sh}{\def\PY@tc##1{\textcolor[rgb]{0.73,0.13,0.13}{##1}}}
\@namedef{PY@tok@s1}{\def\PY@tc##1{\textcolor[rgb]{0.73,0.13,0.13}{##1}}}
\@namedef{PY@tok@mb}{\def\PY@tc##1{\textcolor[rgb]{0.40,0.40,0.40}{##1}}}
\@namedef{PY@tok@mf}{\def\PY@tc##1{\textcolor[rgb]{0.40,0.40,0.40}{##1}}}
\@namedef{PY@tok@mh}{\def\PY@tc##1{\textcolor[rgb]{0.40,0.40,0.40}{##1}}}
\@namedef{PY@tok@mi}{\def\PY@tc##1{\textcolor[rgb]{0.40,0.40,0.40}{##1}}}
\@namedef{PY@tok@il}{\def\PY@tc##1{\textcolor[rgb]{0.40,0.40,0.40}{##1}}}
\@namedef{PY@tok@mo}{\def\PY@tc##1{\textcolor[rgb]{0.40,0.40,0.40}{##1}}}
\@namedef{PY@tok@ch}{\let\PY@it=\textit\def\PY@tc##1{\textcolor[rgb]{0.24,0.48,0.48}{##1}}}
\@namedef{PY@tok@cm}{\let\PY@it=\textit\def\PY@tc##1{\textcolor[rgb]{0.24,0.48,0.48}{##1}}}
\@namedef{PY@tok@cpf}{\let\PY@it=\textit\def\PY@tc##1{\textcolor[rgb]{0.24,0.48,0.48}{##1}}}
\@namedef{PY@tok@c1}{\let\PY@it=\textit\def\PY@tc##1{\textcolor[rgb]{0.24,0.48,0.48}{##1}}}
\@namedef{PY@tok@cs}{\let\PY@it=\textit\def\PY@tc##1{\textcolor[rgb]{0.24,0.48,0.48}{##1}}}

\def\PYZbs{\char`\\}
\def\PYZus{\char`\_}
\def\PYZob{\char`\{}
\def\PYZcb{\char`\}}
\def\PYZca{\char`\^}
\def\PYZam{\char`\&}
\def\PYZlt{\char`\<}
\def\PYZgt{\char`\>}
\def\PYZsh{\char`\#}
\def\PYZpc{\char`\%}
\def\PYZdl{\char`\$}
\def\PYZhy{\char`\-}
\def\PYZsq{\char`\'}
\def\PYZdq{\char`\"}
\def\PYZti{\char`\~}
% for compatibility with earlier versions
\def\PYZat{@}
\def\PYZlb{[}
\def\PYZrb{]}
\makeatother


    % For linebreaks inside Verbatim environment from package fancyvrb.
    \makeatletter
        \newbox\Wrappedcontinuationbox
        \newbox\Wrappedvisiblespacebox
        \newcommand*\Wrappedvisiblespace {\textcolor{red}{\textvisiblespace}}
        \newcommand*\Wrappedcontinuationsymbol {\textcolor{red}{\llap{\tiny$\m@th\hookrightarrow$}}}
        \newcommand*\Wrappedcontinuationindent {3ex }
        \newcommand*\Wrappedafterbreak {\kern\Wrappedcontinuationindent\copy\Wrappedcontinuationbox}
        % Take advantage of the already applied Pygments mark-up to insert
        % potential linebreaks for TeX processing.
        %        {, <, #, %, $, ' and ": go to next line.
        %        _, }, ^, &, >, - and ~: stay at end of broken line.
        % Use of \textquotesingle for straight quote.
        \newcommand*\Wrappedbreaksatspecials {%
            \def\PYGZus{\discretionary{\char`\_}{\Wrappedafterbreak}{\char`\_}}%
            \def\PYGZob{\discretionary{}{\Wrappedafterbreak\char`\{}{\char`\{}}%
            \def\PYGZcb{\discretionary{\char`\}}{\Wrappedafterbreak}{\char`\}}}%
            \def\PYGZca{\discretionary{\char`\^}{\Wrappedafterbreak}{\char`\^}}%
            \def\PYGZam{\discretionary{\char`\&}{\Wrappedafterbreak}{\char`\&}}%
            \def\PYGZlt{\discretionary{}{\Wrappedafterbreak\char`\<}{\char`\<}}%
            \def\PYGZgt{\discretionary{\char`\>}{\Wrappedafterbreak}{\char`\>}}%
            \def\PYGZsh{\discretionary{}{\Wrappedafterbreak\char`\#}{\char`\#}}%
            \def\PYGZpc{\discretionary{}{\Wrappedafterbreak\char`\%}{\char`\%}}%
            \def\PYGZdl{\discretionary{}{\Wrappedafterbreak\char`\$}{\char`\$}}%
            \def\PYGZhy{\discretionary{\char`\-}{\Wrappedafterbreak}{\char`\-}}%
            \def\PYGZsq{\discretionary{}{\Wrappedafterbreak\textquotesingle}{\textquotesingle}}%
            \def\PYGZdq{\discretionary{}{\Wrappedafterbreak\char`\"}{\char`\"}}%
            \def\PYGZti{\discretionary{\char`\~}{\Wrappedafterbreak}{\char`\~}}%
        }
        % Some characters . , ; ? ! / are not pygmentized.
        % This macro makes them "active" and they will insert potential linebreaks
        \newcommand*\Wrappedbreaksatpunct {%
            \lccode`\~`\.\lowercase{\def~}{\discretionary{\hbox{\char`\.}}{\Wrappedafterbreak}{\hbox{\char`\.}}}%
            \lccode`\~`\,\lowercase{\def~}{\discretionary{\hbox{\char`\,}}{\Wrappedafterbreak}{\hbox{\char`\,}}}%
            \lccode`\~`\;\lowercase{\def~}{\discretionary{\hbox{\char`\;}}{\Wrappedafterbreak}{\hbox{\char`\;}}}%
            \lccode`\~`\:\lowercase{\def~}{\discretionary{\hbox{\char`\:}}{\Wrappedafterbreak}{\hbox{\char`\:}}}%
            \lccode`\~`\?\lowercase{\def~}{\discretionary{\hbox{\char`\?}}{\Wrappedafterbreak}{\hbox{\char`\?}}}%
            \lccode`\~`\!\lowercase{\def~}{\discretionary{\hbox{\char`\!}}{\Wrappedafterbreak}{\hbox{\char`\!}}}%
            \lccode`\~`\/\lowercase{\def~}{\discretionary{\hbox{\char`\/}}{\Wrappedafterbreak}{\hbox{\char`\/}}}%
            \catcode`\.\active
            \catcode`\,\active
            \catcode`\;\active
            \catcode`\:\active
            \catcode`\?\active
            \catcode`\!\active
            \catcode`\/\active
            \lccode`\~`\~
        }
    \makeatother

    \let\OriginalVerbatim=\Verbatim
    \makeatletter
    \renewcommand{\Verbatim}[1][1]{%
        %\parskip\z@skip
        \sbox\Wrappedcontinuationbox {\Wrappedcontinuationsymbol}%
        \sbox\Wrappedvisiblespacebox {\FV@SetupFont\Wrappedvisiblespace}%
        \def\FancyVerbFormatLine ##1{\hsize\linewidth
            \vtop{\raggedright\hyphenpenalty\z@\exhyphenpenalty\z@
                \doublehyphendemerits\z@\finalhyphendemerits\z@
                \strut ##1\strut}%
        }%
        % If the linebreak is at a space, the latter will be displayed as visible
        % space at end of first line, and a continuation symbol starts next line.
        % Stretch/shrink are however usually zero for typewriter font.
        \def\FV@Space {%
            \nobreak\hskip\z@ plus\fontdimen3\font minus\fontdimen4\font
            \discretionary{\copy\Wrappedvisiblespacebox}{\Wrappedafterbreak}
            {\kern\fontdimen2\font}%
        }%

        % Allow breaks at special characters using \PYG... macros.
        \Wrappedbreaksatspecials
        % Breaks at punctuation characters . , ; ? ! and / need catcode=\active
        \OriginalVerbatim[#1,codes*=\Wrappedbreaksatpunct]%
    }
    \makeatother

    % Exact colors from NB
    \definecolor{incolor}{HTML}{303F9F}
    \definecolor{outcolor}{HTML}{D84315}
    \definecolor{cellborder}{HTML}{CFCFCF}
    \definecolor{cellbackground}{HTML}{F7F7F7}

    % prompt
    \makeatletter
    \newcommand{\boxspacing}{\kern\kvtcb@left@rule\kern\kvtcb@boxsep}
    \makeatother
    \newcommand{\prompt}[4]{
        {\ttfamily\llap{{\color{#2}[#3]:\hspace{3pt}#4}}\vspace{-\baselineskip}}
    }
    

    
    % Prevent overflowing lines due to hard-to-break entities
    \sloppy
    % Setup hyperref package
    \hypersetup{
      breaklinks=true,  % so long urls are correctly broken across lines
      colorlinks=true,
      urlcolor=urlcolor,
      linkcolor=linkcolor,
      citecolor=citecolor,
      }
    % Slightly bigger margins than the latex defaults
    
    \geometry{verbose,tmargin=1in,bmargin=1in,lmargin=1in,rmargin=1in}
    
    

\begin{document}
    
    

    
\section*{\huge Session 2: An introduction to Qiskit and to IBM's
interface}\label{session-2-an-introduction-to-qiskit-and-to-ibms-interface}


    \begin{tcolorbox}[breakable, size=fbox, boxrule=1pt, pad at break*=1mm,colback=cellbackground, colframe=cellborder]
\prompt{In}{incolor}{30}{\boxspacing}
\begin{Verbatim}[commandchars=\\\{\}]
\PY{c+c1}{\PYZsh{} A classical simulator of quantum circuits}
\PY{k+kn}{from} \PY{n+nn}{qiskit\PYZus{}aer} \PY{k+kn}{import} \PY{n}{AerSimulator}

\PY{c+c1}{\PYZsh{} Class QuantumCircuit, with methods (functions) to define quantum circuits}
\PY{k+kn}{from} \PY{n+nn}{qiskit} \PY{k+kn}{import} \PY{n}{QuantumCircuit}

\PY{c+c1}{\PYZsh{} A function yielding a histogram with the probabilities of each of the eigenstates taken by the measured qubits}
\PY{k+kn}{from} \PY{n+nn}{qiskit}\PY{n+nn}{.}\PY{n+nn}{visualization} \PY{k+kn}{import} \PY{n}{plot\PYZus{}histogram}

\PY{c+c1}{\PYZsh{} Class Statvector, to retrieve the state of the circuit}
\PY{k+kn}{from} \PY{n+nn}{qiskit}\PY{n+nn}{.}\PY{n+nn}{quantum\PYZus{}info} \PY{k+kn}{import} \PY{n}{Statevector}

\PY{c+c1}{\PYZsh{} Class PhaseGate, essential for quantum Fourier transform}
\PY{k+kn}{from} \PY{n+nn}{qiskit}\PY{n+nn}{.}\PY{n+nn}{circuit}\PY{n+nn}{.}\PY{n+nn}{library} \PY{k+kn}{import} \PY{n}{PhaseGate}

\PY{c+c1}{\PYZsh{} Other usual imports}
\PY{k+kn}{import} \PY{n+nn}{numpy} \PY{k}{as} \PY{n+nn}{np}
\PY{k+kn}{import} \PY{n+nn}{matplotlib}\PY{n+nn}{.}\PY{n+nn}{pyplot} \PY{k}{as} \PY{n+nn}{plt}

\PY{c+c1}{\PYZsh{} Printing precision}
\PY{n}{np}\PY{o}{.}\PY{n}{set\PYZus{}printoptions}\PY{p}{(}\PY{n}{precision}\PY{o}{=}\PY{l+m+mi}{2}\PY{p}{)}
\end{Verbatim}
\end{tcolorbox}

    \textbf{Sections 1 and 2 walk us through the basics of Qiskit.}

    \section*{\texorpdfstring{\textbf{3. Executing
    circuits}}{3. Executing circuits}}\label{executing-circuits}

    \textcolor{teal}{\textbf{Question:} Why does the histogram have only one bar
corresponding to \textbar1\textgreater{} ?}

\textbf{Answer:} Because the qubit is initialized to the state
\(|1\rangle\), it has no component along \(|0\rangle\).

    \begin{tcolorbox}[breakable, size=fbox, boxrule=1pt, pad at break*=1mm,colback=cellbackground, colframe=cellborder]
\prompt{In}{incolor}{2}{\boxspacing}
\begin{Verbatim}[commandchars=\\\{\}]
\PY{n}{sim} \PY{o}{=} \PY{n}{AerSimulator}\PY{p}{(}\PY{p}{)}
\PY{n}{qc11} \PY{o}{=} \PY{n}{QuantumCircuit}\PY{p}{(}\PY{l+m+mi}{1}\PY{p}{,}\PY{l+m+mi}{1}\PY{p}{)} \PY{c+c1}{\PYZsh{} circuit with 1 qubit y 1 classical bit}
\PY{n}{state\PYZus{}1} \PY{o}{=} \PY{p}{[}\PY{l+m+mi}{0}\PY{p}{,}\PY{l+m+mi}{1}\PY{p}{]} \PY{c+c1}{\PYZsh{} this would be |1\PYZgt{}, trivially normalized}
\PY{n}{qc11}\PY{o}{.}\PY{n}{initialize}\PY{p}{(}\PY{n}{state\PYZus{}1}\PY{p}{,}\PY{l+m+mi}{0}\PY{p}{)}
\PY{n}{qc11}\PY{o}{.}\PY{n}{measure}\PY{p}{(}\PY{l+m+mi}{0}\PY{p}{,}\PY{l+m+mi}{0}\PY{p}{)}
\PY{n}{resultmeasure} \PY{o}{=} \PY{n}{sim}\PY{o}{.}\PY{n}{run}\PY{p}{(}\PY{n}{qc11}\PY{p}{)}\PY{o}{.}\PY{n}{result}\PY{p}{(}\PY{p}{)} \PY{c+c1}{\PYZsh{} Execute the circuite qc11 in the simulator}
\PY{n}{counts} \PY{o}{=} \PY{n}{resultmeasure}\PY{o}{.}\PY{n}{get\PYZus{}counts}\PY{p}{(}\PY{p}{)} \PY{c+c1}{\PYZsh{} Results of the measurements}
\PY{n}{plot\PYZus{}histogram}\PY{p}{(}\PY{n}{counts}\PY{p}{)} \PY{c+c1}{\PYZsh{} Plot a histogram with that variable}
\end{Verbatim}
\end{tcolorbox}
 
            
\prompt{Out}{outcolor}{2}{}
    
    \begin{center}
    \adjustimage{max size={0.9\linewidth}{0.9\paperheight}}{ZapataFergusonM_Session2_files/ZapataFergusonM_Session2_4_0.png}
    \end{center}
    { \hspace*{\fill} \\}
    

    \textbf{4. Generation of truly random numbers with a quantum computer}

\textcolor{teal}{Question: Can you write in a piece of paper the initial state in the
second line of the next paragraph?}

\textbf{Answer:} Let
\(|+\rangle=\frac{1}{\sqrt{2}}(|0\rangle+|1\rangle)\) in the
computational, single-qubit basis --- the initial state is
\(|+\rangle\otimes|+\rangle\otimes|+\rangle\otimes|+\rangle\). In the
\(\{|n\rangle\}_{n=0}^{15}\) basis, where \(|n\rangle\) is the binary
representation of n, taking the digits \(0\) or \(1\) to be qubits
\(|0\rangle\), \(|1\rangle\) tensored with each other,
\(|\mathrm{init}\rangle=\frac{1}{4}\sum\limits_{n=0}^{15}|n\rangle\).
Indeed, this is what the \texttt{draw} method shows, an equal
superposition of states, which is what we would need to simulate a
uniform probability distribution.

    \begin{tcolorbox}[breakable, size=fbox, boxrule=1pt, pad at break*=1mm,colback=cellbackground, colframe=cellborder]
\prompt{In}{incolor}{3}{\boxspacing}
\begin{Verbatim}[commandchars=\\\{\}]
\PY{n}{qrng} \PY{o}{=} \PY{n}{QuantumCircuit}\PY{p}{(}\PY{l+m+mi}{4}\PY{p}{,}\PY{l+m+mi}{4}\PY{p}{)}
\PY{n}{init\PYZus{}state} \PY{o}{=} \PY{n}{np}\PY{o}{.}\PY{n}{kron}\PY{p}{(}\PY{p}{[}\PY{l+m+mi}{1}\PY{p}{,}\PY{l+m+mi}{1}\PY{p}{]}\PY{p}{,}\PY{n}{np}\PY{o}{.}\PY{n}{kron}\PY{p}{(}\PY{p}{[}\PY{l+m+mi}{1}\PY{p}{,}\PY{l+m+mi}{1}\PY{p}{]}\PY{p}{,}\PY{n}{np}\PY{o}{.}\PY{n}{kron}\PY{p}{(}\PY{p}{[}\PY{l+m+mi}{1}\PY{p}{,}\PY{l+m+mi}{1}\PY{p}{]}\PY{p}{,}\PY{p}{[}\PY{l+m+mi}{1}\PY{p}{,}\PY{l+m+mi}{1}\PY{p}{]}\PY{p}{)}\PY{p}{)}\PY{p}{)} \PY{c+c1}{\PYZsh{} Kron is a Python command for Cartesian (or Kronecker) products}
\PY{n}{init\PYZus{}state\PYZus{}normal} \PY{o}{=} \PY{n}{init\PYZus{}state}\PY{o}{/}\PY{n}{np}\PY{o}{.}\PY{n}{linalg}\PY{o}{.}\PY{n}{norm}\PY{p}{(}\PY{n}{init\PYZus{}state}\PY{p}{)}
\PY{n}{qrng}\PY{o}{.}\PY{n}{initialize}\PY{p}{(}\PY{n}{init\PYZus{}state\PYZus{}normal}\PY{p}{,}\PY{p}{[}\PY{l+m+mi}{0}\PY{p}{,}\PY{l+m+mi}{1}\PY{p}{,}\PY{l+m+mi}{2}\PY{p}{,}\PY{l+m+mi}{3}\PY{p}{]}\PY{p}{)}
\PY{n}{qrng}\PY{o}{.}\PY{n}{draw}\PY{p}{(}\PY{n}{output}\PY{o}{=}\PY{l+s+s1}{\PYZsq{}}\PY{l+s+s1}{mpl}\PY{l+s+s1}{\PYZsq{}}\PY{p}{)}
\end{Verbatim}
\end{tcolorbox}
 
            
\prompt{Out}{outcolor}{3}{}
    
    \begin{center}
    \adjustimage{max size={0.9\linewidth}{0.9\paperheight}}{ZapataFergusonM_Session2_files/ZapataFergusonM_Session2_7_0.png}
    \end{center}
    { \hspace*{\fill} \\}
    

    We may encapsulate random number generation in the following function:

    \begin{tcolorbox}[breakable, size=fbox, boxrule=1pt, pad at break*=1mm,colback=cellbackground, colframe=cellborder]
\prompt{In}{incolor}{4}{\boxspacing}
\begin{Verbatim}[commandchars=\\\{\}]
\PY{k}{def} \PY{n+nf}{QRNG}\PY{p}{(}\PY{n}{N\PYZus{}qubits}\PY{p}{)}\PY{p}{:} \PY{c+c1}{\PYZsh{}a quantum random number generator}
    \PY{n}{qrng} \PY{o}{=} \PY{n}{QuantumCircuit}\PY{p}{(}\PY{n}{N\PYZus{}qubits}\PY{p}{)}
    \PY{n}{qrng}\PY{o}{.}\PY{n}{h}\PY{p}{(}\PY{n+nb}{range}\PY{p}{(}\PY{n}{N\PYZus{}qubits}\PY{p}{)}\PY{p}{)}
    \PY{n}{qrng}\PY{o}{.}\PY{n}{measure\PYZus{}all}\PY{p}{(}\PY{p}{)}
    \PY{n}{result} \PY{o}{=} \PY{n}{sim}\PY{o}{.}\PY{n}{run}\PY{p}{(}\PY{n}{qrng}\PY{p}{,} \PY{n}{shots}\PY{o}{=}\PY{l+m+mi}{1}\PY{p}{)}\PY{o}{.}\PY{n}{result}\PY{p}{(}\PY{p}{)}
    \PY{n}{counts} \PY{o}{=} \PY{n}{result}\PY{o}{.}\PY{n}{get\PYZus{}counts}\PY{p}{(}\PY{p}{)}
    \PY{k}{return} \PY{n+nb}{int}\PY{p}{(}\PY{o}{*}\PY{n}{counts}\PY{p}{,} \PY{l+m+mi}{2}\PY{p}{)}
\end{Verbatim}
\end{tcolorbox}

\textcolor{teal}{Question: Run the circuit a few times and write down the number obtained.}


\textbf{Answer:} After running it a few times we obtained the following
list of numbers:
\texttt{14,\ 3,\ 5,\ 9,\ 8,\ 12,\ 1,\ 10,\ 6,\ 11,\ 2,\ 6,\ 0,\ 10,\ 7,\ 13,\ 6,\ 14,\ 5,\ 1,\ 0,\ 14,\ 4}

    \begin{tcolorbox}[breakable, size=fbox, boxrule=1pt, pad at break*=1mm,colback=cellbackground, colframe=cellborder]
\prompt{In}{incolor}{5}{\boxspacing}
\begin{Verbatim}[commandchars=\\\{\}]
\PY{n}{QRNG}\PY{p}{(}\PY{l+m+mi}{4}\PY{p}{)}
\end{Verbatim}
\end{tcolorbox}

            \begin{tcolorbox}[breakable, size=fbox, boxrule=.5pt, pad at break*=1mm, opacityfill=0]
\prompt{Out}{outcolor}{5}{\boxspacing}
\begin{Verbatim}[commandchars=\\\{\}]
6
\end{Verbatim}
\end{tcolorbox}
        
\textcolor{teal}{Question: Write a loop that generates 32 random numbers and plot a
histogram with their distribution.}


\textbf{Answer:}

    \begin{tcolorbox}[breakable, size=fbox, boxrule=1pt, pad at break*=1mm,colback=cellbackground, colframe=cellborder]
\prompt{In}{incolor}{6}{\boxspacing}
\begin{Verbatim}[commandchars=\\\{\}]
\PY{n}{N\PYZus{}qubits} \PY{o}{=} \PY{l+m+mi}{4}
\PY{n}{N} \PY{o}{=} \PY{l+m+mi}{32}
\PY{n}{distrib} \PY{o}{=} \PY{p}{[}\PY{n}{QRNG}\PY{p}{(}\PY{n}{N\PYZus{}qubits}\PY{p}{)} \PY{k}{for} \PY{n}{\PYZus{}} \PY{o+ow}{in} \PY{n+nb}{range}\PY{p}{(}\PY{n}{N}\PY{p}{)}\PY{p}{]}

\PY{n}{dim} \PY{o}{=} \PY{l+m+mi}{2}\PY{o}{*}\PY{o}{*}\PY{n}{N\PYZus{}qubits}
\PY{n}{plt}\PY{o}{.}\PY{n}{hist}\PY{p}{(}\PY{n}{distrib}\PY{p}{,} \PY{n}{bins}\PY{o}{=}\PY{n+nb}{range}\PY{p}{(}\PY{n}{dim}\PY{o}{+}\PY{l+m+mi}{1}\PY{p}{)}\PY{p}{,} \PY{n}{density}\PY{o}{=}\PY{k+kc}{True}\PY{p}{,} \PY{n}{align}\PY{o}{=}\PY{l+s+s1}{\PYZsq{}}\PY{l+s+s1}{left}\PY{l+s+s1}{\PYZsq{}}\PY{p}{,} \PY{n}{edgecolor}\PY{o}{=}\PY{l+s+s1}{\PYZsq{}}\PY{l+s+s1}{black}\PY{l+s+s1}{\PYZsq{}}\PY{p}{,} \PY{n}{linewidth}\PY{o}{=}\PY{l+m+mi}{1}\PY{p}{)}
\PY{n}{plt}\PY{o}{.}\PY{n}{xticks}\PY{p}{(}\PY{n}{ticks}\PY{o}{=}\PY{n+nb}{range}\PY{p}{(}\PY{n}{dim}\PY{p}{)}\PY{p}{,} \PY{n}{labels}\PY{o}{=}\PY{n+nb}{range}\PY{p}{(}\PY{n}{dim}\PY{p}{)}\PY{p}{)}
\PY{n}{plt}\PY{o}{.}\PY{n}{title}\PY{p}{(}\PY{l+s+s1}{\PYZsq{}}\PY{l+s+s1}{Quantum RNG distribution}\PY{l+s+s1}{\PYZsq{}}\PY{p}{)}
\end{Verbatim}
\end{tcolorbox}

            \begin{tcolorbox}[breakable, size=fbox, boxrule=.5pt, pad at break*=1mm, opacityfill=0]
\prompt{Out}{outcolor}{6}{\boxspacing}
\begin{Verbatim}[commandchars=\\\{\}]
Text(0.5, 1.0, 'Quantum RNG distribution')
\end{Verbatim}
\end{tcolorbox}
        
    \begin{center}
    \adjustimage{max size={0.9\linewidth}{0.9\paperheight}}{ZapataFergusonM_Session2_files/ZapataFergusonM_Session2_13_1.png}
    \end{center}
    { \hspace*{\fill} \\}
    
    \textcolor{teal}{Question: Write down in a piece of paper all the methods that you have
learned here.}

    \textbf{Answer:} We have learnt the following methods:

    Applied to a \texttt{QuantumCircuit} 

    \begin{itemize}
        \item \texttt{draw}: Plots a diagram of
        the circuit. 
        \item \texttt{initialize}: Given a normalised statevector of
        the appropriate dimension, prepares the circuit or subset of the circuit
        in that state. 
        \item \texttt{measure}: Collapses a particular qubit and
        stores result in a particular cbit. 
        \item \texttt{measure\_all}: Adds a cbit
        for each qubit in the circuit and measures all.
    \end{itemize}
    
    Applied to an \texttt{AerSimulator} 
    
    \begin{itemize}
        \item  \texttt{run}: Runs \texttt{QuantumCircuit} on the Aer backend and returns 
        asynchronous simulation job.
    \end{itemize}
    

    Applied to an \texttt{AerJob} 

    \begin{itemize}
        \item \texttt{result}: Awaits job result.
    \end{itemize}
    
    Applied to a \texttt{Result} 
    
    \begin{itemize}
        \item \texttt{get\_counts}: Gets the histogram
        data of a quantum simulation.
    \end{itemize}
    

    \section*{5. Quantum Fourier Transform}\label{quantum-fourier-transform}


    The discrete Fourier transform (DFT) changes the elements of a
computational basis \(B\) to the Fourier basis \(F(B)\) according to the
formula

\[
\left|j\right\rangle_{B}\rightarrow \left|j\right\rangle_{F(B)} =  \frac{1}{\sqrt{N}}\sum^{N-1}_{k=0} e^{2\pi i j k /N}\left|k\right\rangle_{B}
\]

It is worth noting that the elements of the computational basis in a
\(2^n\) dimensional Hilbert space may be labeled by the number \(j\), or
by its representation as an \(n\)-bit binary string,

\[
j \equiv j_1 j_2 ... j_n \equiv \sum\limits_{\ell=1}^{n} j_\ell 2^{n-\ell}
\]

, i.e.~we can choose the basis states to be the tensor products of the
single-qubit basis states \(|j_\ell\rangle_B\), \(j_\ell\in\{0,1\}\).

We call the application of the DFT to a quantum state the quantum
Fourier transform (QFT), and it has a computationally efficient
alternative formula:

\[
QFT_n |j\rangle_{B} = \frac{1}{\sqrt{N}}\sum^{N-1}_{k=0} e^{2\pi i j k /N}\left|k\right\rangle_{B} = 
\frac{1}{2^{n/2}}\sum^{2^n-1}_{k=0} e^{2\pi i j k 2^{-n}}\left|k\right\rangle_{B} = 
\]

\[
= \frac{1}{2^{n/2}}\sum^{1}_{k_1=0}\sum^{1}_{k_2=0}...\sum^{1}_{k_n=0}
e^{2\pi i j \sum\limits_{\ell=1}^{n} k_\ell 2^{-\ell}}\left|k_1 k_2 ... k_n\right\rangle_{B} =
\]

\[
= \frac{1}{2^{n/2}}\sum^{1}_{k_1=0}e^{2\pi i j k_1 2^{-1}}|k_1\rangle_B
\sum^{1}_{k_2=0}e^{2\pi i j k_2 2^{-2}}|k_2\rangle_B... 
\sum^{1}_{k_n=0}e^{2\pi i j k_n 2^{-n}}|k_n\rangle_B = 
\frac{1}{2^{n/2}} \bigotimes\limits_{m=1}^{n}\sum\limits_{k_m=0}^1 e^{2\pi i j k_m 2^{-m}}|k_\ell\rangle_B = 
\]

\[
= \frac{1}{2^{n/2}} \bigotimes\limits_{m=1}^{n}\left(|0\rangle + e^{2\pi i j 2^{-m}}|1\rangle \right) = 
\frac{1}{2^{n/2}} \bigotimes\limits_{m=1}^{n}\left(|0\rangle + e^{2\pi i \sum\limits_{\ell=1}^n j_\ell 2^{n-\ell-m}}|1\rangle \right) 
\]

Since integer rotations in the complex plane don't add any phase, the
expression for the QFT finally simplifies to

\[
QFT_n |j\rangle_{B} = \frac{1}{2^{n/2}} \bigotimes\limits_{m=1}^{n}\left(|0\rangle + e^{2\pi i \sum\limits_{\ell=1}^{m} j_{n+\ell-m} 2^{-\ell}}|1\rangle \right) 
\]

\textbf{In the exercise notebook we're walked through programming the
QFT:}

{\color{teal}
The operation

\[\alpha |0\rangle + \beta | 1 \rangle \rightarrow  \alpha | 0 \rangle +\beta e^{2\pi i\ /\ 2^k} | 1 \rangle \]

Can be represented in matrix form as

\[R_k = \begin{pmatrix}
1 & 0 \\
0 & e^{2\pi i /2^k}
\end{pmatrix},\]

while one of the basic transformations in the Qiskit package is
\emph{PhaseGate}, with matrix representation

\[\text{PhaseGate}(\phi) = \begin{pmatrix}
1 & 0 \\
0 & e^{i\phi}
\end{pmatrix}.\]

Hence,

\textbf{Question 5.1.} Check that the following function implements the
\(R_k\) transformation:}


\textbf{Answer:}

    \begin{tcolorbox}[breakable, size=fbox, boxrule=1pt, pad at break*=1mm,colback=cellbackground, colframe=cellborder]
\prompt{In}{incolor}{7}{\boxspacing}
\begin{Verbatim}[commandchars=\\\{\}]
\PY{k}{def} \PY{n+nf}{R}\PY{p}{(}\PY{n}{k}\PY{p}{)}\PY{p}{:}
    \PY{k}{return} \PY{n}{PhaseGate}\PY{p}{(}\PY{p}{(}\PY{l+m+mf}{2.}\PY{o}{*}\PY{n}{np}\PY{o}{.}\PY{n}{pi}\PY{p}{)}\PY{o}{/}\PY{n+nb}{pow}\PY{p}{(}\PY{l+m+mi}{2}\PY{p}{,}\PY{n}{k}\PY{p}{)}\PY{p}{)}

\PY{c+c1}{\PYZsh{} Indeed, we can plot R(k) ∀ k with a parametrised circuit.}
\PY{c+c1}{\PYZsh{} We see the transformation is implemented as desired.}
\PY{k+kn}{from} \PY{n+nn}{qiskit}\PY{n+nn}{.}\PY{n+nn}{circuit} \PY{k+kn}{import} \PY{n}{Parameter}
\PY{n}{k} \PY{o}{=} \PY{n}{Parameter}\PY{p}{(}\PY{l+s+s1}{\PYZsq{}}\PY{l+s+s1}{k}\PY{l+s+s1}{\PYZsq{}}\PY{p}{)}
\PY{n}{ans} \PY{o}{=} \PY{n}{QuantumCircuit}\PY{p}{(}\PY{l+m+mi}{1}\PY{p}{)}
\PY{n}{ans}\PY{o}{.}\PY{n}{append}\PY{p}{(}\PY{n}{R}\PY{p}{(}\PY{n}{k}\PY{p}{)}\PY{p}{,}\PY{p}{[}\PY{l+m+mi}{0}\PY{p}{]}\PY{p}{)}
\PY{n}{ans}\PY{o}{.}\PY{n}{draw}\PY{p}{(}\PY{n}{output}\PY{o}{=}\PY{l+s+s1}{\PYZsq{}}\PY{l+s+s1}{mpl}\PY{l+s+s1}{\PYZsq{}}\PY{p}{)}
\end{Verbatim}
\end{tcolorbox}
 
            
\prompt{Out}{outcolor}{7}{}
    
    \begin{center}
    \adjustimage{max size={0.9\linewidth}{0.9\paperheight}}{ZapataFergusonM_Session2_files/ZapataFergusonM_Session2_19_0.png}
    \end{center}
    { \hspace*{\fill} \\}
    

    We can try this gate and compare it to what we should get for a couple
of values of \(k\)

    \begin{tcolorbox}[breakable, size=fbox, boxrule=1pt, pad at break*=1mm,colback=cellbackground, colframe=cellborder]
\prompt{In}{incolor}{31}{\boxspacing}
\begin{Verbatim}[commandchars=\\\{\}]
\PY{n}{maxK} \PY{o}{=} \PY{l+m+mi}{6}
\PY{n}{result} \PY{o}{=} \PY{p}{[}\PY{l+m+mi}{0}\PY{p}{]} \PY{o}{*} \PY{n}{maxK}
\PY{n}{expected} \PY{o}{=} \PY{p}{[}\PY{l+m+mi}{0}\PY{p}{]} \PY{o}{*} \PY{n}{maxK}

\PY{k}{for} \PY{n}{k} \PY{o+ow}{in} \PY{n+nb}{range}\PY{p}{(}\PY{n}{maxK}\PY{p}{)}\PY{p}{:}
  \PY{c+c1}{\PYZsh{} Rk circuit}
  \PY{n}{circ1} \PY{o}{=} \PY{n}{QuantumCircuit}\PY{p}{(}\PY{l+m+mi}{1}\PY{p}{)}
  \PY{n}{circ1}\PY{o}{.}\PY{n}{x}\PY{p}{(}\PY{l+m+mi}{0}\PY{p}{)} \PY{c+c1}{\PYZsh{} Initialise to |1\PYZgt{}}
  \PY{n}{circ1}\PY{o}{.}\PY{n}{append}\PY{p}{(}\PY{n}{R}\PY{p}{(}\PY{n}{k}\PY{p}{)}\PY{p}{,} \PY{p}{[}\PY{l+m+mi}{0}\PY{p}{]}\PY{p}{)} \PY{c+c1}{\PYZsh{} Apply Rk}
  \PY{n}{psi} \PY{o}{=} \PY{n}{Statevector}\PY{p}{(}\PY{n}{circ1}\PY{p}{)}
  \PY{n}{result}\PY{p}{[}\PY{n}{k}\PY{p}{]} \PY{o}{=} \PY{n}{psi}\PY{p}{[}\PY{l+m+mi}{1}\PY{p}{]}
  \PY{c+c1}{\PYZsh{} Expected phase}
  \PY{n}{expected}\PY{p}{[}\PY{n}{k}\PY{p}{]} \PY{o}{=} \PY{n}{np}\PY{o}{.}\PY{n}{exp}\PY{p}{(}\PY{l+m+mi}{2}\PY{o}{*}\PY{n}{np}\PY{o}{.}\PY{n}{pi}\PY{o}{*}\PY{p}{(}\PY{l+m+mi}{1}\PY{n}{j}\PY{p}{)} \PY{o}{/} \PY{l+m+mi}{2}\PY{o}{*}\PY{o}{*}\PY{n}{k}\PY{p}{)}

\PY{n+nb}{print}\PY{p}{(}
\PY{l+s+sa}{f}\PY{l+s+s1}{\PYZsq{}\PYZsq{}\PYZsq{}}
\PY{l+s+s1}{What we got        }\PY{l+s+si}{\PYZob{}}\PY{n}{np}\PY{o}{.}\PY{n}{array}\PY{p}{(}\PY{n}{result}\PY{p}{)}\PY{l+s+si}{\PYZcb{}}
\PY{l+s+s1}{What we should get }\PY{l+s+si}{\PYZob{}}\PY{n}{np}\PY{o}{.}\PY{n}{array}\PY{p}{(}\PY{n}{expected}\PY{p}{)}\PY{l+s+si}{\PYZcb{}}\PY{l+s+s1}{\PYZsq{}\PYZsq{}\PYZsq{}}\PY{p}{)}
\end{Verbatim}
\end{tcolorbox}

    \begin{Verbatim}[commandchars=\\\{\}]

What we got        [ 1.00e+00-2.45e-16j -1.00e+00+1.22e-16j  6.12e-17+1.00e+00j
7.07e-01+7.07e-01j  9.24e-01+3.83e-01j  9.81e-01+1.95e-01j]
What we should get [ 1.00e+00-2.45e-16j -1.00e+00+1.22e-16j  6.12e-17+1.00e+00j
7.07e-01+7.07e-01j  9.24e-01+3.83e-01j  9.81e-01+1.95e-01j]
    \end{Verbatim}

    We can appreciate the \(R_k\) gate performs the previously described
transformation perfectly, although there may be some precision error in
both calculations.

\textbf{In sections 5.1-5.2 we're walked through programming a
3-register QFT.} {\color{teal}At the step at which we accomplish

\[| j_1j_2j_3\rangle \rightarrow | \psi_1\rangle = \frac{1}{\sqrt{2}}\left(\left|0\right\rangle+e^{2\pi i (j_1/2 +j_2/4)}\left|1\right\rangle\right)\otimes | j_2j_3\rangle.\]}


    \begin{tcolorbox}[breakable, size=fbox, boxrule=1pt, pad at break*=1mm,colback=cellbackground, colframe=cellborder]
\prompt{In}{incolor}{9}{\boxspacing}
\begin{Verbatim}[commandchars=\\\{\}]
\PY{n}{qft\PYZus{}circuit} \PY{o}{=} \PY{n}{QuantumCircuit}\PY{p}{(}\PY{l+m+mi}{3}\PY{p}{,} \PY{n}{name} \PY{o}{=} \PY{l+s+s1}{\PYZsq{}}\PY{l+s+s1}{QFT}\PY{l+s+s1}{\PYZsq{}}\PY{p}{)}
\PY{n}{qft\PYZus{}circuit}\PY{o}{.}\PY{n}{h}\PY{p}{(}\PY{l+m+mi}{2}\PY{p}{)}
\PY{n}{qft\PYZus{}circuit}\PY{o}{.}\PY{n}{draw}\PY{p}{(}\PY{p}{)}
\PY{n}{qft\PYZus{}circuit}\PY{o}{.}\PY{n}{append}\PY{p}{(}\PY{n}{R}\PY{p}{(}\PY{l+m+mi}{2}\PY{p}{)}\PY{o}{.}\PY{n}{control}\PY{p}{(}\PY{p}{)}\PY{p}{,}\PY{p}{[}\PY{l+m+mi}{2}\PY{p}{,}\PY{l+m+mi}{1}\PY{p}{]}\PY{p}{)}
\PY{n}{qft\PYZus{}circuit}\PY{o}{.}\PY{n}{draw}\PY{p}{(}\PY{n}{output}\PY{o}{=}\PY{l+s+s1}{\PYZsq{}}\PY{l+s+s1}{mpl}\PY{l+s+s1}{\PYZsq{}}\PY{p}{)}
\end{Verbatim}
\end{tcolorbox}
 
            
\prompt{Out}{outcolor}{9}{}
    
    \begin{center}
    \adjustimage{max size={0.9\linewidth}{0.9\paperheight}}{ZapataFergusonM_Session2_files/ZapataFergusonM_Session2_24_0.png}
    \end{center}
    { \hspace*{\fill} \\}
    

\textcolor{teal}{\textbf{Question 5.2} Write the state resulting from applying that
transformation to the initial ket \(| 010\rangle\). Check, with the
Statevector class that the circuit generates the correct output.}
    

\textbf{Answer:} We may compute the resulting state by hand. It is
\(\frac{1}{\sqrt{2}}(|0\rangle+i|1\rangle)\otimes|10\rangle\) or, in the
binary basis, \(\frac{1}{\sqrt{2}}(|2\rangle+i|6\rangle)\).

    \begin{tcolorbox}[breakable, size=fbox, boxrule=1pt, pad at break*=1mm,colback=cellbackground, colframe=cellborder]
\prompt{In}{incolor}{32}{\boxspacing}
\begin{Verbatim}[commandchars=\\\{\}]
\PY{n}{init\PYZus{}state} \PY{o}{=} \PY{n}{np}\PY{o}{.}\PY{n}{zeros}\PY{p}{(}\PY{l+m+mi}{8}\PY{p}{)}
\PY{n}{init\PYZus{}state}\PY{p}{[}\PY{l+m+mi}{2}\PY{p}{]} \PY{o}{=} \PY{l+m+mi}{1}
\PY{n}{qc} \PY{o}{=} \PY{n}{QuantumCircuit}\PY{p}{(}\PY{l+m+mi}{3}\PY{p}{)}
\PY{n}{qc}\PY{o}{.}\PY{n}{initialize}\PY{p}{(}\PY{n}{init\PYZus{}state}\PY{p}{,} \PY{n+nb}{range}\PY{p}{(}\PY{l+m+mi}{3}\PY{p}{)}\PY{p}{)}
\PY{n}{qc}\PY{o}{.}\PY{n}{append}\PY{p}{(}\PY{n}{qft\PYZus{}circuit}\PY{p}{,} \PY{n+nb}{range}\PY{p}{(}\PY{l+m+mi}{3}\PY{p}{)}\PY{p}{)}
\PY{n}{phians} \PY{o}{=} \PY{n}{Statevector}\PY{p}{(}\PY{n}{qc}\PY{p}{)}

\PY{c+c1}{\PYZsh{} By printing the result, this checks out}
\PY{n}{phians}
\end{Verbatim}
\end{tcolorbox}

    \begin{Verbatim}[commandchars=\\\{\}]
Statevector([0.00e+00+0.j  , 0.00e+00+0.j  , 7.07e-01+0.j  ,
             0.00e+00+0.j  , 0.00e+00+0.j  , 0.00e+00+0.j  ,
             4.33e-17+0.71j, 0.00e+00+0.j  ],
            dims=(2, 2, 2))
    \end{Verbatim}

    \paragraph{\texorpdfstring{ 5.3 QFT full
exercise}{ 5.3 QFT full exercise}}\label{qft-full-exercise}

{\color{teal}Please implement now the Quantum Fourier Transform as a gate acting over
5 qubits. Apply this gate to the following states.

\[ |\phi_1\rangle = \frac{1}{2^{5/2}}\left(\left|0\right\rangle+\left|1\right\rangle\right)\left(\left|0\right\rangle+\left|1\right\rangle\right)\left(\left|0\right\rangle+\left|1\right\rangle\right)\left(\left|0\right\rangle+\left|1\right\rangle\right)\left(\left|0\right\rangle-\left|1\right\rangle\right)\]
\[ |\phi_2\rangle = \frac{1}{2^{5/2}}\left(\left|0\right\rangle+\left|1\right\rangle\right)\left(\left|0\right\rangle+\left|1\right\rangle\right)\left(\left|0\right\rangle+\left|1\right\rangle\right)\left(\left|0\right\rangle-\left|1\right\rangle\right)\left(\left|0\right\rangle+\left|1\right\rangle\right)\]
\[ |\phi_3\rangle = \frac{1}{2^{5/2}}\left(\left|0\right\rangle+\left|1\right\rangle\right)\left(\left|0\right\rangle+\left|1\right\rangle\right)\left(\left|0\right\rangle-\left|1\right\rangle\right)\left(\left|0\right\rangle+\left|1\right\rangle\right)\left(\left|0\right\rangle+\left|1\right\rangle\right)\]
\[ |\phi_4\rangle = \frac{1}{2^{5/2}}\left(\left|0\right\rangle+\left|1\right\rangle\right)\left(\left|0\right\rangle-\left|1\right\rangle\right)\left(\left|0\right\rangle+\left|1\right\rangle\right)\left(\left|0\right\rangle+\left|1\right\rangle\right)\left(\left|0\right\rangle+\left|1\right\rangle\right)\]

Briefly comment on the results.}

\textbf{Answer:} A QFT circuit implements the following transformation:

\[
|j_1j_2...j_n\rangle_B\rightarrow \frac{1}{2^{n/2}}
\left(|0\rangle + e^{2\pi i \frac{j_n}{2}}|1\rangle \right) 
\left(|0\rangle + e^{2\pi i (\frac{j_{n-1}}{2}+\frac{j_n}{4})}|1\rangle \right)
... 
\left(|0\rangle + e^{2\pi i (\frac{j_1}{2}+...+\frac{j_n}{2^n})}|1\rangle \right) 
\]

As we can see, the value of qubit 1 (\(j_n\)) appears in every term (and
\(j_{n-1}\) appears in every term up to the penultimate, so on and so
forth). This implies we have to act on the least significant bit last,
so as to not modify \(j_n\) while we still need its value (likewise with
\(j_{n-1}\), etc.).

On the other hand, the least significant half of the transformed bits
require information from bits more significant than them, so we have to
apply the QFT from least significant bit to most significant bit.

The way to solve this conundrum is to store the transformed qubits in
reverse order, and to afterwards apply a series of swaps to recover the
proper ordering. We can apply the required phases with Hadamard and
controlled \(R_k\) gates.

    \begin{tcolorbox}[breakable, size=fbox, boxrule=1pt, pad at break*=1mm,colback=cellbackground, colframe=cellborder]
\prompt{In}{incolor}{11}{\boxspacing}
\begin{Verbatim}[commandchars=\\\{\}]
\PY{c+c1}{\PYZsh{} First, it is actually easier and more convenient to implement an N register QFT}

\PY{k}{def} \PY{n+nf}{QFT}\PY{p}{(}\PY{n}{N}\PY{p}{)}\PY{p}{:}
    \PY{n}{qc} \PY{o}{=} \PY{n}{QuantumCircuit}\PY{p}{(}\PY{n}{N}\PY{p}{)}
    \PY{k}{for} \PY{n}{i} \PY{o+ow}{in} \PY{n}{np}\PY{o}{.}\PY{n}{arange}\PY{p}{(}\PY{n}{N}\PY{o}{\PYZhy{}}\PY{l+m+mi}{1}\PY{p}{,} \PY{o}{\PYZhy{}}\PY{l+m+mi}{1}\PY{p}{,} \PY{o}{\PYZhy{}}\PY{l+m+mi}{1}\PY{p}{)}\PY{p}{:}
        \PY{n}{qc}\PY{o}{.}\PY{n}{h}\PY{p}{(}\PY{n}{i}\PY{p}{)}
        \PY{k}{for} \PY{n}{j} \PY{o+ow}{in} \PY{n+nb}{range}\PY{p}{(}\PY{n}{i}\PY{o}{\PYZhy{}}\PY{l+m+mi}{1}\PY{p}{,} \PY{o}{\PYZhy{}}\PY{l+m+mi}{1}\PY{p}{,} \PY{o}{\PYZhy{}}\PY{l+m+mi}{1}\PY{p}{)}\PY{p}{:}
            \PY{n}{qc}\PY{o}{.}\PY{n}{append}\PY{p}{(}\PY{n}{R}\PY{p}{(}\PY{n}{i}\PY{o}{\PYZhy{}}\PY{n}{j}\PY{o}{+}\PY{l+m+mi}{1}\PY{p}{)}\PY{o}{.}\PY{n}{control}\PY{p}{(}\PY{p}{)}\PY{p}{,}\PY{p}{[}\PY{n}{i}\PY{p}{,}\PY{n}{j}\PY{p}{]}\PY{p}{)}
    \PY{k}{for} \PY{n}{i} \PY{o+ow}{in} \PY{n+nb}{range}\PY{p}{(}\PY{n}{N}\PY{o}{/}\PY{o}{/}\PY{l+m+mi}{2}\PY{p}{)}\PY{p}{:}
        \PY{n}{qc}\PY{o}{.}\PY{n}{swap}\PY{p}{(}\PY{n}{i}\PY{p}{,} \PY{n}{N}\PY{o}{\PYZhy{}}\PY{n}{i}\PY{o}{\PYZhy{}}\PY{l+m+mi}{1}\PY{p}{)}
    \PY{k}{return} \PY{n}{qc}

\PY{n}{QFT}\PY{p}{(}\PY{l+m+mi}{5}\PY{p}{)}\PY{o}{.}\PY{n}{draw}\PY{p}{(}\PY{n}{output}\PY{o}{=}\PY{l+s+s1}{\PYZsq{}}\PY{l+s+s1}{mpl}\PY{l+s+s1}{\PYZsq{}}\PY{p}{)}
\end{Verbatim}
\end{tcolorbox}
 
            
\prompt{Out}{outcolor}{11}{}
    
    \begin{center}
    \adjustimage{max size={0.9\linewidth}{0.9\paperheight}}{ZapataFergusonM_Session2_files/ZapataFergusonM_Session2_28_0.png}
    \end{center}
    { \hspace*{\fill} \\}
    

    \begin{tcolorbox}[breakable, size=fbox, boxrule=1pt, pad at break*=1mm,colback=cellbackground, colframe=cellborder]
\prompt{In}{incolor}{12}{\boxspacing}
\begin{Verbatim}[commandchars=\\\{\}]
\PY{c+c1}{\PYZsh{} Next, let\PYZsq{}s notice the kets are}
\PY{c+c1}{\PYZsh{} |phi1\PYZgt{} = |+\PYZgt{}|+\PYZgt{}|+\PYZgt{}|+\PYZgt{}|\PYZhy{}\PYZgt{}}
\PY{c+c1}{\PYZsh{} |phi2\PYZgt{} = |+\PYZgt{}|+\PYZgt{}|+\PYZgt{}|\PYZhy{}\PYZgt{}|+\PYZgt{}}
\PY{c+c1}{\PYZsh{} |phi3\PYZgt{} = |+\PYZgt{}|+\PYZgt{}|\PYZhy{}\PYZgt{}|+\PYZgt{}|+\PYZgt{}}
\PY{c+c1}{\PYZsh{} |phi4\PYZgt{} = |+\PYZgt{}|\PYZhy{}\PYZgt{}|+\PYZgt{}|+\PYZgt{}|+\PYZgt{}}
\PY{c+c1}{\PYZsh{} So it will be easiest to prepare the initial states with Hadamard gates as follows:}


\PY{k}{def} \PY{n+nf}{phi}\PY{p}{(}\PY{n}{n}\PY{p}{)}\PY{p}{:}
    \PY{n}{qc} \PY{o}{=} \PY{n}{QuantumCircuit}\PY{p}{(}\PY{l+m+mi}{5}\PY{p}{)}
    \PY{n}{qc}\PY{o}{.}\PY{n}{x}\PY{p}{(}\PY{n}{n}\PY{p}{)}
    \PY{k}{for} \PY{n}{i} \PY{o+ow}{in} \PY{n+nb}{range}\PY{p}{(}\PY{l+m+mi}{5}\PY{p}{)}\PY{p}{:}
        \PY{n}{qc}\PY{o}{.}\PY{n}{h}\PY{p}{(}\PY{n}{i}\PY{p}{)}
    \PY{k}{return} \PY{n}{qc}

\PY{n}{phi}\PY{p}{(}\PY{l+m+mi}{0}\PY{p}{)}\PY{o}{.}\PY{n}{draw}\PY{p}{(}\PY{n}{output}\PY{o}{=}\PY{l+s+s1}{\PYZsq{}}\PY{l+s+s1}{mpl}\PY{l+s+s1}{\PYZsq{}}\PY{p}{)}
\end{Verbatim}
\end{tcolorbox}
 
            
\prompt{Out}{outcolor}{12}{}
    
    \begin{center}
    \adjustimage{max size={0.9\linewidth}{0.9\paperheight}}{ZapataFergusonM_Session2_files/ZapataFergusonM_Session2_29_0.png}
    \end{center}
    { \hspace*{\fill} \\}
    

    \begin{tcolorbox}[breakable, size=fbox, boxrule=1pt, pad at break*=1mm,colback=cellbackground, colframe=cellborder]
\prompt{In}{incolor}{13}{\boxspacing}
\begin{Verbatim}[commandchars=\\\{\}]
\PY{c+c1}{\PYZsh{} Next we must create a 5\PYZhy{}register QFT gate}
\PY{n}{qft5} \PY{o}{=} \PY{n}{QFT}\PY{p}{(}\PY{l+m+mi}{5}\PY{p}{)}\PY{o}{.}\PY{n}{to\PYZus{}gate}\PY{p}{(}\PY{p}{)}
\PY{c+c1}{\PYZsh{} And apply it to each prepared state}

\PY{n}{prec} \PY{o}{=} \PY{l+m+mi}{2} \PY{c+c1}{\PYZsh{} Decimal precision}
\PY{n}{phihat\PYZus{}q} \PY{o}{=} \PY{n}{np}\PY{o}{.}\PY{n}{empty}\PY{p}{(}\PY{p}{(}\PY{l+m+mi}{4}\PY{p}{,} \PY{l+m+mi}{32}\PY{p}{)}\PY{p}{,} \PY{n}{dtype}\PY{o}{=}\PY{n}{np}\PY{o}{.}\PY{n}{cdouble}\PY{p}{)}
\PY{n}{ket\PYZus{}q} \PY{o}{=} \PY{p}{[}\PY{p}{]}
\PY{k}{for} \PY{n}{k} \PY{o+ow}{in} \PY{n+nb}{range}\PY{p}{(}\PY{l+m+mi}{4}\PY{p}{)}\PY{p}{:}
    \PY{n}{qc} \PY{o}{=} \PY{n}{phi}\PY{p}{(}\PY{n}{k}\PY{p}{)}
    \PY{c+c1}{\PYZsh{} Quantum Fourier transform}
    \PY{n}{qc}\PY{o}{.}\PY{n}{append}\PY{p}{(}\PY{n}{qft5}\PY{p}{,} \PY{n+nb}{range}\PY{p}{(}\PY{l+m+mi}{5}\PY{p}{)}\PY{p}{)}
    \PY{n}{phihat\PYZus{}q}\PY{p}{[}\PY{n}{k}\PY{p}{]} \PY{o}{=} \PY{n}{Statevector}\PY{p}{(}\PY{n}{qc}\PY{p}{)}
    \PY{c+c1}{\PYZsh{} Format ket as a linear combination of basis vectors (string)}
    \PY{n}{idx} \PY{o}{=} \PY{n}{np}\PY{o}{.}\PY{n}{where}\PY{p}{(}\PY{n}{np}\PY{o}{.}\PY{n}{abs}\PY{p}{(}\PY{n}{phihat\PYZus{}q}\PY{p}{[}\PY{n}{k}\PY{p}{]}\PY{p}{)} \PY{o}{\PYZgt{}} \PY{l+m+mi}{10}\PY{o}{*}\PY{o}{*}\PY{o}{\PYZhy{}}\PY{n}{prec}\PY{p}{)}
    \PY{n}{ket\PYZus{}q}\PY{o}{.}\PY{n}{append}\PY{p}{(}\PY{l+s+s1}{\PYZsq{}}\PY{l+s+s1}{+}\PY{l+s+s1}{\PYZsq{}}\PY{o}{.}\PY{n}{join}\PY{p}{(}\PY{p}{[}\PY{l+s+sa}{f}\PY{l+s+s1}{\PYZsq{}}\PY{l+s+s1}{(}\PY{l+s+si}{\PYZob{}}\PY{n}{phihat\PYZus{}q}\PY{p}{[}\PY{n}{k}\PY{p}{]}\PY{p}{[}\PY{n}{j}\PY{p}{]}\PY{l+s+si}{:}\PY{l+s+s1}{.}\PY{l+s+si}{\PYZob{}}\PY{n}{prec}\PY{l+s+si}{\PYZcb{}}\PY{l+s+s1}{f}\PY{l+s+si}{\PYZcb{}}\PY{l+s+s1}{)|}\PY{l+s+si}{\PYZob{}}\PY{n}{j}\PY{l+s+si}{\PYZcb{}}\PY{l+s+s1}{\PYZgt{}}\PY{l+s+s1}{\PYZsq{}} \PY{k}{for} \PY{n}{j} \PY{o+ow}{in} \PY{n}{idx}\PY{p}{[}\PY{l+m+mi}{0}\PY{p}{]}\PY{p}{]}\PY{p}{)}\PY{p}{)}

\PY{c+c1}{\PYZsh{} Print out Fourier transformed kets}
\PY{n}{ket\PYZus{}q}
\end{Verbatim}
\end{tcolorbox}

            \begin{tcolorbox}[breakable, size=fbox, boxrule=.5pt, pad at break*=1mm, opacityfill=0]
\prompt{Out}{outcolor}{13}{\boxspacing}
\begin{Verbatim}[commandchars=\\\{\}]
['(1.00+0.00j)|16>',
 '(0.50+0.50j)|8>+(0.50-0.50j)|24>',
 '(0.25+0.60j)|4>+(0.25+0.10j)|12>+(0.25-0.10j)|20>+(0.25-0.60j)|28>',
 '(0.12+0.63j)|2>+(0.12+0.19j)|6>+(0.12+0.08j)|10>+(0.12+0.02j)|14>+(0.12-
0.02j)|18>+(0.12-0.08j)|22>+(0.12-0.19j)|26>+(0.12-0.63j)|30>']
\end{Verbatim}
\end{tcolorbox}
        
    Let \(QFT_5\) be the quantum Fourier transform of 5 qubits, we obtained
the following results:

\begin{itemize}
    \item \(QFT_5|\phi_1\rangle = |16⟩\) 
    \item \(QFT_5|\phi_2\rangle = \frac{1}{2}(1+i)|8⟩+\frac{1}{2}(1-i)|24⟩\) 
    \item \(QFT_5|\phi_3\rangle = (0.25+0.60i)|4⟩+(0.25+0.10 i)|12⟩+(0.25-0.10 i) |20⟩ + (0.25-0.60i)|28⟩\)
    \item \(QFT_5|\phi_4\rangle = (0.12+0.63i)|2⟩+(0.12+0.19i)|6⟩+(0.12+0.08i)|10⟩+(0.12+0.02i)|14⟩+(0.12-0.02i)|18⟩+(0.12-0.08i)|22⟩+(0.12-0.19i)|26⟩+(0.12-0.63i)|30⟩\) 
\end{itemize}

The QFT is so efficient because the number of quantum gates it requires
for \(n\) qubits scales like \(O(n^2)\), whereas the dimension of the
Hilbert space scales like \(N=2^n\). This can be compared to best DFT
algorithm, the Fast Fourier Transform (FFT), which has complexity
\(O(N\log N)=O(n2^n)\) --- that is, the QFT has an exponential advantage
over the classical DFT.

We could also check whether we obtain the same result via classical
computing methods (bearing in mind that the convention for Fourier /
inverse Fourier transform in signal processing is opposite that of
quantum computing).

    \begin{tcolorbox}[breakable, size=fbox, boxrule=1pt, pad at break*=1mm,colback=cellbackground, colframe=cellborder]
\prompt{In}{incolor}{14}{\boxspacing}
\begin{Verbatim}[commandchars=\\\{\}]
\PY{n}{phihat\PYZus{}c} \PY{o}{=} \PY{n}{np}\PY{o}{.}\PY{n}{empty}\PY{p}{(}\PY{p}{(}\PY{l+m+mi}{4}\PY{p}{,} \PY{l+m+mi}{32}\PY{p}{)}\PY{p}{,} \PY{n}{dtype}\PY{o}{=}\PY{n}{np}\PY{o}{.}\PY{n}{cdouble}\PY{p}{)}
\PY{n}{ket\PYZus{}c} \PY{o}{=} \PY{p}{[}\PY{p}{]}
\PY{k}{for} \PY{n}{k} \PY{o+ow}{in} \PY{n+nb}{range}\PY{p}{(}\PY{l+m+mi}{4}\PY{p}{)}\PY{p}{:}
    \PY{n}{qc} \PY{o}{=} \PY{n}{phi}\PY{p}{(}\PY{n}{k}\PY{p}{)}
    \PY{c+c1}{\PYZsh{} Classical Fourier transform}
    \PY{n}{phihat\PYZus{}c}\PY{p}{[}\PY{n}{k}\PY{p}{]} \PY{o}{=} \PY{n}{np}\PY{o}{.}\PY{n}{fft}\PY{o}{.}\PY{n}{ifft}\PY{p}{(}\PY{n}{Statevector}\PY{p}{(}\PY{n}{qc}\PY{p}{)}\PY{p}{,} \PY{n}{norm}\PY{o}{=}\PY{l+s+s1}{\PYZsq{}}\PY{l+s+s1}{ortho}\PY{l+s+s1}{\PYZsq{}}\PY{p}{)}
    \PY{c+c1}{\PYZsh{} Format ket as a linear combination of basis vectors (string)}
    \PY{n}{idx} \PY{o}{=} \PY{n}{np}\PY{o}{.}\PY{n}{where}\PY{p}{(}\PY{n}{np}\PY{o}{.}\PY{n}{abs}\PY{p}{(}\PY{n}{phihat\PYZus{}c}\PY{p}{[}\PY{n}{k}\PY{p}{]}\PY{p}{)} \PY{o}{\PYZgt{}} \PY{l+m+mi}{10}\PY{o}{*}\PY{o}{*}\PY{o}{\PYZhy{}}\PY{n}{prec}\PY{p}{)}
    \PY{n}{ket\PYZus{}c}\PY{o}{.}\PY{n}{append}\PY{p}{(}\PY{l+s+s1}{\PYZsq{}}\PY{l+s+s1}{+}\PY{l+s+s1}{\PYZsq{}}\PY{o}{.}\PY{n}{join}\PY{p}{(}\PY{p}{[}\PY{l+s+sa}{f}\PY{l+s+s1}{\PYZsq{}}\PY{l+s+s1}{(}\PY{l+s+si}{\PYZob{}}\PY{n}{phihat\PYZus{}c}\PY{p}{[}\PY{n}{k}\PY{p}{]}\PY{p}{[}\PY{n}{j}\PY{p}{]}\PY{l+s+si}{:}\PY{l+s+s1}{.}\PY{l+s+si}{\PYZob{}}\PY{n}{prec}\PY{l+s+si}{\PYZcb{}}\PY{l+s+s1}{f}\PY{l+s+si}{\PYZcb{}}\PY{l+s+s1}{)|}\PY{l+s+si}{\PYZob{}}\PY{n}{j}\PY{l+s+si}{\PYZcb{}}\PY{l+s+s1}{\PYZgt{}}\PY{l+s+s1}{\PYZsq{}} \PY{k}{for} \PY{n}{j} \PY{o+ow}{in} \PY{n}{idx}\PY{p}{[}\PY{l+m+mi}{0}\PY{p}{]}\PY{p}{]}\PY{p}{)}\PY{p}{)}

\PY{c+c1}{\PYZsh{} Print out Fourier transformed kets}
\PY{n}{ket\PYZus{}c}
\end{Verbatim}
\end{tcolorbox}

            \begin{tcolorbox}[breakable, size=fbox, boxrule=.5pt, pad at break*=1mm, opacityfill=0]
\prompt{Out}{outcolor}{14}{\boxspacing}
\begin{Verbatim}[commandchars=\\\{\}]
['(1.00+0.00j)|16>',
 '(0.50+0.50j)|8>+(0.50-0.50j)|24>',
 '(0.25+0.60j)|4>+(0.25+0.10j)|12>+(0.25-0.10j)|20>+(0.25-0.60j)|28>',
 '(0.12+0.63j)|2>+(0.12+0.19j)|6>+(0.12+0.08j)|10>+(0.12+0.02j)|14>+(0.12-
0.02j)|18>+(0.12-0.08j)|22>+(0.12-0.19j)|26>+(0.12-0.63j)|30>']
\end{Verbatim}
\end{tcolorbox}
        
    Looks promising! Just so our eyes don't deceive us, let's compare both
calculations programatically.

    \begin{tcolorbox}[breakable, size=fbox, boxrule=1pt, pad at break*=1mm,colback=cellbackground, colframe=cellborder]
\prompt{In}{incolor}{15}{\boxspacing}
\begin{Verbatim}[commandchars=\\\{\}]
\PY{n}{np}\PY{o}{.}\PY{n}{max}\PY{p}{(}\PY{n}{np}\PY{o}{.}\PY{n}{abs}\PY{p}{(}\PY{n}{phihat\PYZus{}q} \PY{o}{\PYZhy{}} \PY{n}{phihat\PYZus{}c}\PY{p}{)}\PY{p}{)}
\end{Verbatim}
\end{tcolorbox}

            \begin{tcolorbox}[breakable, size=fbox, boxrule=.5pt, pad at break*=1mm, opacityfill=0]
\prompt{Out}{outcolor}{15}{\boxspacing}
\begin{Verbatim}[commandchars=\\\{\}]
3.3306690738754696e-16
\end{Verbatim}
\end{tcolorbox}
        
    As we can see, the discrepancy between both calculations is at most on
the order of \(10^{-16}\), which could be attributed to machine error.


    % Add a bibliography block to the postdoc
    
    
    
\end{document}
