
\documentclass[11pt]{article}
\usepackage[utf8]{inputenc}
\usepackage{natbib}
%\usepackage[spanish]{babel}
\usepackage{graphicx}

\usepackage{relsize}
\usepackage{float}
\usepackage[papersize={210mm,297mm},tmargin=20mm,bmargin=20mm,lmargin=20mm,rmargin=20mm]{geometry}
\usepackage{subcaption}
\usepackage{multirow}
\usepackage[usenames]{color}
\usepackage[dvipsnames]{xcolor}
%\usepackagefont{caption}
\usepackage[font={color=BlueViolet}]{caption}
%\usepackage[dvipsnames]{xcolor}
\usepackage{amsmath}
\usepackage{mathrsfs}
\usepackage[mathscr]{euscript}
\usepackage{array}
\usepackage{soul}
\usepackage{amssymb}
\usepackage{amsthm}
\usepackage{mathtools}
\usepackage{stmaryrd}
\usepackage{xfrac}
\usepackage{wrapfig} 
\usepackage{tikz}
\usepackage{pgfplots}
\pgfplotsset{compat=1.15}
\usepackage{mathrsfs}
\usepackage{pstricks-add}
\usetikzlibrary{arrows}
\usepackage{physics}
\usepackage{multicol}
\usepackage{stmaryrd}
\usepackage{cancel}
\usepackage{enumerate,hhline}
\usepackage{enumitem}
\usepackage{tikz-3dplot}
\usepackage{tikz-cd}
\usepackage{multirow}
\usepackage{colortbl}
\setlength{\parskip}{4mm}
\setlength{\parindent}{0mm}
\usepackage[RPvoltages]{circuitikz}
\usepackage{tikz}
\usetikzlibrary{quantikz}
\usepackage{nicefrac}
\usepackage{halloweenmath}
\usepackage[T1]{fontenc}
\usepackage{realhats}
\usepackage{amscd, amsfonts}
\usepackage{parskip}
\usepackage{graphicx}
\usepackage{hyperref}
\usepackage{indentfirst}
\usepackage{relsize}
\renewcommand{\figurename}{Figura}
\renewcommand{\tablename}{Tabla}



%\oddsidemargin 0pt
%\evensidemargin 0pt
%\marginparwidth 40pt
%\marginparsep 10pt
%\topmargin -20pt
%\headsep 10pt
%\textheight 27cm
%\textwidth 17cm
%\linespread{1.2}

\parskip 10pt
\parindent 20pt


\begin{document}
\begin{center}
            \textsc{Universidad Complutense de Madrid}\\
            \huge\textsc{Proyecto de Innovación Educativa en
Computación Cuántica}\\
            \LARGE Curso 2023-2024\\ \vspace{.2cm}
            \large Grupo 1 Viernes\\ \vspace{.2cm}
            \normalsize Autores:\\ \vspace{-0cm}
            \begin{multicols}{3}
             Ruth Macías Asenjo \\  Eduardo Martínez Valdelvira\\
             Fernando Poblet Estévez \\ Miguel González Cuenca\\
              Juan Antonio Viñuelas Iniesta \\ Martín Zapata Ferguson
            \end{multicols}
\end{center}
\begin{enumerate}[label*=\arabic*.]
\setcounter{enumi}{1}
    \item Exercises
    \begin{enumerate}[label*=\arabic*.]
        \item What are the following quantum gate products equivalent to?
        \begin{enumerate}[label=(\alph*)]
            \item $H\cdot H = I$
            \item $X\cdot X = I$
            \item $Y\cdot Y = I$
            \item $Z\cdot Z = I$
            \item $X^{1/2}\cdot X^{1/2} = X$
            \item $X^{1/4}\cdot X^{1/4} = X^{1/2} $

            $$
            X^{1/2} = \pm\frac{1}{2}
            \begin{pmatrix}
             1\pm i & 1\mp i \\
             1\mp i & 1\pm i
            \end{pmatrix}
            $$

            However, in QUIRK's implementation

            $$
            X^{1/2} = \frac{1}{2}
            \begin{pmatrix}
            1+i & 1-i \\
            1-i & 1+i
            \end{pmatrix}
            $$
            
            \item $H\cdot Y \cdot H = -Y$
        \end{enumerate}
        \item Remember that a \textit{global} phase change does not alter a qubit. Build a quantum circuit that, from
a single qubit initialized to $\ket{0}$, generates the following 1-qubit states:\begin{enumerate}[label=(\alph*)]
            \item $\ket{1} = X\ket{0}$
            \item $\ket{+}=\frac{1}{\sqrt{2}}(\ket{0}+\ket{1}) = H \ket{0}$
            \item $\ket{-}=\frac{1}{\sqrt{2}}(\ket{0}-\ket{1}) = H\cdot X\ket{0}$
            \item $\ket{R}=\frac{1}{\sqrt{2}}(\ket{0}+i\ket{1}) = H\cdot X^{1/2}\ket{0} $
            \item $\ket{L}=\frac{1}{\sqrt{2}}(\ket{0}-i\ket{1}) = H\cdot X^{1/2}\cdot X\ket{0} $
        \end{enumerate}
    \end{enumerate}
    \item  Unitary transformations
    \begin{enumerate}[label*=\arabic*.]
        \item  (conceptual, not on QUIRK) Which of the following operations is/are NOT unitary, if any?
        Not unitary: a), c), d), e), f)
        \begin{enumerate}[label=(\alph*)]
            \item Sorting a list of elements.

            A sorted list could have been given in any initial configuration, so this operation is not invertible and thus not unitary.
            
            \item Applying a certain permutation.

            Say we have a quantum state $\ket{\psi}=\ket{\psi}_1\ket{\psi}_2...\ket{\psi}_n$ and a permutation operator which acts as $P\ket{\psi}=\ket{\psi}_{P_1}\ket{\psi}_{P_2}...\ket{\psi}_{P_n}$, $\{P_i\}_{i=1}^n\in S_n$. It is clear that it preserves the inner product, since

            $$
            \bra{\phi}P^\dagger P\ket{\psi} = \bra{\phi}_{P_1}\ket{\psi}_{P_1}\bra{\phi}_{P_2}\ket{\psi}_{P_2}...\bra{\phi}_{P_n}\ket{\psi}_{P_n} = \bra{\phi}_1\ket{\psi}_1\bra{\phi}_2\ket{\psi}_2...\bra{\phi}_n\ket{\psi}_n = \bra{\phi}\ket{\psi}
            $$

            Therefore the permutation operator $P$ is unitary.
            
            \item Adding two elements ($a$ and $b$) keeping none of them.

            Any two numbers in a field can add to another number in said field, therefore the operation is not invertible and thus not unitary.
            
            \item Adding two elements ($a$ and $b$) keeping one and only one of them

            This operation may be invertible, however in matrix form:

            $$
            \begin{pmatrix}
            1 & 1 \\
            0 & 1
            \end{pmatrix}
            \begin{pmatrix}
                a \\ b
            \end{pmatrix} =
            \begin{pmatrix}
                a+b \\ b
            \end{pmatrix}
            $$

            The matrix inverse of this operation is not its adjoint, in fact it is

            $$
            \begin{pmatrix}
                1 & -1\\
                0 & 1
            \end{pmatrix}
            $$
            
            \item Symmetryzing two states. That is, if we have $\ket{i} \ket{j}$, generating $\frac{1}{\sqrt{2}}(\ket{i}\ket{j}+\ket{j}\ket{i})$.

            The symmetrisation operator $P_S=\frac{1}{2}\sum\limits_{\pi\in S_2}\pi$ is a projector to the subspace of symmetric states and thus not invertible. Originally $\ket{i}\ket{j}$ could have had an unknown antisymmetric component which symmetrisation would eliminate.
            
            \item Given a list of elements, generating a new list of elements that verify certain properties without keeping a copy of the original list. For instance, if the list is numerical, keeping only numbers that are greater that some given number.

            In the very example provided, the original list could have had any number below the cutoff, so the operation is not in general invertible.
        \end{enumerate}
        
        \item Check on QUIRK that the Pauli-$X$, $Y$ and $Z$ operations, represented on the computational basis $\{\ket{0}$, $\ket{1}\}$ by the Pauli matrices $\sigma_x$, $\sigma_y$ and $\sigma_z$, are unitary. Compute their inverses and check that they are equal to their transpose conjugates.

        We can easily check by hand that

        $$
        \sigma_x = \sigma^{-1}_x = \sigma^\dagger_x = 
        \begin{pmatrix}
            0 & 1 \\
            1 & 0
        \end{pmatrix}
        $$

        $$
        \sigma_y = \sigma^{-1}_y = \sigma^\dagger_y = 
        \begin{pmatrix}
            0 & -i \\
            i & 0
        \end{pmatrix}
        $$

        $$
        \sigma_z = \sigma^{-1}_z = \sigma^\dagger_z = 
        \begin{pmatrix}
            1 & 0 \\
            0 & -1
        \end{pmatrix}
        $$
        
    \end{enumerate}
    \item Multi-qubit quantum gates
    \begin{enumerate}[label*=\arabic*.]
    \item Control gates: \texttt{CNOT} and others
    \begin{enumerate}[label=(\alph*)]
    \item Hover the mouse pointer over the two default qubit lines and check that both qubits are on the $\ket{0}$
    state.
    \item Drop a \texttt{NOT} gate over the second line and annotate the change on the corresponding Bloch sphere.

    \texttt{NOT}$\ket{0}=\ket{1}$
    
    \item Drop a \texttt{control} gate (small black circle) over the first line and on the same horizontal position that
    the first Pauli-X (\texttt{NOT}) gate. A new vertical segment joining the \texttt{control} and the \texttt{NOT} gate should
    appear. Annotate any change on any of the Bloch spheres.

    \texttt{CNOT}$\ket{00}=\ket{00}$

    A controlled gate only takes effect if the control qubit has a $\ket{1}$ component.
    
    \item Change the state of the first qubit to $\ket{1}$ by clicking on the $\ket{0}$ symbol on the left of the corresponding qubit line. Annotate any changes on the Bloch spheres.
    
    \texttt{CNOT}$\ket{01}=\ket{11}$  

    The controlled gate produces a bit flip.
    
\item Repeat the experiment using an \texttt{anti-control} (small white circle) instead of the \texttt{control}.

The \texttt{anti-control} does the opposite, it takes effect if the control qubit has a $\ket{0}$ component.

\item Repeat the experiment using a Hadamard gate instead of the \texttt{NOT} one.

The control logic is the same as before, the only difference is $H\ket{0}=\ket{+}$.

    \end{enumerate}
    \item Building entangled Bell states\par
    Using a Hadamard gate and a \texttt{CNOT} gate whose control is applied to the output of the Hadamard gate, build a quantum circuit that turns a 2-qubit initial state $\ket{0} \otimes \ket{0}$ to the following Bell states:
    \begin{enumerate}[label=(\alph*)]
        \item $\ket{\Phi^+}=\frac{1}{\sqrt{2}}(\ket{0} \otimes \ket{0}+\ket{1} \otimes \ket{1})$

        \begin{quantikz}
        \lstick{$\ket{0}$} & \gate{H} & \ctrl{1} & \qw \\
        \lstick{$\ket{0}$} & \qw & \targ{} & \qw 
        \end{quantikz}
        
        \item $\ket{\Phi^-}=\frac{1}{\sqrt{2}}(\ket{0} \otimes \ket{0}-\ket{1} \otimes \ket{1})$

        \begin{quantikz}
        \lstick{$\ket{0}$} & \targ{} & \gate{H} & \ctrl{1} & \qw \\
        \lstick{$\ket{0}$} & \qw & \qw & \targ{} & \qw 
        \end{quantikz}
        
        \item $\ket{\Psi^+}=\frac{1}{\sqrt{2}}(\ket{0} \otimes \ket{1}+\ket{1} \otimes \ket{0})$

        \begin{quantikz}
        \lstick{$\ket{0}$} & \gate{H} & \ctrl{1} & \targ{} & \qw \\
        \lstick{$\ket{0}$} & \qw & \targ{} & \qw & \qw 
        \end{quantikz}
        
        \item $\ket{\Psi^-}=\frac{1}{\sqrt{2}}(\ket{0} \otimes \ket{1}-\ket{1} \otimes \ket{0})$

        \begin{quantikz}
        \lstick{$\ket{0}$} & \targ{} & \gate{H} & \ctrl{1} & \targ{} & \qw \\
        \lstick{$\ket{0}$} & \qw & \qw & \targ{} & \qw & \qw 
        \end{quantikz}
        
    \end{enumerate}
    \item No qubit cloning, even by the CNOT gate
    \begin{enumerate}[label*=\arabic*.]
    \item The 1-qubit state $\ket{-}$ is defined as above in 2b by
    \begin{equation*}
        \ket{-}=\frac{1}{\sqrt{2}}(\ket{0}-\ket{1}).
    \end{equation*}
    Using the basis $\{\ket{00},\ket{01},\ket{10},\ket{11}\}$ to describe a 2-qubit quantum state, answer the following
questions:
    \begin{enumerate}[label=(\alph*)]
    \item What is the 2-qubit state if each qubit has been initialized to $\ket{-}$?

    The state is $\ket{-}\otimes\ket{-}=\frac{1}{2}(\ket{00}-\ket{01}-\ket{10}+\ket{11})$.
    
    \item The first qubit is initialized to $\ket{-}$ and the second, to $\ket{0}$. What is the 2-qubit state if a \texttt{CNOT}
gate has been applied targeting the second qubit, the first one being the control?

The state is \texttt{CNOT}$(\ket{0}\otimes\ket{-})=\frac{1}{\sqrt{2}}(\ket{00}-\ket{11})$.

\item Can you use a \texttt{CNOT} gate for breaking the no-clone theorem? Please, justify your answer on
the previous questions of this exercise.

At the very least, as this counterexample shows, we cannot use the \texttt{CNOT} gate to clone states as \texttt{CNOT}$(\ket{0}\otimes\ket{\psi})=\ket{\psi}\otimes\ket{\psi}$. 

    \end{enumerate}
    \item Using Hadamard gates, prepare 2 quantum registers, each one composed of 2 qubits. The states of
this registers will be:
\begin{itemize}
    \item Register A: $\ket{-}\otimes\ket{-}=(HX\ket{0})\otimes (HX\ket{0})$
    \item  Register B: the first qubit is initialized to $\ket{-}=HX\ket{0}$. The second one is initialized by applying a \texttt{CNOT} gate to the quantum state $\ket{0}$. The control of this \texttt{CNOT} gate is the first qubit, previously initialized to $\ket{-}$.
\end{itemize}
Operator $X$ is the Pauli-$X$ operator (\texttt{NOT} gate). Answer the following questions:
\begin{enumerate}[label=(\alph*)]
 \item Compute the probability of measuring a $\ket{1}$ state in each of the four qubits.

The probability to measure a $\ket{1}$ in each qubit is $50\%$.
 
 \item Is there any difference among the four Bloch spheres?

The Bloch spheres for each qubit are very different, we can only obtain states of the form $\ket{00\mathrm{xx}}$ or $\ket{11\mathrm{xx}}$, meaning the qubits in register B can only be measured in the same state. The first two are in a superposition of all possible states.
 
 \item Include an additional Hadamard gate in each qubit before the measurement. Taking into
account that $H^2 = I$, do you see anything strange? Can a \texttt{CNOT} gate alter the \textit{control} qubit?
Does the concept of individual qubit make any sense in the presence of quantum entanglement
effects?

Given that $H^2=I$, the first two qubits now become $X\otimes X\ket{00}=\ket{11}$, whereas the second two become the Bell state $\ket{\Psi^+}$ (as in exercise 4.2). A \texttt{CNOT} gate \textit{does} in general alter the control qubit --- in fact, it only doesn't when the control qubit is either in the $\ket{0}$ or in the $\ket{1}$ state. The notion of individual qubit breaks down when logic gates apply operations to all qubits and they become entangled (their measurement outcomes become correlated).

\item Using the Kronecker (tensor) product and the matrix formalism, compute by hand (or by a
matricial language like Octave or Matlab) the effect of 2 Hadamard gates, $H \otimes H$, on a Bell
state, $\displaystyle \ket{\Phi^-}=\frac{1}{\sqrt{2}}(\ket{0} \otimes \ket{0}-\ket{1} \otimes \ket{1})$ . Can you explain the QUIRK results?

The result is $\frac{1}{\sqrt{2}}(\ket{01}+\ket{10})$. In QUIRK we can simply read out the results of the applied gate in the amplitude display.

\end{enumerate}
    \end{enumerate}
    \item Quantum entanglement of distant qubits via SWAP gates\par
    On the memory of current state-of-the-art quantum computers the qubits are coupled only to their near
neighbours. In order to entangle distant qubits, the 2-qubit swapping gate (\texttt{SWAP}) is used. Such a gate is
implemented on QUIRK. Please, use it to solve the following exercise:
    \begin{enumerate}[label=(\alph*)]
\item Build a 4-qubit quantum circuit. It should generate 2 Bell pairs, $\ket{\Psi^+}=\frac{1}{\sqrt{2}}(\ket{00}+\ket{11})$ and $\ket{\Psi^-}=\frac{1}{\sqrt{2}}(\ket{00}-\ket{11})$. These pairs are formed by qubits $(1,3)$ for $\ket{\Psi^+}$ and by $(2,4)$ for $\ket{\Psi^-}$. As a restriction, consider that the \texttt{CNOT} gates can only be applied to nearest-neighbour qubits. That
is, they can entangle qubits $(1,2)$, $(2,3)$ or $(3,4)$. You should use a single \texttt{SWAP} gate, as it would be done on an actual quantum computer.

\begin{quantikz}
    \lstick{$\ket{0}$} & \qw & \qw & \targ{} & \qw & \qw \\
    \lstick{$\ket{0}$} & \qw & \gate{H} & \ctrl{-1} &  \swap{1} & \qw \\
    \lstick{$\ket{0}$} & \qw & \qw & \targ{} & \targX{} & \qw \\
    \lstick{$\ket{0}$} & \targ{} & \gate{H} & \ctrl{-1} & \qw & \qw
\end{quantikz}

\item Check that qubits $(1,3)$ and $(2,4)$ are entangled, so that they form Bell pairs. In order to prove
this, use the fact that $UU^\dagger = I$. Here, $U =$\texttt{CNOT}$\cdot (H \otimes I)$ and $U^\dagger = (H \otimes I) \cdot $ \texttt{CNOT}, since
$H^\dagger = H^{-1} = H$ and \texttt{CNOT}$^\dagger = $ \texttt{CNOT}$^{-1} = $ \texttt{CNOT}. Hence, if you reverse the quantum circuit that
generated the Bell pairs $\ket{\Psi^-}\ket{\Psi^+}$, a $\ket{0000}$ state should be recovered. Check this with the
pairs $(1,3)$ and $(2,4)$. Repeat the process with qubits $(1,2)$ and $(3,4)$, that should not be entangled.
Repeat the process, changing the initial state of each of the 4 qubits.

\begin{quantikz}
    \lstick{$\ket{0}$} & \qw & \gate{H} & \ctrl{1} & \qw & \ctrl{2} & \qw & \gate{H} & \qw & \qw & \rstick{$\ket{0}$} \\
    \lstick{$\ket{0}$} & \qw & \qw & \targ{} &  \swap{1} & \qw & \targ{} & \qw & \qw & \qw & \rstick{$\ket{0}$} \\
    \lstick{$\ket{0}$} & \qw & \qw & \targ{} & \targX{} & \targ{} & \qw & \qw & \qw & \qw & \rstick{$\ket{0}$} \\
    \lstick{$\ket{0}$} & \targ{} & \gate{H} & \ctrl{-1} & \qw & \qw & \ctrl{-2} & \gate{H} & \targ{} & \qw & \rstick{$\ket{0}$} 
\end{quantikz}

By applying the circuit which creates the Bell pairs in reversed order (swapping the corresponding qubits) we recover $\ket{0000}$ as expected.

\item Build a gate that swaps the values of 2 qubits with 3 \texttt{CNOT} gates. Compare its action over the 4
possible input values with the \texttt{SWAP} gate.

The \texttt{SWAP} gate may be decomposed as follows:

\begin{quantikz}
    \lstick{$\ket{\psi}$} & \ctrl{1} & \targ{} & \ctrl{1} & \qw & \rstick{$\ket{\phi}$} \\
    \lstick{$\ket{\phi}$} & \targ{} &  \ctrl{-1} & \targ{} & \qw & \rstick{$\ket{\psi}$}\\
\end{quantikz}

This composite gate has the same effect as a \texttt{SWAP} gate. In the $\{\ket{00},\ket{01},\ket{10},\ket{11}\}$ basis, the matrices of these gates read

$$
\texttt{CNOT}_{1\rightarrow2}=
\begin{pmatrix}
    1 & 0 & 0 & 0 \\
    0 & 1 & 0 & 0 \\
    0 & 0 & 0 & 1 \\
    0 & 0 & 1 & 0
\end{pmatrix}
$$

$$
\texttt{CNOT}_{2\rightarrow1}=
\begin{pmatrix}
    1 & 0 & 0 & 0 \\
    0 & 0 & 0 & 1 \\
    0 & 0 & 1 & 0 \\
    0 & 1 & 0 & 0
\end{pmatrix}
$$

$$
\texttt{SWAP}=
\begin{pmatrix}
    1 & 0 & 0 & 0 \\
    0 & 0 & 1 & 0 \\
    0 & 1 & 0 & 0 \\
    0 & 0 & 0 & 1
\end{pmatrix}
$$

The matrix product $\texttt{CNOT}_{1\rightarrow2}\texttt{CNOT}_{2\rightarrow1}\texttt{CNOT}_{1\rightarrow2}$ indeed yields a \texttt{SWAP} gate. Its action on the basis elements is given by its column vectors.

\item Can you implement the permutation of 5 qubits $(1, 2, 3, 4, 5) \rightarrow (2, 4, 1, 3, 5)$ in Quirk?

\begin{quantikz}
    \lstick{$\ket{\psi}_1$} & \swap{1} & \qw & \qw \\
    \lstick{$\ket{\psi}_2$} & \targX{} & \swap{1} & \qw \\
    \lstick{$\ket{\psi}_3$} & \swap{1} & \targX{} & \qw \\
    \lstick{$\ket{\psi}_4$} & \targX{} & \qw & \qw \\
    \lstick{$\ket{\psi}_5$} & \qw      & \qw & \qw \\
\end{quantikz}

\end{enumerate}
    \end{enumerate}

\end{enumerate}

\end{document}