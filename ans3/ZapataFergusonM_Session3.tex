\documentclass[11pt]{article}

    \usepackage[breakable]{tcolorbox}
    \usepackage{parskip} % Stop auto-indenting (to mimic markdown behaviour)
    

    % Basic figure setup, for now with no caption control since it's done
    % automatically by Pandoc (which extracts ![](path) syntax from Markdown).
    \usepackage{graphicx}
    % Maintain compatibility with old templates. Remove in nbconvert 6.0
    \let\Oldincludegraphics\includegraphics
    % Ensure that by default, figures have no caption (until we provide a
    % proper Figure object with a Caption API and a way to capture that
    % in the conversion process - todo).
    \usepackage{caption}
    \DeclareCaptionFormat{nocaption}{}
    \captionsetup{format=nocaption,aboveskip=0pt,belowskip=0pt}

    \usepackage{float}
    \floatplacement{figure}{H} % forces figures to be placed at the correct location
    \usepackage{xcolor} % Allow colors to be defined
    \usepackage{enumerate} % Needed for markdown enumerations to work
    \usepackage{geometry} % Used to adjust the document margins
    \usepackage{amsmath} % Equations
    \usepackage{amssymb} % Equations
    \usepackage{textcomp} % defines textquotesingle
    % Hack from http://tex.stackexchange.com/a/47451/13684:
    \AtBeginDocument{%
        \def\PYZsq{\textquotesingle}% Upright quotes in Pygmentized code
    }
    \usepackage{upquote} % Upright quotes for verbatim code
    \usepackage{eurosym} % defines \euro

    \usepackage{iftex}
    \ifPDFTeX
        \usepackage[T1]{fontenc}
        \IfFileExists{alphabeta.sty}{
              \usepackage{alphabeta}
          }{
              \usepackage[mathletters]{ucs}
              \usepackage[utf8x]{inputenc}
          }
    \else
        \usepackage{fontspec}
        \usepackage{unicode-math}
    \fi

    \usepackage{fancyvrb} % verbatim replacement that allows latex
    \usepackage{grffile} % extends the file name processing of package graphics
                         % to support a larger range
    \makeatletter % fix for old versions of grffile with XeLaTeX
    \@ifpackagelater{grffile}{2019/11/01}
    {
      % Do nothing on new versions
    }
    {
      \def\Gread@@xetex#1{%
        \IfFileExists{"\Gin@base".bb}%
        {\Gread@eps{\Gin@base.bb}}%
        {\Gread@@xetex@aux#1}%
      }
    }
    \makeatother
    \usepackage[Export]{adjustbox} % Used to constrain images to a maximum size
    \adjustboxset{max size={0.9\linewidth}{0.9\paperheight}}

    % The hyperref package gives us a pdf with properly built
    % internal navigation ('pdf bookmarks' for the table of contents,
    % internal cross-reference links, web links for URLs, etc.)
    \usepackage{hyperref}
    % The default LaTeX title has an obnoxious amount of whitespace. By default,
    % titling removes some of it. It also provides customization options.
    \usepackage{titling}
    \usepackage{longtable} % longtable support required by pandoc >1.10
    \usepackage{booktabs}  % table support for pandoc > 1.12.2
    \usepackage{array}     % table support for pandoc >= 2.11.3
    \usepackage{calc}      % table minipage width calculation for pandoc >= 2.11.1
    \usepackage[inline]{enumitem} % IRkernel/repr support (it uses the enumerate* environment)
    \usepackage[normalem]{ulem} % ulem is needed to support strikethroughs (\sout)
                                % normalem makes italics be italics, not underlines
    \usepackage{soul}      % strikethrough (\st) support for pandoc >= 3.0.0
    \usepackage{mathrsfs}
    

    
    % Colors for the hyperref package
    \definecolor{urlcolor}{rgb}{0,.145,.698}
    \definecolor{linkcolor}{rgb}{.71,0.21,0.01}
    \definecolor{citecolor}{rgb}{.12,.54,.11}

    % ANSI colors
    \definecolor{ansi-black}{HTML}{3E424D}
    \definecolor{ansi-black-intense}{HTML}{282C36}
    \definecolor{ansi-red}{HTML}{E75C58}
    \definecolor{ansi-red-intense}{HTML}{B22B31}
    \definecolor{ansi-green}{HTML}{00A250}
    \definecolor{ansi-green-intense}{HTML}{007427}
    \definecolor{ansi-yellow}{HTML}{DDB62B}
    \definecolor{ansi-yellow-intense}{HTML}{B27D12}
    \definecolor{ansi-blue}{HTML}{208FFB}
    \definecolor{ansi-blue-intense}{HTML}{0065CA}
    \definecolor{ansi-magenta}{HTML}{D160C4}
    \definecolor{ansi-magenta-intense}{HTML}{A03196}
    \definecolor{ansi-cyan}{HTML}{60C6C8}
    \definecolor{ansi-cyan-intense}{HTML}{258F8F}
    \definecolor{ansi-white}{HTML}{C5C1B4}
    \definecolor{ansi-white-intense}{HTML}{A1A6B2}
    \definecolor{ansi-default-inverse-fg}{HTML}{FFFFFF}
    \definecolor{ansi-default-inverse-bg}{HTML}{000000}

    % common color for the border for error outputs.
    \definecolor{outerrorbackground}{HTML}{FFDFDF}

    % commands and environments needed by pandoc snippets
    % extracted from the output of `pandoc -s`
    \providecommand{\tightlist}{%
      \setlength{\itemsep}{0pt}\setlength{\parskip}{0pt}}
    \DefineVerbatimEnvironment{Highlighting}{Verbatim}{commandchars=\\\{\}}
    % Add ',fontsize=\small' for more characters per line
    \newenvironment{Shaded}{}{}
    \newcommand{\KeywordTok}[1]{\textcolor[rgb]{0.00,0.44,0.13}{\textbf{{#1}}}}
    \newcommand{\DataTypeTok}[1]{\textcolor[rgb]{0.56,0.13,0.00}{{#1}}}
    \newcommand{\DecValTok}[1]{\textcolor[rgb]{0.25,0.63,0.44}{{#1}}}
    \newcommand{\BaseNTok}[1]{\textcolor[rgb]{0.25,0.63,0.44}{{#1}}}
    \newcommand{\FloatTok}[1]{\textcolor[rgb]{0.25,0.63,0.44}{{#1}}}
    \newcommand{\CharTok}[1]{\textcolor[rgb]{0.25,0.44,0.63}{{#1}}}
    \newcommand{\StringTok}[1]{\textcolor[rgb]{0.25,0.44,0.63}{{#1}}}
    \newcommand{\CommentTok}[1]{\textcolor[rgb]{0.38,0.63,0.69}{\textit{{#1}}}}
    \newcommand{\OtherTok}[1]{\textcolor[rgb]{0.00,0.44,0.13}{{#1}}}
    \newcommand{\AlertTok}[1]{\textcolor[rgb]{1.00,0.00,0.00}{\textbf{{#1}}}}
    \newcommand{\FunctionTok}[1]{\textcolor[rgb]{0.02,0.16,0.49}{{#1}}}
    \newcommand{\RegionMarkerTok}[1]{{#1}}
    \newcommand{\ErrorTok}[1]{\textcolor[rgb]{1.00,0.00,0.00}{\textbf{{#1}}}}
    \newcommand{\NormalTok}[1]{{#1}}

    % Additional commands for more recent versions of Pandoc
    \newcommand{\ConstantTok}[1]{\textcolor[rgb]{0.53,0.00,0.00}{{#1}}}
    \newcommand{\SpecialCharTok}[1]{\textcolor[rgb]{0.25,0.44,0.63}{{#1}}}
    \newcommand{\VerbatimStringTok}[1]{\textcolor[rgb]{0.25,0.44,0.63}{{#1}}}
    \newcommand{\SpecialStringTok}[1]{\textcolor[rgb]{0.73,0.40,0.53}{{#1}}}
    \newcommand{\ImportTok}[1]{{#1}}
    \newcommand{\DocumentationTok}[1]{\textcolor[rgb]{0.73,0.13,0.13}{\textit{{#1}}}}
    \newcommand{\AnnotationTok}[1]{\textcolor[rgb]{0.38,0.63,0.69}{\textbf{\textit{{#1}}}}}
    \newcommand{\CommentVarTok}[1]{\textcolor[rgb]{0.38,0.63,0.69}{\textbf{\textit{{#1}}}}}
    \newcommand{\VariableTok}[1]{\textcolor[rgb]{0.10,0.09,0.49}{{#1}}}
    \newcommand{\ControlFlowTok}[1]{\textcolor[rgb]{0.00,0.44,0.13}{\textbf{{#1}}}}
    \newcommand{\OperatorTok}[1]{\textcolor[rgb]{0.40,0.40,0.40}{{#1}}}
    \newcommand{\BuiltInTok}[1]{{#1}}
    \newcommand{\ExtensionTok}[1]{{#1}}
    \newcommand{\PreprocessorTok}[1]{\textcolor[rgb]{0.74,0.48,0.00}{{#1}}}
    \newcommand{\AttributeTok}[1]{\textcolor[rgb]{0.49,0.56,0.16}{{#1}}}
    \newcommand{\InformationTok}[1]{\textcolor[rgb]{0.38,0.63,0.69}{\textbf{\textit{{#1}}}}}
    \newcommand{\WarningTok}[1]{\textcolor[rgb]{0.38,0.63,0.69}{\textbf{\textit{{#1}}}}}


    % Define a nice break command that doesn't care if a line doesn't already
    % exist.
    \def\br{\hspace*{\fill} \\* }
    % Math Jax compatibility definitions
    \def\gt{>}
    \def\lt{<}
    \let\Oldtex\TeX
    \let\Oldlatex\LaTeX
    \renewcommand{\TeX}{\textrm{\Oldtex}}
    \renewcommand{\LaTeX}{\textrm{\Oldlatex}}
    % Document parameters
    % Document title
    \title{ZapataFergusonM\_Session3}
    
    
    
    
    
    
    
% Pygments definitions
\makeatletter
\def\PY@reset{\let\PY@it=\relax \let\PY@bf=\relax%
    \let\PY@ul=\relax \let\PY@tc=\relax%
    \let\PY@bc=\relax \let\PY@ff=\relax}
\def\PY@tok#1{\csname PY@tok@#1\endcsname}
\def\PY@toks#1+{\ifx\relax#1\empty\else%
    \PY@tok{#1}\expandafter\PY@toks\fi}
\def\PY@do#1{\PY@bc{\PY@tc{\PY@ul{%
    \PY@it{\PY@bf{\PY@ff{#1}}}}}}}
\def\PY#1#2{\PY@reset\PY@toks#1+\relax+\PY@do{#2}}

\@namedef{PY@tok@w}{\def\PY@tc##1{\textcolor[rgb]{0.73,0.73,0.73}{##1}}}
\@namedef{PY@tok@c}{\let\PY@it=\textit\def\PY@tc##1{\textcolor[rgb]{0.24,0.48,0.48}{##1}}}
\@namedef{PY@tok@cp}{\def\PY@tc##1{\textcolor[rgb]{0.61,0.40,0.00}{##1}}}
\@namedef{PY@tok@k}{\let\PY@bf=\textbf\def\PY@tc##1{\textcolor[rgb]{0.00,0.50,0.00}{##1}}}
\@namedef{PY@tok@kp}{\def\PY@tc##1{\textcolor[rgb]{0.00,0.50,0.00}{##1}}}
\@namedef{PY@tok@kt}{\def\PY@tc##1{\textcolor[rgb]{0.69,0.00,0.25}{##1}}}
\@namedef{PY@tok@o}{\def\PY@tc##1{\textcolor[rgb]{0.40,0.40,0.40}{##1}}}
\@namedef{PY@tok@ow}{\let\PY@bf=\textbf\def\PY@tc##1{\textcolor[rgb]{0.67,0.13,1.00}{##1}}}
\@namedef{PY@tok@nb}{\def\PY@tc##1{\textcolor[rgb]{0.00,0.50,0.00}{##1}}}
\@namedef{PY@tok@nf}{\def\PY@tc##1{\textcolor[rgb]{0.00,0.00,1.00}{##1}}}
\@namedef{PY@tok@nc}{\let\PY@bf=\textbf\def\PY@tc##1{\textcolor[rgb]{0.00,0.00,1.00}{##1}}}
\@namedef{PY@tok@nn}{\let\PY@bf=\textbf\def\PY@tc##1{\textcolor[rgb]{0.00,0.00,1.00}{##1}}}
\@namedef{PY@tok@ne}{\let\PY@bf=\textbf\def\PY@tc##1{\textcolor[rgb]{0.80,0.25,0.22}{##1}}}
\@namedef{PY@tok@nv}{\def\PY@tc##1{\textcolor[rgb]{0.10,0.09,0.49}{##1}}}
\@namedef{PY@tok@no}{\def\PY@tc##1{\textcolor[rgb]{0.53,0.00,0.00}{##1}}}
\@namedef{PY@tok@nl}{\def\PY@tc##1{\textcolor[rgb]{0.46,0.46,0.00}{##1}}}
\@namedef{PY@tok@ni}{\let\PY@bf=\textbf\def\PY@tc##1{\textcolor[rgb]{0.44,0.44,0.44}{##1}}}
\@namedef{PY@tok@na}{\def\PY@tc##1{\textcolor[rgb]{0.41,0.47,0.13}{##1}}}
\@namedef{PY@tok@nt}{\let\PY@bf=\textbf\def\PY@tc##1{\textcolor[rgb]{0.00,0.50,0.00}{##1}}}
\@namedef{PY@tok@nd}{\def\PY@tc##1{\textcolor[rgb]{0.67,0.13,1.00}{##1}}}
\@namedef{PY@tok@s}{\def\PY@tc##1{\textcolor[rgb]{0.73,0.13,0.13}{##1}}}
\@namedef{PY@tok@sd}{\let\PY@it=\textit\def\PY@tc##1{\textcolor[rgb]{0.73,0.13,0.13}{##1}}}
\@namedef{PY@tok@si}{\let\PY@bf=\textbf\def\PY@tc##1{\textcolor[rgb]{0.64,0.35,0.47}{##1}}}
\@namedef{PY@tok@se}{\let\PY@bf=\textbf\def\PY@tc##1{\textcolor[rgb]{0.67,0.36,0.12}{##1}}}
\@namedef{PY@tok@sr}{\def\PY@tc##1{\textcolor[rgb]{0.64,0.35,0.47}{##1}}}
\@namedef{PY@tok@ss}{\def\PY@tc##1{\textcolor[rgb]{0.10,0.09,0.49}{##1}}}
\@namedef{PY@tok@sx}{\def\PY@tc##1{\textcolor[rgb]{0.00,0.50,0.00}{##1}}}
\@namedef{PY@tok@m}{\def\PY@tc##1{\textcolor[rgb]{0.40,0.40,0.40}{##1}}}
\@namedef{PY@tok@gh}{\let\PY@bf=\textbf\def\PY@tc##1{\textcolor[rgb]{0.00,0.00,0.50}{##1}}}
\@namedef{PY@tok@gu}{\let\PY@bf=\textbf\def\PY@tc##1{\textcolor[rgb]{0.50,0.00,0.50}{##1}}}
\@namedef{PY@tok@gd}{\def\PY@tc##1{\textcolor[rgb]{0.63,0.00,0.00}{##1}}}
\@namedef{PY@tok@gi}{\def\PY@tc##1{\textcolor[rgb]{0.00,0.52,0.00}{##1}}}
\@namedef{PY@tok@gr}{\def\PY@tc##1{\textcolor[rgb]{0.89,0.00,0.00}{##1}}}
\@namedef{PY@tok@ge}{\let\PY@it=\textit}
\@namedef{PY@tok@gs}{\let\PY@bf=\textbf}
\@namedef{PY@tok@ges}{\let\PY@bf=\textbf\let\PY@it=\textit}
\@namedef{PY@tok@gp}{\let\PY@bf=\textbf\def\PY@tc##1{\textcolor[rgb]{0.00,0.00,0.50}{##1}}}
\@namedef{PY@tok@go}{\def\PY@tc##1{\textcolor[rgb]{0.44,0.44,0.44}{##1}}}
\@namedef{PY@tok@gt}{\def\PY@tc##1{\textcolor[rgb]{0.00,0.27,0.87}{##1}}}
\@namedef{PY@tok@err}{\def\PY@bc##1{{\setlength{\fboxsep}{\string -\fboxrule}\fcolorbox[rgb]{1.00,0.00,0.00}{1,1,1}{\strut ##1}}}}
\@namedef{PY@tok@kc}{\let\PY@bf=\textbf\def\PY@tc##1{\textcolor[rgb]{0.00,0.50,0.00}{##1}}}
\@namedef{PY@tok@kd}{\let\PY@bf=\textbf\def\PY@tc##1{\textcolor[rgb]{0.00,0.50,0.00}{##1}}}
\@namedef{PY@tok@kn}{\let\PY@bf=\textbf\def\PY@tc##1{\textcolor[rgb]{0.00,0.50,0.00}{##1}}}
\@namedef{PY@tok@kr}{\let\PY@bf=\textbf\def\PY@tc##1{\textcolor[rgb]{0.00,0.50,0.00}{##1}}}
\@namedef{PY@tok@bp}{\def\PY@tc##1{\textcolor[rgb]{0.00,0.50,0.00}{##1}}}
\@namedef{PY@tok@fm}{\def\PY@tc##1{\textcolor[rgb]{0.00,0.00,1.00}{##1}}}
\@namedef{PY@tok@vc}{\def\PY@tc##1{\textcolor[rgb]{0.10,0.09,0.49}{##1}}}
\@namedef{PY@tok@vg}{\def\PY@tc##1{\textcolor[rgb]{0.10,0.09,0.49}{##1}}}
\@namedef{PY@tok@vi}{\def\PY@tc##1{\textcolor[rgb]{0.10,0.09,0.49}{##1}}}
\@namedef{PY@tok@vm}{\def\PY@tc##1{\textcolor[rgb]{0.10,0.09,0.49}{##1}}}
\@namedef{PY@tok@sa}{\def\PY@tc##1{\textcolor[rgb]{0.73,0.13,0.13}{##1}}}
\@namedef{PY@tok@sb}{\def\PY@tc##1{\textcolor[rgb]{0.73,0.13,0.13}{##1}}}
\@namedef{PY@tok@sc}{\def\PY@tc##1{\textcolor[rgb]{0.73,0.13,0.13}{##1}}}
\@namedef{PY@tok@dl}{\def\PY@tc##1{\textcolor[rgb]{0.73,0.13,0.13}{##1}}}
\@namedef{PY@tok@s2}{\def\PY@tc##1{\textcolor[rgb]{0.73,0.13,0.13}{##1}}}
\@namedef{PY@tok@sh}{\def\PY@tc##1{\textcolor[rgb]{0.73,0.13,0.13}{##1}}}
\@namedef{PY@tok@s1}{\def\PY@tc##1{\textcolor[rgb]{0.73,0.13,0.13}{##1}}}
\@namedef{PY@tok@mb}{\def\PY@tc##1{\textcolor[rgb]{0.40,0.40,0.40}{##1}}}
\@namedef{PY@tok@mf}{\def\PY@tc##1{\textcolor[rgb]{0.40,0.40,0.40}{##1}}}
\@namedef{PY@tok@mh}{\def\PY@tc##1{\textcolor[rgb]{0.40,0.40,0.40}{##1}}}
\@namedef{PY@tok@mi}{\def\PY@tc##1{\textcolor[rgb]{0.40,0.40,0.40}{##1}}}
\@namedef{PY@tok@il}{\def\PY@tc##1{\textcolor[rgb]{0.40,0.40,0.40}{##1}}}
\@namedef{PY@tok@mo}{\def\PY@tc##1{\textcolor[rgb]{0.40,0.40,0.40}{##1}}}
\@namedef{PY@tok@ch}{\let\PY@it=\textit\def\PY@tc##1{\textcolor[rgb]{0.24,0.48,0.48}{##1}}}
\@namedef{PY@tok@cm}{\let\PY@it=\textit\def\PY@tc##1{\textcolor[rgb]{0.24,0.48,0.48}{##1}}}
\@namedef{PY@tok@cpf}{\let\PY@it=\textit\def\PY@tc##1{\textcolor[rgb]{0.24,0.48,0.48}{##1}}}
\@namedef{PY@tok@c1}{\let\PY@it=\textit\def\PY@tc##1{\textcolor[rgb]{0.24,0.48,0.48}{##1}}}
\@namedef{PY@tok@cs}{\let\PY@it=\textit\def\PY@tc##1{\textcolor[rgb]{0.24,0.48,0.48}{##1}}}

\def\PYZbs{\char`\\}
\def\PYZus{\char`\_}
\def\PYZob{\char`\{}
\def\PYZcb{\char`\}}
\def\PYZca{\char`\^}
\def\PYZam{\char`\&}
\def\PYZlt{\char`\<}
\def\PYZgt{\char`\>}
\def\PYZsh{\char`\#}
\def\PYZpc{\char`\%}
\def\PYZdl{\char`\$}
\def\PYZhy{\char`\-}
\def\PYZsq{\char`\'}
\def\PYZdq{\char`\"}
\def\PYZti{\char`\~}
% for compatibility with earlier versions
\def\PYZat{@}
\def\PYZlb{[}
\def\PYZrb{]}
\makeatother


    % For linebreaks inside Verbatim environment from package fancyvrb.
    \makeatletter
        \newbox\Wrappedcontinuationbox
        \newbox\Wrappedvisiblespacebox
        \newcommand*\Wrappedvisiblespace {\textcolor{red}{\textvisiblespace}}
        \newcommand*\Wrappedcontinuationsymbol {\textcolor{red}{\llap{\tiny$\m@th\hookrightarrow$}}}
        \newcommand*\Wrappedcontinuationindent {3ex }
        \newcommand*\Wrappedafterbreak {\kern\Wrappedcontinuationindent\copy\Wrappedcontinuationbox}
        % Take advantage of the already applied Pygments mark-up to insert
        % potential linebreaks for TeX processing.
        %        {, <, #, %, $, ' and ": go to next line.
        %        _, }, ^, &, >, - and ~: stay at end of broken line.
        % Use of \textquotesingle for straight quote.
        \newcommand*\Wrappedbreaksatspecials {%
            \def\PYGZus{\discretionary{\char`\_}{\Wrappedafterbreak}{\char`\_}}%
            \def\PYGZob{\discretionary{}{\Wrappedafterbreak\char`\{}{\char`\{}}%
            \def\PYGZcb{\discretionary{\char`\}}{\Wrappedafterbreak}{\char`\}}}%
            \def\PYGZca{\discretionary{\char`\^}{\Wrappedafterbreak}{\char`\^}}%
            \def\PYGZam{\discretionary{\char`\&}{\Wrappedafterbreak}{\char`\&}}%
            \def\PYGZlt{\discretionary{}{\Wrappedafterbreak\char`\<}{\char`\<}}%
            \def\PYGZgt{\discretionary{\char`\>}{\Wrappedafterbreak}{\char`\>}}%
            \def\PYGZsh{\discretionary{}{\Wrappedafterbreak\char`\#}{\char`\#}}%
            \def\PYGZpc{\discretionary{}{\Wrappedafterbreak\char`\%}{\char`\%}}%
            \def\PYGZdl{\discretionary{}{\Wrappedafterbreak\char`\$}{\char`\$}}%
            \def\PYGZhy{\discretionary{\char`\-}{\Wrappedafterbreak}{\char`\-}}%
            \def\PYGZsq{\discretionary{}{\Wrappedafterbreak\textquotesingle}{\textquotesingle}}%
            \def\PYGZdq{\discretionary{}{\Wrappedafterbreak\char`\"}{\char`\"}}%
            \def\PYGZti{\discretionary{\char`\~}{\Wrappedafterbreak}{\char`\~}}%
        }
        % Some characters . , ; ? ! / are not pygmentized.
        % This macro makes them "active" and they will insert potential linebreaks
        \newcommand*\Wrappedbreaksatpunct {%
            \lccode`\~`\.\lowercase{\def~}{\discretionary{\hbox{\char`\.}}{\Wrappedafterbreak}{\hbox{\char`\.}}}%
            \lccode`\~`\,\lowercase{\def~}{\discretionary{\hbox{\char`\,}}{\Wrappedafterbreak}{\hbox{\char`\,}}}%
            \lccode`\~`\;\lowercase{\def~}{\discretionary{\hbox{\char`\;}}{\Wrappedafterbreak}{\hbox{\char`\;}}}%
            \lccode`\~`\:\lowercase{\def~}{\discretionary{\hbox{\char`\:}}{\Wrappedafterbreak}{\hbox{\char`\:}}}%
            \lccode`\~`\?\lowercase{\def~}{\discretionary{\hbox{\char`\?}}{\Wrappedafterbreak}{\hbox{\char`\?}}}%
            \lccode`\~`\!\lowercase{\def~}{\discretionary{\hbox{\char`\!}}{\Wrappedafterbreak}{\hbox{\char`\!}}}%
            \lccode`\~`\/\lowercase{\def~}{\discretionary{\hbox{\char`\/}}{\Wrappedafterbreak}{\hbox{\char`\/}}}%
            \catcode`\.\active
            \catcode`\,\active
            \catcode`\;\active
            \catcode`\:\active
            \catcode`\?\active
            \catcode`\!\active
            \catcode`\/\active
            \lccode`\~`\~
        }
    \makeatother

    \let\OriginalVerbatim=\Verbatim
    \makeatletter
    \renewcommand{\Verbatim}[1][1]{%
        %\parskip\z@skip
        \sbox\Wrappedcontinuationbox {\Wrappedcontinuationsymbol}%
        \sbox\Wrappedvisiblespacebox {\FV@SetupFont\Wrappedvisiblespace}%
        \def\FancyVerbFormatLine ##1{\hsize\linewidth
            \vtop{\raggedright\hyphenpenalty\z@\exhyphenpenalty\z@
                \doublehyphendemerits\z@\finalhyphendemerits\z@
                \strut ##1\strut}%
        }%
        % If the linebreak is at a space, the latter will be displayed as visible
        % space at end of first line, and a continuation symbol starts next line.
        % Stretch/shrink are however usually zero for typewriter font.
        \def\FV@Space {%
            \nobreak\hskip\z@ plus\fontdimen3\font minus\fontdimen4\font
            \discretionary{\copy\Wrappedvisiblespacebox}{\Wrappedafterbreak}
            {\kern\fontdimen2\font}%
        }%

        % Allow breaks at special characters using \PYG... macros.
        \Wrappedbreaksatspecials
        % Breaks at punctuation characters . , ; ? ! and / need catcode=\active
        \OriginalVerbatim[#1,codes*=\Wrappedbreaksatpunct]%
    }
    \makeatother

    % Exact colors from NB
    \definecolor{incolor}{HTML}{303F9F}
    \definecolor{outcolor}{HTML}{D84315}
    \definecolor{cellborder}{HTML}{CFCFCF}
    \definecolor{cellbackground}{HTML}{F7F7F7}

    % prompt
    \makeatletter
    \newcommand{\boxspacing}{\kern\kvtcb@left@rule\kern\kvtcb@boxsep}
    \makeatother
    \newcommand{\prompt}[4]{
        {\ttfamily\llap{{\color{#2}[#3]:\hspace{3pt}#4}}\vspace{-\baselineskip}}
    }
    

    
    % Prevent overflowing lines due to hard-to-break entities
    \sloppy
    % Setup hyperref package
    \hypersetup{
      breaklinks=true,  % so long urls are correctly broken across lines
      colorlinks=true,
      urlcolor=urlcolor,
      linkcolor=linkcolor,
      citecolor=citecolor,
      }
    % Slightly bigger margins than the latex defaults
    
    \geometry{verbose,tmargin=1in,bmargin=1in,lmargin=1in,rmargin=1in}
    
    

\begin{document}
    
    

    
    \section*{Session 3: Execution of real quantum circuits on IBM's
computers}\label{session-3-execution-of-real-quantum-circuits-on-ibms-computers}

    \begin{tcolorbox}[breakable, size=fbox, boxrule=1pt, pad at break*=1mm,colback=cellbackground, colframe=cellborder]
\prompt{In}{incolor}{2}{\boxspacing}
\begin{Verbatim}[commandchars=\\\{\}]
\PY{c+c1}{\PYZsh{} Importing standard Qiskit libraries}
\PY{k+kn}{from} \PY{n+nn}{qiskit} \PY{k+kn}{import} \PY{n}{QuantumCircuit}\PY{p}{,} \PY{n}{transpile}
\PY{k+kn}{from} \PY{n+nn}{qiskit\PYZus{}aer} \PY{k+kn}{import} \PY{n}{AerSimulator}
\PY{k+kn}{from} \PY{n+nn}{qiskit\PYZus{}ibm\PYZus{}runtime} \PY{k+kn}{import} \PY{n}{QiskitRuntimeService}\PY{p}{,} \PY{n}{SamplerV2}\PY{p}{,} \PY{n}{EstimatorV2}
\PY{k+kn}{from} \PY{n+nn}{qiskit}\PY{n+nn}{.}\PY{n+nn}{quantum\PYZus{}info} \PY{k+kn}{import} \PY{n}{SparsePauliOp}\PY{p}{,} \PY{n}{Operator}

\PY{c+c1}{\PYZsh{} Other imports}
\PY{k+kn}{import} \PY{n+nn}{numpy} \PY{k}{as} \PY{n+nn}{np}
\PY{k+kn}{import} \PY{n+nn}{pandas} \PY{k}{as} \PY{n+nn}{pd}
\PY{k+kn}{import} \PY{n+nn}{matplotlib}\PY{n+nn}{.}\PY{n+nn}{pyplot} \PY{k}{as} \PY{n+nn}{plt}
\end{Verbatim}
\end{tcolorbox}

    In the lab session we used premium accounts to fast forward over the
long queue list of IBM's quantum machines. However at home we can use
our free accounts to schedule calculations.

    \begin{tcolorbox}[breakable, size=fbox, boxrule=1pt, pad at break*=1mm,colback=cellbackground, colframe=cellborder]
\prompt{In}{incolor}{3}{\boxspacing}
\begin{Verbatim}[commandchars=\\\{\}]
\PY{c+c1}{\PYZsh{} Recover previous session information}
\PY{n}{df} \PY{o}{=} \PY{n}{pd}\PY{o}{.}\PY{n}{read\PYZus{}csv}\PY{p}{(}\PY{l+s+s1}{\PYZsq{}}\PY{l+s+s1}{session.csv}\PY{l+s+s1}{\PYZsq{}}\PY{p}{)}

\PY{c+c1}{\PYZsh{} Loading account}
\PY{n}{service} \PY{o}{=} \PY{n}{QiskitRuntimeService}\PY{p}{(}\PY{n}{channel}\PY{o}{=}\PY{l+s+s1}{\PYZsq{}}\PY{l+s+s1}{ibm\PYZus{}quantum}\PY{l+s+s1}{\PYZsq{}}\PY{p}{,} \PY{n}{token}\PY{o}{=}\PY{n}{df}\PY{o}{.}\PY{n}{token}\PY{p}{[}\PY{l+m+mi}{0}\PY{p}{]}\PY{p}{)} 

\PY{c+c1}{\PYZsh{} Simulation machine}
\PY{n}{backend\PYZus{}S} \PY{o}{=} \PY{n}{AerSimulator}\PY{p}{(}\PY{p}{)}
\PY{c+c1}{\PYZsh{} In the lab we used backend\PYZus{}S = service.backend(\PYZsq{}ibmq\PYZus{}qasm\PYZus{}simulator\PYZsq{}), }
\PY{c+c1}{\PYZsh{} however cloud simulators have been deprecated and will be removed on 15 May 2024}
\end{Verbatim}
\end{tcolorbox}

    \begin{tcolorbox}[breakable, size=fbox, boxrule=1pt, pad at break*=1mm,colback=cellbackground, colframe=cellborder]
\prompt{In}{incolor}{ }{\boxspacing}
\begin{Verbatim}[commandchars=\\\{\}]
\PY{c+c1}{\PYZsh{} Quantum machine}
\PY{n}{backend\PYZus{}Q} \PY{o}{=} \PY{n}{service}\PY{o}{.}\PY{n}{least\PYZus{}busy}\PY{p}{(}\PY{n}{operational}\PY{o}{=}\PY{k+kc}{True}\PY{p}{,} \PY{n}{simulator}\PY{o}{=}\PY{k+kc}{False}\PY{p}{,} \PY{n}{min\PYZus{}num\PYZus{}qubits}\PY{o}{=}\PY{l+m+mi}{20}\PY{p}{)}

\PY{c+c1}{\PYZsh{} Save backend}
\PY{n}{df}\PY{o}{.}\PY{n}{backend\PYZus{}Q} \PY{o}{=} \PY{n}{backend\PYZus{}Q}\PY{o}{.}\PY{n}{name}
\PY{n}{df}\PY{o}{.}\PY{n}{to\PYZus{}csv}\PY{p}{(}\PY{l+s+s1}{\PYZsq{}}\PY{l+s+s1}{session.txt}\PY{l+s+s1}{\PYZsq{}}\PY{p}{,} \PY{n}{index}\PY{o}{=}\PY{k+kc}{False}\PY{p}{)}
\end{Verbatim}
\end{tcolorbox}

    \begin{tcolorbox}[breakable, size=fbox, boxrule=1pt, pad at break*=1mm,colback=cellbackground, colframe=cellborder]
\prompt{In}{incolor}{3}{\boxspacing}
\begin{Verbatim}[commandchars=\\\{\}]
\PY{n}{backend\PYZus{}Q} \PY{o}{=} \PY{n}{service}\PY{o}{.}\PY{n}{get\PYZus{}backend}\PY{p}{(}\PY{n}{df}\PY{o}{.}\PY{n}{backend\PYZus{}Q}\PY{p}{[}\PY{l+m+mi}{0}\PY{p}{]}\PY{p}{)} 
\PY{n}{backend\PYZus{}Q}\PY{o}{.}\PY{n}{status}\PY{p}{(}\PY{p}{)}
\end{Verbatim}
\end{tcolorbox}

            \begin{tcolorbox}[breakable, size=fbox, boxrule=.5pt, pad at break*=1mm, opacityfill=0]
\prompt{Out}{outcolor}{3}{\boxspacing}
\begin{Verbatim}[commandchars=\\\{\}]
<qiskit.providers.models.backendstatus.BackendStatus at 0x2109434b090>
\end{Verbatim}
\end{tcolorbox}
        
    \begin{tcolorbox}[breakable, size=fbox, boxrule=1pt, pad at break*=1mm,colback=cellbackground, colframe=cellborder]
\prompt{In}{incolor}{4}{\boxspacing}
\begin{Verbatim}[commandchars=\\\{\}]
\PY{c+c1}{\PYZsh{} Maximum execution time in seconds}
\PY{n}{my\PYZus{}options} \PY{o}{=} \PY{p}{\PYZob{}}\PY{l+s+s1}{\PYZsq{}}\PY{l+s+s1}{max\PYZus{}execution\PYZus{}time}\PY{l+s+s1}{\PYZsq{}}\PY{p}{:} \PY{l+m+mi}{30}\PY{p}{\PYZcb{}}

\PY{c+c1}{\PYZsh{} Classical and quantum estimators}
\PY{n}{estimator\PYZus{}S} \PY{o}{=} \PY{n}{EstimatorV2}\PY{p}{(}\PY{n}{backend}\PY{o}{=}\PY{n}{backend\PYZus{}S}\PY{p}{)}
\PY{n}{estimator\PYZus{}Q} \PY{o}{=} \PY{n}{EstimatorV2}\PY{p}{(}\PY{n}{backend}\PY{o}{=}\PY{n}{backend\PYZus{}Q}\PY{p}{,} \PY{n}{options}\PY{o}{=}\PY{n}{my\PYZus{}options}\PY{p}{)}

\PY{c+c1}{\PYZsh{} Classical and quantum samplers}
\PY{n}{sampler\PYZus{}S} \PY{o}{=} \PY{n}{SamplerV2}\PY{p}{(}\PY{n}{backend}\PY{o}{=}\PY{n}{backend\PYZus{}S}\PY{p}{)}
\PY{n}{sampler\PYZus{}Q} \PY{o}{=} \PY{n}{SamplerV2}\PY{p}{(}\PY{n}{backend}\PY{o}{=}\PY{n}{backend\PYZus{}Q}\PY{p}{,} \PY{n}{options}\PY{o}{=}\PY{n}{my\PYZus{}options}\PY{p}{)}
\end{Verbatim}
\end{tcolorbox}

    \subsection*{\texorpdfstring{\textbf{Random numbers}
}{Random numbers }}\label{random-numbers}

The generation of random numbers using a classical simulator of a
quantum machine was conducted during the second session. Now, let's
generate ``true random numbers''! To achieve this, we define a 3-qubit
circuit that prepares each of them in the superposition
\(\tfrac{1}{\sqrt2}(|0\rangle+|1\rangle)\) and measures them in the
computational basis \({|0\rangle,|1\rangle}\). This setup ensures that
the measurement outcomes will be 0 or 1 with equal probabilities of 0.5.

\textbf{Question 1}: Study the definition of the circuit with
care.


\textbf{Answer:} It creates a barrier of Hadamard gates, which prepares
each qubit in an equal superposition of \(|0\rangle\) and \(|1\rangle\).
The circuit assigns three qubits to each random number to be generated,
meaning the random numbers range between \(0\) and \(7\). This is an
example of the \emph{Hadamard transform}, which in this case performs
the following operation:

\[
H^{\otimes 90}|0\rangle^{\otimes 90} = |+\rangle^{\otimes 90}
\]

More generally, if we have a register initialised to a state
\(|j\rangle\) of the computational basis, the action of a Hadamard
transform is the following:

\[
H^{\otimes n}|j\rangle = \bigotimes_{i=0}^{n}H|j_i\rangle = 
\frac{1}{2^{n/2}} \bigotimes_{i=0}^{n}(|0\rangle+(-1)^{j_i}|1\rangle) = 
\frac{1}{2^{n/2}} \bigotimes_{i=0}^{n}\sum\limits_{k_i=0}^1 (-1)^{k_ij_i}|k_i\rangle =
\]

\[
= \frac{1}{2^{n/2}} \sum\limits_{k_1=0}^1 (-1)^{k_1j_1}|k_1\rangle
\sum\limits_{k_2=0}^1 (-1)^{k_2j_2}|k_2\rangle... 
\sum\limits_{k_n=0}^1 (-1)^{k_nj_n}|k_n\rangle = 
\]

\[
= \frac{1}{2^{n/2}} \sum\limits_{k_1=0}^1\sum\limits_{k_2=0}^1...\sum\limits_{k_n=0}^1
 (-1)^{\sum\limits_{i=1}^{n}k_ij_i}|k_1k_2...k_n\rangle \equiv
\]

\[
\equiv\frac{1}{2^{n/2}} \sum\limits_{k=0}^{2^n-1}(-1)^{\sum\limits_{i=1}^{n}k_ij_i}|k\rangle
\]

    \begin{tcolorbox}[breakable, size=fbox, boxrule=1pt, pad at break*=1mm,colback=cellbackground, colframe=cellborder]
\prompt{In}{incolor}{100}{\boxspacing}
\begin{Verbatim}[commandchars=\\\{\}]
\PY{c+c1}{\PYZsh{} How many random numbers will be produced in a single shot:}
\PY{n}{Nnumbers} \PY{o}{=} \PY{l+m+mi}{30}  \PY{c+c1}{\PYZsh{} Must be smaller than [127 qubits/3]=42}
\PY{c+c1}{\PYZsh{} Prepare the input circuit}
\PY{n}{QRNG} \PY{o}{=} \PY{n}{QuantumCircuit}\PY{p}{(}\PY{l+m+mi}{3}\PY{o}{*}\PY{n}{Nnumbers}\PY{p}{)} \PY{c+c1}{\PYZsh{} Three qubits per number}
\PY{n}{QRNG}\PY{o}{.}\PY{n}{h}\PY{p}{(}\PY{n+nb}{range}\PY{p}{(}\PY{l+m+mi}{3}\PY{o}{*}\PY{n}{Nnumbers}\PY{p}{)}\PY{p}{)} \PY{c+c1}{\PYZsh{} Apply Hadamard gate to each of the qubits}
\PY{n}{QRNG}\PY{o}{.}\PY{n}{measure\PYZus{}all}\PY{p}{(}\PY{p}{)} \PY{c+c1}{\PYZsh{} Measure all qubits}
\PY{n}{QRNG} \PY{o}{=} \PY{n}{transpile}\PY{p}{(}\PY{n}{QRNG}\PY{p}{,} \PY{n}{backend}\PY{o}{=}\PY{n}{backend\PYZus{}Q}\PY{p}{)} \PY{c+c1}{\PYZsh{} Adapt circuit to architecture of quantum machine}
\end{Verbatim}
\end{tcolorbox}

    \begin{tcolorbox}[breakable, size=fbox, boxrule=1pt, pad at break*=1mm,colback=cellbackground, colframe=cellborder]
\prompt{In}{incolor}{34}{\boxspacing}
\begin{Verbatim}[commandchars=\\\{\}]
\PY{c+c1}{\PYZsh{} Execute the circuit directly on a quantum computer}
\PY{n}{job\PYZus{}qrng} \PY{o}{=} \PY{n}{sampler\PYZus{}Q}\PY{o}{.}\PY{n}{run}\PY{p}{(}\PY{p}{[}\PY{n}{QRNG}\PY{p}{]}\PY{p}{,} \PY{n}{shots}\PY{o}{=}\PY{l+m+mi}{1}\PY{p}{)}

\PY{c+c1}{\PYZsh{} Save job identification code for later}
\PY{n}{df}\PY{o}{.}\PY{n}{qrng\PYZus{}id} \PY{o}{=} \PY{n}{job\PYZus{}qrng}\PY{o}{.}\PY{n}{job\PYZus{}id}\PY{p}{(}\PY{p}{)}
\PY{n}{df}\PY{o}{.}\PY{n}{to\PYZus{}csv}\PY{p}{(}\PY{l+s+s1}{\PYZsq{}}\PY{l+s+s1}{session.csv}\PY{l+s+s1}{\PYZsq{}}\PY{p}{,} \PY{n}{index}\PY{o}{=}\PY{k+kc}{False}\PY{p}{)}
\end{Verbatim}
\end{tcolorbox}

    \begin{Verbatim}[commandchars=\\\{\}]
c:\textbackslash{}Users\textbackslash{}zapat\textbackslash{}Escritorio\textbackslash{}CODE\textbackslash{}PIE\_Compu\_Cuantica\textbackslash{}venv\textbackslash{}Lib\textbackslash{}site-
packages\textbackslash{}qiskit\_ibm\_runtime\textbackslash{}qiskit\_runtime\_service.py:879: UserWarning: Your
current pending jobs are estimated to consume 649.5556236231932 quantum seconds,
but you only have 600 quantum seconds left in your monthly quota; therefore, it
is likely this job will be canceled
  warnings.warn(warning\_message)
    \end{Verbatim}

    \begin{tcolorbox}[breakable, size=fbox, boxrule=1pt, pad at break*=1mm,colback=cellbackground, colframe=cellborder]
\prompt{In}{incolor}{81}{\boxspacing}
\begin{Verbatim}[commandchars=\\\{\}]
\PY{c+c1}{\PYZsh{} The following cell retrieves information about the job}
\PY{n}{job\PYZus{}qrng} \PY{o}{=} \PY{n}{service}\PY{o}{.}\PY{n}{job}\PY{p}{(}\PY{n}{df}\PY{o}{.}\PY{n}{qrng\PYZus{}id}\PY{p}{[}\PY{l+m+mi}{0}\PY{p}{]}\PY{p}{)}
\PY{n}{job\PYZus{}qrng}\PY{o}{.}\PY{n}{status}\PY{p}{(}\PY{p}{)} 
\end{Verbatim}
\end{tcolorbox}

            \begin{tcolorbox}[breakable, size=fbox, boxrule=.5pt, pad at break*=1mm, opacityfill=0]
\prompt{Out}{outcolor}{81}{\boxspacing}
\begin{Verbatim}[commandchars=\\\{\}]
<JobStatus.DONE: 'job has successfully run'>
\end{Verbatim}
\end{tcolorbox}
        
    \begin{tcolorbox}[breakable, size=fbox, boxrule=1pt, pad at break*=1mm,colback=cellbackground, colframe=cellborder]
\prompt{In}{incolor}{82}{\boxspacing}
\begin{Verbatim}[commandchars=\\\{\}]
\PY{c+c1}{\PYZsh{} The next statement can put your session on hold until the job runs and returns}
\PY{n}{result\PYZus{}qrng} \PY{o}{=} \PY{n}{job\PYZus{}qrng}\PY{o}{.}\PY{n}{result}\PY{p}{(}\PY{p}{)}
\end{Verbatim}
\end{tcolorbox}

\textbf{Question 2}: Comment on your results.


\textbf{Answer:} We obtained the following result from the IBM Quantum
computer run in the lab with UCM tokens:

\texttt{{[}\{236732400872330227231122054:\ 1.0\}{]}}

After depuration we obtain the following list of numbers

\texttt{{[}6,\ 0,\ 2,\ 1,\ 6,\ 4,\ 4,\ 5,\ 2,\ 2,\ 3,\ 3,\ 4,\ 2,\ 2,\ 3,\ 2,\ 1,\ 2,\ 5,\ 2,\ 2,\ 2,\ 0,\ 2,\ 2,\ 7,\ 1,\ 4,\ 1{]}}

that we can plot in a histogram.

    \begin{tcolorbox}[breakable, size=fbox, boxrule=1pt, pad at break*=1mm,colback=cellbackground, colframe=cellborder]
\prompt{In}{incolor}{7}{\boxspacing}
\begin{Verbatim}[commandchars=\\\{\}]
\PY{n}{RandomNumbers} \PY{o}{=} \PY{p}{[}\PY{l+m+mi}{6}\PY{p}{,} \PY{l+m+mi}{0}\PY{p}{,} \PY{l+m+mi}{2}\PY{p}{,} \PY{l+m+mi}{1}\PY{p}{,} \PY{l+m+mi}{6}\PY{p}{,} \PY{l+m+mi}{4}\PY{p}{,} \PY{l+m+mi}{4}\PY{p}{,} \PY{l+m+mi}{5}\PY{p}{,} \PY{l+m+mi}{2}\PY{p}{,} \PY{l+m+mi}{2}\PY{p}{,} \PY{l+m+mi}{3}\PY{p}{,} \PY{l+m+mi}{3}\PY{p}{,} \PY{l+m+mi}{4}\PY{p}{,} \PY{l+m+mi}{2}\PY{p}{,} \PY{l+m+mi}{2}\PY{p}{,} \PY{l+m+mi}{3}\PY{p}{,} \PY{l+m+mi}{2}\PY{p}{,} \PY{l+m+mi}{1}\PY{p}{,} \PY{l+m+mi}{2}\PY{p}{,} \PY{l+m+mi}{5}\PY{p}{,} \PY{l+m+mi}{2}\PY{p}{,} \PY{l+m+mi}{2}\PY{p}{,} \PY{l+m+mi}{2}\PY{p}{,} \PY{l+m+mi}{0}\PY{p}{,} \PY{l+m+mi}{2}\PY{p}{,} \PY{l+m+mi}{2}\PY{p}{,} \PY{l+m+mi}{7}\PY{p}{,} \PY{l+m+mi}{1}\PY{p}{,} \PY{l+m+mi}{4}\PY{p}{,} \PY{l+m+mi}{1}\PY{p}{]}
\PY{n}{plt}\PY{o}{.}\PY{n}{hist}\PY{p}{(}\PY{n}{RandomNumbers}\PY{p}{,} \PY{n}{bins}\PY{o}{=}\PY{n+nb}{range}\PY{p}{(}\PY{l+m+mi}{9}\PY{p}{)}\PY{p}{,} \PY{n}{align}\PY{o}{=}\PY{l+s+s1}{\PYZsq{}}\PY{l+s+s1}{left}\PY{l+s+s1}{\PYZsq{}}\PY{p}{,} \PY{n}{edgecolor}\PY{o}{=}\PY{l+s+s1}{\PYZsq{}}\PY{l+s+s1}{black}\PY{l+s+s1}{\PYZsq{}}\PY{p}{,} \PY{n}{linewidth}\PY{o}{=}\PY{l+m+mi}{1}\PY{p}{)}
\PY{n}{plt}\PY{o}{.}\PY{n}{show}\PY{p}{(}\PY{p}{)}
\end{Verbatim}
\end{tcolorbox}

    \begin{center}
    \adjustimage{max size={0.9\linewidth}{0.9\paperheight}}{ZapataFergusonM_Session3_files/ZapataFergusonM_Session3_13_0.png}
    \end{center}
    { \hspace*{\fill} \\}
    
    We can likewise retrieve our own results run at home.

    \begin{tcolorbox}[breakable, size=fbox, boxrule=1pt, pad at break*=1mm,colback=cellbackground, colframe=cellborder]
\prompt{In}{incolor}{134}{\boxspacing}
\begin{Verbatim}[commandchars=\\\{\}]
\PY{n}{bits}\PY{p}{,} \PY{o}{=} \PY{n}{result\PYZus{}qrng}\PY{p}{[}\PY{l+m+mi}{0}\PY{p}{]}\PY{o}{.}\PY{n}{data}\PY{o}{.}\PY{n}{meas}\PY{o}{.}\PY{n}{get\PYZus{}counts}\PY{p}{(}\PY{p}{)}
\PY{n}{numbers} \PY{o}{=} \PY{p}{[}\PY{n}{bits}\PY{p}{[}\PY{n}{i}\PY{p}{:}\PY{n}{i}\PY{o}{+}\PY{l+m+mi}{3}\PY{p}{]} \PY{k}{for} \PY{n}{i} \PY{o+ow}{in} \PY{n+nb}{range}\PY{p}{(}\PY{l+m+mi}{0}\PY{p}{,} \PY{l+m+mi}{3}\PY{o}{*}\PY{n}{Nnumbers}\PY{p}{,} \PY{l+m+mi}{3}\PY{p}{)}\PY{p}{]}
\PY{n}{decimal\PYZus{}numbers} \PY{o}{=} \PY{p}{[}\PY{n+nb}{int}\PY{p}{(}\PY{n}{n}\PY{p}{,} \PY{l+m+mi}{2}\PY{p}{)} \PY{k}{for} \PY{n}{n} \PY{o+ow}{in} \PY{n}{numbers}\PY{p}{]}
\PY{n}{plt}\PY{o}{.}\PY{n}{hist}\PY{p}{(}\PY{n}{decimal\PYZus{}numbers}\PY{p}{,} \PY{n}{bins}\PY{o}{=}\PY{n+nb}{range}\PY{p}{(}\PY{l+m+mi}{9}\PY{p}{)}\PY{p}{,} \PY{n}{align}\PY{o}{=}\PY{l+s+s1}{\PYZsq{}}\PY{l+s+s1}{left}\PY{l+s+s1}{\PYZsq{}}\PY{p}{,} \PY{n}{edgecolor}\PY{o}{=}\PY{l+s+s1}{\PYZsq{}}\PY{l+s+s1}{black}\PY{l+s+s1}{\PYZsq{}}\PY{p}{,} \PY{n}{linewidth}\PY{o}{=}\PY{l+m+mi}{1}\PY{p}{)}
\PY{n}{plt}\PY{o}{.}\PY{n}{show}\PY{p}{(}\PY{p}{)}
\end{Verbatim}
\end{tcolorbox}

    \begin{center}
    \adjustimage{max size={0.9\linewidth}{0.9\paperheight}}{ZapataFergusonM_Session3_files/ZapataFergusonM_Session3_15_0.png}
    \end{center}
    { \hspace*{\fill} \\}
    
    The distribution is not ``particularly uniform'', it rather looks more
log-normal. However, we generated very few random numbers, to draw any
statistical conclusions we would likely have to produce many more.

    \subsection*{\texorpdfstring{\textbf{Deutsch's
algorithm}}{Deutsch's algorithm}}\label{deutschs-algorithm}

To encode the action of a Boolean fuction, Deutsch's algorithm uses the
following operation over two qubits

\[
|x\rangle|y\rangle\to |x\rangle|y\oplus f(x)\rangle
\]

where \(\oplus\) denotes the binary addition (i.e.~the addition mod. 2).

\textbf{Homework 1} (to do later after the
lab on pen and paper): Prove that this is a unitary operation.


\textbf{Answer:} It is clear that the quantum oracle preserves the norm,
therefore it is unitary. If we take
\((x,y)\in\{0,1\}^2, f(x)\in\{0,1\}\), we have that
\(y\oplus f(x)\in\{0,1\}\) --- in other words,
\(|x\rangle|y\oplus f(x)\rangle\) is a ket from the computational basis
of the Hilbert space. The operation might be non-invertible if it mapped
two different kets to a single one, but this is impossible, since
\(0\oplus f(x)\neq 1\oplus f(x)\):

\[
(\langle x|\langle 0|)U_f^\dagger U_f(|x\rangle|1\rangle) =
(\langle x|\langle 0\oplus f(x)|)(|x\rangle|1\oplus f(x)\rangle) = 
\langle x|x\rangle\langle 0\oplus f(x)|1\oplus f(x)\rangle = 0
\]

Alternatively, given there exist a finite number of Boolean functions,
we can calculate the oracle matrix for each \(f\) to better understand
what \(U_f\) does. If we place the control qubit \(|x\rangle\) first, as
we'll later do in the circuits (so
\(|y,x\rangle\rightarrow|y\oplus f(x), x\rangle\)), the representations
of \(U_f\) are

\[
\begin{aligned}
f(x)&=0:\quad
U_f = \begin{pmatrix}
1 & 0 & 0 & 0 \\
0 & 1 & 0 & 0 \\
0 & 0 & 1 & 0 \\
0 & 0 & 0 & 1
\end{pmatrix},
&
f(x)&=1:\quad
U_f = \begin{pmatrix}
0 & 0 & 1 & 0 \\
0 & 0 & 0 & 1 \\
1 & 0 & 0 & 0 \\
0 & 1 & 0 & 0
\end{pmatrix}
\end{aligned}
\]

\[
\begin{aligned}
f(x)&=x: \quad
U_f = \begin{pmatrix}
1 & 0 & 0 & 0 \\
0 & 0 & 0 & 1 \\
0 & 0 & 1 & 0 \\
0 & 1 & 0 & 0
\end{pmatrix},
&
f(x)&=\neg x:\quad
U_f = \begin{pmatrix}
0 & 0 & 1 & 0 \\
0 & 1 & 0 & 0 \\
1 & 0 & 0 & 0 \\
0 & 0 & 0 & 1
\end{pmatrix}
\end{aligned}
\]

These matrices all represent permutations, which are unitary operations.

textbf{Question 2}: Complete the following
cells to create quantum circuits for Oracle\_f that decides whether
\(f\) is (or not) constant using the classical method with two
evaluations.


\textbf{Answer:} To evaluate \(f\) using the oracle we can measure the
leftmost qubit in the following two states:

\[
U_f|0\rangle|0\rangle = |f(0)\rangle|0\rangle
\]

\[
U_f|0\rangle|1\rangle = |f(1)\rangle|1\rangle
\]

N.B.: Qubit significance and indexing in Qiskit goes from right to left,
top to bottom, \(0\) to \(n\).

    \begin{tcolorbox}[breakable, size=fbox, boxrule=1pt, pad at break*=1mm,colback=cellbackground, colframe=cellborder]
\prompt{In}{incolor}{9}{\boxspacing}
\begin{Verbatim}[commandchars=\\\{\}]
\PY{n}{Oracle\PYZus{}f} \PY{o}{=} \PY{n}{Operator}\PY{p}{(}\PY{p}{[}\PY{p}{[}\PY{l+m+mi}{0}\PY{p}{,}\PY{l+m+mi}{0}\PY{p}{,}\PY{l+m+mi}{1}\PY{p}{,}\PY{l+m+mi}{0}\PY{p}{]}\PY{p}{,}\PY{p}{[}\PY{l+m+mi}{0}\PY{p}{,}\PY{l+m+mi}{1}\PY{p}{,}\PY{l+m+mi}{0}\PY{p}{,}\PY{l+m+mi}{0}\PY{p}{]}\PY{p}{,}\PY{p}{[}\PY{l+m+mi}{1}\PY{p}{,}\PY{l+m+mi}{0}\PY{p}{,}\PY{l+m+mi}{0}\PY{p}{,}\PY{l+m+mi}{0}\PY{p}{]}\PY{p}{,}\PY{p}{[}\PY{l+m+mi}{0}\PY{p}{,}\PY{l+m+mi}{0}\PY{p}{,}\PY{l+m+mi}{0}\PY{p}{,}\PY{l+m+mi}{1}\PY{p}{]}\PY{p}{]}\PY{p}{)}
\PY{n}{Oracle\PYZus{}g} \PY{o}{=} \PY{n}{Operator}\PY{p}{(}\PY{p}{[}\PY{p}{[}\PY{l+m+mi}{0}\PY{p}{,}\PY{l+m+mi}{0}\PY{p}{,}\PY{l+m+mi}{1}\PY{p}{,}\PY{l+m+mi}{0}\PY{p}{]}\PY{p}{,}\PY{p}{[}\PY{l+m+mi}{0}\PY{p}{,}\PY{l+m+mi}{0}\PY{p}{,}\PY{l+m+mi}{0}\PY{p}{,}\PY{l+m+mi}{1}\PY{p}{]}\PY{p}{,}\PY{p}{[}\PY{l+m+mi}{1}\PY{p}{,}\PY{l+m+mi}{0}\PY{p}{,}\PY{l+m+mi}{0}\PY{p}{,}\PY{l+m+mi}{0}\PY{p}{]}\PY{p}{,}\PY{p}{[}\PY{l+m+mi}{0}\PY{p}{,}\PY{l+m+mi}{1}\PY{p}{,}\PY{l+m+mi}{0}\PY{p}{,}\PY{l+m+mi}{0}\PY{p}{]}\PY{p}{]}\PY{p}{)}

\PY{c+c1}{\PYZsh{} First evaluation}
\PY{n}{Check\PYZus{}f0} \PY{o}{=} \PY{n}{QuantumCircuit}\PY{p}{(}\PY{l+m+mi}{2}\PY{p}{,}\PY{l+m+mi}{1}\PY{p}{)}
\PY{n}{Check\PYZus{}f0}\PY{o}{.}\PY{n}{unitary}\PY{p}{(}\PY{n}{Oracle\PYZus{}f}\PY{p}{,} \PY{p}{[}\PY{l+m+mi}{0}\PY{p}{,} \PY{l+m+mi}{1}\PY{p}{]}\PY{p}{,} \PY{n}{label}\PY{o}{=}\PY{l+s+s1}{\PYZsq{}}\PY{l+s+s1}{Oracle\PYZus{}f}\PY{l+s+s1}{\PYZsq{}}\PY{p}{)}
\PY{n}{Check\PYZus{}f0}\PY{o}{.}\PY{n}{measure}\PY{p}{(}\PY{l+m+mi}{1}\PY{p}{,} \PY{l+m+mi}{0}\PY{p}{)}
\PY{n}{Check\PYZus{}f0}\PY{o}{.}\PY{n}{draw}\PY{p}{(}\PY{n}{output}\PY{o}{=}\PY{l+s+s1}{\PYZsq{}}\PY{l+s+s1}{mpl}\PY{l+s+s1}{\PYZsq{}}\PY{p}{)}
\end{Verbatim}
\end{tcolorbox}
 
            
\prompt{Out}{outcolor}{9}{}
    
    \begin{center}
    \adjustimage{max size={0.9\linewidth}{0.9\paperheight}}{ZapataFergusonM_Session3_files/ZapataFergusonM_Session3_19_0.png}
    \end{center}
    { \hspace*{\fill} \\}
    

    \begin{tcolorbox}[breakable, size=fbox, boxrule=1pt, pad at break*=1mm,colback=cellbackground, colframe=cellborder]
\prompt{In}{incolor}{123}{\boxspacing}
\begin{Verbatim}[commandchars=\\\{\}]
\PY{c+c1}{\PYZsh{} Second evaluation}
\PY{n}{Check\PYZus{}f1} \PY{o}{=} \PY{n}{QuantumCircuit}\PY{p}{(}\PY{l+m+mi}{2}\PY{p}{,}\PY{l+m+mi}{1}\PY{p}{)}
\PY{n}{Check\PYZus{}f1}\PY{o}{.}\PY{n}{x}\PY{p}{(}\PY{l+m+mi}{0}\PY{p}{)}
\PY{n}{Check\PYZus{}f1}\PY{o}{.}\PY{n}{unitary}\PY{p}{(}\PY{n}{Oracle\PYZus{}f}\PY{p}{,} \PY{p}{[}\PY{l+m+mi}{0}\PY{p}{,} \PY{l+m+mi}{1}\PY{p}{]}\PY{p}{,} \PY{n}{label}\PY{o}{=}\PY{l+s+s1}{\PYZsq{}}\PY{l+s+s1}{Oracle\PYZus{}f}\PY{l+s+s1}{\PYZsq{}}\PY{p}{)}
\PY{n}{Check\PYZus{}f1}\PY{o}{.}\PY{n}{measure}\PY{p}{(}\PY{l+m+mi}{1}\PY{p}{,}\PY{l+m+mi}{0}\PY{p}{)}
\PY{n}{Check\PYZus{}f1}\PY{o}{.}\PY{n}{draw}\PY{p}{(}\PY{n}{output}\PY{o}{=}\PY{l+s+s1}{\PYZsq{}}\PY{l+s+s1}{mpl}\PY{l+s+s1}{\PYZsq{}}\PY{p}{)}
\end{Verbatim}
\end{tcolorbox}
 
            
\prompt{Out}{outcolor}{123}{}
    
    \begin{center}
    \adjustimage{max size={0.9\linewidth}{0.9\paperheight}}{ZapataFergusonM_Session3_files/ZapataFergusonM_Session3_20_0.png}
    \end{center}
    { \hspace*{\fill} \\}
    

    \begin{tcolorbox}[breakable, size=fbox, boxrule=1pt, pad at break*=1mm,colback=cellbackground, colframe=cellborder]
\prompt{In}{incolor}{183}{\boxspacing}
\begin{Verbatim}[commandchars=\\\{\}]
\PY{c+c1}{\PYZsh{} Run both evaluations}
\PY{n}{job\PYZus{}clf} \PY{o}{=} \PY{n}{sampler\PYZus{}S}\PY{o}{.}\PY{n}{run}\PY{p}{(}\PY{p}{[}\PY{n}{Check\PYZus{}f0}\PY{p}{,} \PY{n}{Check\PYZus{}f1}\PY{p}{]}\PY{p}{,} \PY{n}{shots}\PY{o}{=}\PY{l+m+mi}{1}\PY{p}{)}
\PY{n}{result\PYZus{}clf} \PY{o}{=} \PY{n}{job\PYZus{}clf}\PY{o}{.}\PY{n}{result}\PY{p}{(}\PY{p}{)}
\end{Verbatim}
\end{tcolorbox}

\textbf{Question 3}: By printing
``result\_clf.quasi\_dists'', explain why the following cell gives the
calculated result for \(f(0)\) and \(f(1)\).


\textbf{Answer:} The function \(f(x)\) appears to be \(\neg x\), and
therefore a balanced function. Indeed, if we compare the
\texttt{Oracle\_f} with the matrix we calculated above for
\(f(x)=\neg x\), they coincide.

N.B.: For my calculations at home I used the \texttt{SamplerV2} class
which has slightly different methods for retrieving results as compared
to \texttt{Sampler}.

    \begin{tcolorbox}[breakable, size=fbox, boxrule=1pt, pad at break*=1mm,colback=cellbackground, colframe=cellborder]
\prompt{In}{incolor}{184}{\boxspacing}
\begin{Verbatim}[commandchars=\\\{\}]
\PY{c+c1}{\PYZsh{} Print the measurement results}
\PY{n+nb}{print}\PY{p}{(}\PY{l+s+sa}{f}\PY{l+s+s1}{\PYZsq{}}\PY{l+s+s1}{f(0) = }\PY{l+s+si}{\PYZob{}}\PY{n+nb}{int}\PY{p}{(}\PY{o}{*}\PY{n}{result\PYZus{}clf}\PY{p}{[}\PY{l+m+mi}{0}\PY{p}{]}\PY{o}{.}\PY{n}{data}\PY{o}{.}\PY{n}{c}\PY{o}{.}\PY{n}{get\PYZus{}counts}\PY{p}{(}\PY{p}{)}\PY{p}{)}\PY{l+s+si}{\PYZcb{}}\PY{l+s+s1}{\PYZsq{}}\PY{p}{)}
\PY{n+nb}{print}\PY{p}{(}\PY{l+s+sa}{f}\PY{l+s+s1}{\PYZsq{}}\PY{l+s+s1}{f(1) = }\PY{l+s+si}{\PYZob{}}\PY{n+nb}{int}\PY{p}{(}\PY{o}{*}\PY{n}{result\PYZus{}clf}\PY{p}{[}\PY{l+m+mi}{1}\PY{p}{]}\PY{o}{.}\PY{n}{data}\PY{o}{.}\PY{n}{c}\PY{o}{.}\PY{n}{get\PYZus{}counts}\PY{p}{(}\PY{p}{)}\PY{p}{)}\PY{l+s+si}{\PYZcb{}}\PY{l+s+s1}{\PYZsq{}}\PY{p}{)}
\end{Verbatim}
\end{tcolorbox}

    \begin{Verbatim}[commandchars=\\\{\}]
f(0) = 1
f(1) = 0
    \end{Verbatim}

\textbf{Question 4}: Complete the following
cells to create quantum circuits for Oracle\_g that solve the problem of
whether \(g\) is constant using the classical method of two evaluations.


\textbf{Answer:} We may repeat the previous process. We find that
\(g(x)=1\), therefore \(g\) is a constant function.

    \begin{tcolorbox}[breakable, size=fbox, boxrule=1pt, pad at break*=1mm,colback=cellbackground, colframe=cellborder]
\prompt{In}{incolor}{186}{\boxspacing}
\begin{Verbatim}[commandchars=\\\{\}]
\PY{c+c1}{\PYZsh{} First evaluation}
\PY{n}{Check\PYZus{}g0} \PY{o}{=} \PY{n}{QuantumCircuit}\PY{p}{(}\PY{l+m+mi}{2}\PY{p}{,}\PY{l+m+mi}{1}\PY{p}{)}
\PY{n}{Check\PYZus{}g0}\PY{o}{.}\PY{n}{unitary}\PY{p}{(}\PY{n}{Oracle\PYZus{}g}\PY{p}{,} \PY{p}{[}\PY{l+m+mi}{0}\PY{p}{,} \PY{l+m+mi}{1}\PY{p}{]}\PY{p}{,} \PY{n}{label}\PY{o}{=}\PY{l+s+s1}{\PYZsq{}}\PY{l+s+s1}{Oracle\PYZus{}g}\PY{l+s+s1}{\PYZsq{}}\PY{p}{)}
\PY{n}{Check\PYZus{}g0}\PY{o}{.}\PY{n}{measure}\PY{p}{(}\PY{l+m+mi}{1}\PY{p}{,} \PY{l+m+mi}{0}\PY{p}{)}

\PY{c+c1}{\PYZsh{} Second evaluation}
\PY{n}{Check\PYZus{}g1} \PY{o}{=} \PY{n}{QuantumCircuit}\PY{p}{(}\PY{l+m+mi}{2}\PY{p}{,}\PY{l+m+mi}{1}\PY{p}{)}
\PY{n}{Check\PYZus{}g1}\PY{o}{.}\PY{n}{x}\PY{p}{(}\PY{l+m+mi}{0}\PY{p}{)}
\PY{n}{Check\PYZus{}g1}\PY{o}{.}\PY{n}{unitary}\PY{p}{(}\PY{n}{Oracle\PYZus{}g}\PY{p}{,} \PY{p}{[}\PY{l+m+mi}{0}\PY{p}{,} \PY{l+m+mi}{1}\PY{p}{]}\PY{p}{,} \PY{n}{label}\PY{o}{=}\PY{l+s+s1}{\PYZsq{}}\PY{l+s+s1}{Oracle\PYZus{}g}\PY{l+s+s1}{\PYZsq{}}\PY{p}{)}
\PY{n}{Check\PYZus{}g1}\PY{o}{.}\PY{n}{measure}\PY{p}{(}\PY{l+m+mi}{1}\PY{p}{,}\PY{l+m+mi}{0}\PY{p}{)}

\PY{c+c1}{\PYZsh{} Run both evaluations}
\PY{n}{job\PYZus{}clg} \PY{o}{=} \PY{n}{sampler\PYZus{}S}\PY{o}{.}\PY{n}{run}\PY{p}{(}\PY{p}{[}\PY{n}{Check\PYZus{}g0}\PY{p}{,} \PY{n}{Check\PYZus{}g1}\PY{p}{]}\PY{p}{,} \PY{n}{shots}\PY{o}{=}\PY{l+m+mi}{1}\PY{p}{)}
\PY{n}{result\PYZus{}clg} \PY{o}{=} \PY{n}{job\PYZus{}clg}\PY{o}{.}\PY{n}{result}\PY{p}{(}\PY{p}{)}

\PY{c+c1}{\PYZsh{} Print the measurement results}
\PY{n+nb}{print}\PY{p}{(}\PY{l+s+sa}{f}\PY{l+s+s1}{\PYZsq{}}\PY{l+s+s1}{g(0) = }\PY{l+s+si}{\PYZob{}}\PY{n+nb}{int}\PY{p}{(}\PY{o}{*}\PY{n}{result\PYZus{}clg}\PY{p}{[}\PY{l+m+mi}{0}\PY{p}{]}\PY{o}{.}\PY{n}{data}\PY{o}{.}\PY{n}{c}\PY{o}{.}\PY{n}{get\PYZus{}counts}\PY{p}{(}\PY{p}{)}\PY{p}{)}\PY{l+s+si}{\PYZcb{}}\PY{l+s+s1}{\PYZsq{}}\PY{p}{)}
\PY{n+nb}{print}\PY{p}{(}\PY{l+s+sa}{f}\PY{l+s+s1}{\PYZsq{}}\PY{l+s+s1}{g(1) = }\PY{l+s+si}{\PYZob{}}\PY{n+nb}{int}\PY{p}{(}\PY{o}{*}\PY{n}{result\PYZus{}clg}\PY{p}{[}\PY{l+m+mi}{1}\PY{p}{]}\PY{o}{.}\PY{n}{data}\PY{o}{.}\PY{n}{c}\PY{o}{.}\PY{n}{get\PYZus{}counts}\PY{p}{(}\PY{p}{)}\PY{p}{)}\PY{l+s+si}{\PYZcb{}}\PY{l+s+s1}{\PYZsq{}}\PY{p}{)}
\end{Verbatim}
\end{tcolorbox}

    \begin{Verbatim}[commandchars=\\\{\}]
g(0) = 1
g(1) = 1
    \end{Verbatim}

    The idea of Deutsch's algorithm is based on the use of quantum
superposition to attempt a simultaneous evaluation and comparison of
\(f(0)\) and \(f(1)\) with a single action of the quantum oracle.

If \(f(0)=f(1)\),

\[
U_f|+\rangle|-\rangle=\frac{1}{2}\big[|0\rangle|f(0)\rangle-|0\rangle|1\oplus f(0)\rangle+|1\rangle|f(0)\rangle-|1\rangle|1\oplus f(0)\rangle\big]\]
\[=\frac{1}{2}\big[\big(|0\rangle+|1\rangle\big)|f(0)\rangle-\big(|0\rangle+|1\rangle\big)|1\oplus f(0)\rangle\big]=|+\rangle \left[\frac{|f(0)\rangle-|1\oplus f(0)\rangle}{\sqrt{2}}\right]
\]

\paragraph{\texorpdfstring{\textbf{Homework question 5}: Prove (after
the lab, on pen and
paper):}{Homework question 5: Prove (after the lab, on pen and paper):}}\label{homework-question-5-prove-after-the-lab-on-pen-and-paper}

the alternative case, that if \(f(0)\neq f(1)\), the final state
becomes:

\[
|-\rangle \left[\frac{|f(0)\rangle-|f(1)\rangle}{\sqrt{2}}\right].
\]


    \textbf{Answer:} If \(f(0)\neq f(1)\), then \(f(0)\oplus 1=f(1)\) and
\(f(1)\oplus 1=f(0)\), therefore \[
|+\rangle|-\rangle=\frac{1}{2}\big(|0\rangle|0\rangle-|0\rangle|1\rangle+|1\rangle|0\rangle-|1\rangle|1\rangle\big)\to \]
\[\frac{1}{2}\big[|0\rangle|f(0)\rangle-|0\rangle|1\oplus f(0)\rangle+|1\rangle|f(1)\rangle-|1\rangle|1\oplus f(1)\rangle\big] = \]
\[ = \frac{1}{2}\big[|0\rangle|f(0)\rangle-|0\rangle|f(1)\rangle+|1\rangle|f(1)\rangle-|1\rangle|f(0)\rangle\big] = \]
\[ = \frac{1}{2}\big[|0\rangle-|1\rangle\big]\big[|f(0)\rangle-|f(1)\rangle\big] =|-\rangle \left[\frac{|f(0)\rangle-|f(1)\rangle}{\sqrt{2}}\right]\]

After running \(|+\rangle|-\rangle\) through the oracle, we may apply a
Hadamard gate to the leftmost qubit and then measure it. This way we
obtain \(H|+\rangle=|0\rangle\) if \(f\) is constant and
\(H|-\rangle=|1\rangle\) if \(f\) is balanced.

A better way to understand the solution to Deutsch's problem is through
\emph{phase kickback}, the fact that controlled operations apply a phase
to their control qubits. By considering it, we can simultaneously solve
both cases in the problem statement:

\[
U_f|+\rangle|-\rangle = \frac{1}{2}
\bigg(|0\rangle\big(|f(0)\oplus 0\rangle-|f(0)\oplus 1\rangle\big)+|1\rangle\big(|f(1)\oplus 0\rangle-|f(1)\oplus 1\rangle\big)\bigg) = 
\]

\[
=  \frac{1}{2}
\bigg((-1)^{f(0)} |0\rangle\big(|0\rangle-|1\rangle\big)+(-1)^{f(1)} |1\rangle\big(|0\rangle-|1\rangle\big)\bigg) = 
\]

\[
= \frac{1}{\sqrt{2}}\bigg((-1)^{f(0)} |0\rangle+(-1)^{f(1)} |1\rangle\bigg)|-\rangle =
\]

\[
= \frac{(-1)^{f(0)}}{\sqrt{2}}\bigg(|0\rangle+(-1)^{f(0)\oplus f(1)} |1\rangle\bigg)|-\rangle
\]

The modest quantum advantage in the Deutsch algortihm can be improved by
keeping phase kickback in mind. In the Deutsch-Jozsa algorithm {[}1{]},
the oracle now implements \(f:\{0,1\}^n\rightarrow\{0,1\}\), which we
are promised is either constant or balanced, and we are asked to
classify it. We start with state \(|0\rangle^{\otimes n}|1\rangle\), to
which we apply a Hadamard transform:

\[
H^{\otimes(n+1)}(|0\rangle^{\otimes n}|1\rangle) = \frac{1}{2^{n/2}}\sum\limits_{x=0}^{2^n-1}|x\rangle|-\rangle
\]

If we now run this state through the oracle, as before, we get

\[
 \frac{1}{2^{n/2}}\sum\limits_{x=0}^{2^n-1}(-1)^{f(x)}|x\rangle|-\rangle
\]

Next, we take the \(n\) control qubits and apply a Hadamard transform to
them:

\[
H^{\otimes n}\left[\frac{1}{2^{n/2}}\sum\limits_{x=0}^{2^n-1}(-1)^{f(x)}|x\rangle\right]= 
\frac{1}{2^{n/2}}\sum\limits_{x=0}^{2^n-1}(-1)^{f(x)}
\left[\frac{1}{2^{n/2}}\sum\limits_{y=0}^{2^n-1}(-1)^{x\cdot y}|y\rangle\right] = 
\sum\limits_{y=0}^{2^n-1}\left[\frac{1}{2^n}\sum\limits_{x=0}^{2^n-1}(-1)^{f(x)}(-1)^{x\cdot y}\right]|y\rangle
\]

Finally, we conclude that the probability to measure \(|z\rangle\) is
\(\left|\frac{1}{2^n}\sum\limits_{x=0}^{2^n-1}(-1)^{f(x)}(-1)^{x\cdot z}\right|^2\),
so

\[
\mathrm{Prob}(|0\rangle^{\otimes n}) =  
\left|\frac{1}{2^n}\sum\limits_{x=0}^{2^n-1}(-1)^{f(x)}\right|^2 =
\begin{cases}
1\ \text{if}\ f(x)\ \text{is constant} \\[1em]
0\ \text{if}\ f(x)\ \text{is balanced}
\end{cases}
\]

thus solving the problem in a single evaluation of \(f\). For a
classical solution, we would have to check just over half of all
possible bit strings (\(2^{n-1}+1\)), meaning the Deutsch-Jozsa
algorithm has an exponential advantage over classical algorithms.

\textbf{Question 6}: Define a quantum
circuit to implement Deutsch's algorithm with the function \(f\) and
another with the function \(g\).


\textbf{Answer:}

    For the function \(f\):

    \begin{tcolorbox}[breakable, size=fbox, boxrule=1pt, pad at break*=1mm,colback=cellbackground, colframe=cellborder]
\prompt{In}{incolor}{10}{\boxspacing}
\begin{Verbatim}[commandchars=\\\{\}]
\PY{n}{deutsch\PYZus{}f} \PY{o}{=} \PY{n}{QuantumCircuit}\PY{p}{(}\PY{l+m+mi}{2}\PY{p}{,}\PY{l+m+mi}{1}\PY{p}{)}
\PY{n}{deutsch\PYZus{}f}\PY{o}{.}\PY{n}{x}\PY{p}{(}\PY{l+m+mi}{1}\PY{p}{)}
\PY{n}{deutsch\PYZus{}f}\PY{o}{.}\PY{n}{h}\PY{p}{(}\PY{p}{[}\PY{l+m+mi}{0}\PY{p}{,} \PY{l+m+mi}{1}\PY{p}{]}\PY{p}{)}
\PY{n}{deutsch\PYZus{}f}\PY{o}{.}\PY{n}{unitary}\PY{p}{(}\PY{n}{Oracle\PYZus{}f}\PY{p}{,} \PY{p}{[}\PY{l+m+mi}{0}\PY{p}{,} \PY{l+m+mi}{1}\PY{p}{]}\PY{p}{,} \PY{n}{label}\PY{o}{=}\PY{l+s+s1}{\PYZsq{}}\PY{l+s+s1}{Oracle\PYZus{}f}\PY{l+s+s1}{\PYZsq{}}\PY{p}{)}
\PY{n}{deutsch\PYZus{}f}\PY{o}{.}\PY{n}{h}\PY{p}{(}\PY{l+m+mi}{0}\PY{p}{)}
\PY{n}{deutsch\PYZus{}f}\PY{o}{.}\PY{n}{measure}\PY{p}{(}\PY{l+m+mi}{0}\PY{p}{,}\PY{l+m+mi}{0}\PY{p}{)}
\PY{n}{deutsch\PYZus{}f}\PY{o}{.}\PY{n}{draw}\PY{p}{(}\PY{n}{output}\PY{o}{=}\PY{l+s+s1}{\PYZsq{}}\PY{l+s+s1}{mpl}\PY{l+s+s1}{\PYZsq{}}\PY{p}{)}
\end{Verbatim}
\end{tcolorbox}
 
            
\prompt{Out}{outcolor}{10}{}
    
    \begin{center}
    \adjustimage{max size={0.9\linewidth}{0.9\paperheight}}{ZapataFergusonM_Session3_files/ZapataFergusonM_Session3_30_0.png}
    \end{center}
    { \hspace*{\fill} \\}
    

    And for the function \(g\) we just change the oracle:

    \begin{tcolorbox}[breakable, size=fbox, boxrule=1pt, pad at break*=1mm,colback=cellbackground, colframe=cellborder]
\prompt{In}{incolor}{11}{\boxspacing}
\begin{Verbatim}[commandchars=\\\{\}]
\PY{n}{deutsch\PYZus{}g} \PY{o}{=} \PY{n}{QuantumCircuit}\PY{p}{(}\PY{l+m+mi}{2}\PY{p}{,}\PY{l+m+mi}{1}\PY{p}{)}
\PY{n}{deutsch\PYZus{}g}\PY{o}{.}\PY{n}{x}\PY{p}{(}\PY{l+m+mi}{1}\PY{p}{)}
\PY{n}{deutsch\PYZus{}g}\PY{o}{.}\PY{n}{h}\PY{p}{(}\PY{p}{[}\PY{l+m+mi}{0}\PY{p}{,} \PY{l+m+mi}{1}\PY{p}{]}\PY{p}{)}
\PY{n}{deutsch\PYZus{}g}\PY{o}{.}\PY{n}{unitary}\PY{p}{(}\PY{n}{Oracle\PYZus{}g}\PY{p}{,} \PY{p}{[}\PY{l+m+mi}{0}\PY{p}{,} \PY{l+m+mi}{1}\PY{p}{]}\PY{p}{,} \PY{n}{label}\PY{o}{=}\PY{l+s+s1}{\PYZsq{}}\PY{l+s+s1}{Oracle\PYZus{}g}\PY{l+s+s1}{\PYZsq{}}\PY{p}{)}
\PY{n}{deutsch\PYZus{}g}\PY{o}{.}\PY{n}{h}\PY{p}{(}\PY{l+m+mi}{0}\PY{p}{)}
\PY{n}{deutsch\PYZus{}g}\PY{o}{.}\PY{n}{measure}\PY{p}{(}\PY{l+m+mi}{0}\PY{p}{,}\PY{l+m+mi}{0}\PY{p}{)}
\PY{n}{deutsch\PYZus{}g}\PY{o}{.}\PY{n}{draw}\PY{p}{(}\PY{n}{output}\PY{o}{=}\PY{l+s+s1}{\PYZsq{}}\PY{l+s+s1}{mpl}\PY{l+s+s1}{\PYZsq{}}\PY{p}{)}
\end{Verbatim}
\end{tcolorbox}
 
            
\prompt{Out}{outcolor}{11}{}
    
    \begin{center}
    \adjustimage{max size={0.9\linewidth}{0.9\paperheight}}{ZapataFergusonM_Session3_files/ZapataFergusonM_Session3_32_0.png}
    \end{center}
    { \hspace*{\fill} \\}
    

    \begin{tcolorbox}[breakable, size=fbox, boxrule=1pt, pad at break*=1mm,colback=cellbackground, colframe=cellborder]
\prompt{In}{incolor}{26}{\boxspacing}
\begin{Verbatim}[commandchars=\\\{\}]
\PY{c+c1}{\PYZsh{} Execute the circuit}
\PY{n}{job\PYZus{}Deutsch\PYZus{}S} \PY{o}{=} \PY{n}{sampler\PYZus{}S}\PY{o}{.}\PY{n}{run}\PY{p}{(}\PY{p}{[}\PY{n}{deutsch\PYZus{}f}\PY{p}{,} \PY{n}{deutsch\PYZus{}g}\PY{p}{]}\PY{p}{,} \PY{n}{shots}\PY{o}{=}\PY{l+m+mi}{1}\PY{p}{)}
\PY{n}{result\PYZus{}Deutsch\PYZus{}S} \PY{o}{=} \PY{n}{job\PYZus{}Deutsch\PYZus{}S}\PY{o}{.}\PY{n}{result}\PY{p}{(}\PY{p}{)}

\PY{c+c1}{\PYZsh{} Print the measurement results}
\PY{k}{if} \PY{n+nb}{int}\PY{p}{(}\PY{o}{*}\PY{n}{result\PYZus{}Deutsch\PYZus{}S}\PY{p}{[}\PY{l+m+mi}{0}\PY{p}{]}\PY{o}{.}\PY{n}{data}\PY{o}{.}\PY{n}{c}\PY{o}{.}\PY{n}{get\PYZus{}counts}\PY{p}{(}\PY{p}{)}\PY{p}{)} \PY{o}{==} \PY{l+m+mi}{0}\PY{p}{:}
    \PY{n+nb}{print}\PY{p}{(}\PY{l+s+s1}{\PYZsq{}}\PY{l+s+s1}{f is constant}\PY{l+s+s1}{\PYZsq{}}\PY{p}{)}
\PY{k}{else}\PY{p}{:}
    \PY{n+nb}{print}\PY{p}{(}\PY{l+s+s1}{\PYZsq{}}\PY{l+s+s1}{f is balanced}\PY{l+s+s1}{\PYZsq{}}\PY{p}{)}
\PY{k}{if} \PY{n+nb}{int}\PY{p}{(}\PY{o}{*}\PY{n}{result\PYZus{}Deutsch\PYZus{}S}\PY{p}{[}\PY{l+m+mi}{1}\PY{p}{]}\PY{o}{.}\PY{n}{data}\PY{o}{.}\PY{n}{c}\PY{o}{.}\PY{n}{get\PYZus{}counts}\PY{p}{(}\PY{p}{)}\PY{p}{)} \PY{o}{==} \PY{l+m+mi}{0}\PY{p}{:}
    \PY{n+nb}{print}\PY{p}{(}\PY{l+s+s1}{\PYZsq{}}\PY{l+s+s1}{g is constant}\PY{l+s+s1}{\PYZsq{}}\PY{p}{)}
\PY{k}{else}\PY{p}{:}
    \PY{n+nb}{print}\PY{p}{(}\PY{l+s+s1}{\PYZsq{}}\PY{l+s+s1}{g is balanced}\PY{l+s+s1}{\PYZsq{}}\PY{p}{)}
\end{Verbatim}
\end{tcolorbox}

    \begin{Verbatim}[commandchars=\\\{\}]
f is balanced
g is constant
    \end{Verbatim}

\textbf{Question 7}: Does the result of your
quantum computation using Deutsch's algorithm agree with the previous
classical
result?

    \textbf{Answer:} The classical result is that \(f\) is balanced and
\(g\) is constant. After applying Deutsch's algorithm to both functions,
we obtained the same result --- that is, that \(f\) is balanced and
\(g\) is constant. Our run in the lab with UCM tokens yielded the same
result.

    \begin{tcolorbox}[breakable, size=fbox, boxrule=1pt, pad at break*=1mm,colback=cellbackground, colframe=cellborder]
\prompt{In}{incolor}{13}{\boxspacing}
\begin{Verbatim}[commandchars=\\\{\}]
\PY{c+c1}{\PYZsh{} Transpile circuits to backend architecture}
\PY{n}{deutsch\PYZus{}f\PYZus{}t} \PY{o}{=} \PY{n}{transpile}\PY{p}{(}\PY{n}{deutsch\PYZus{}f}\PY{p}{,} \PY{n}{backend}\PY{o}{=}\PY{n}{backend\PYZus{}Q}\PY{p}{)}
\PY{n}{deutsch\PYZus{}g\PYZus{}t} \PY{o}{=} \PY{n}{transpile}\PY{p}{(}\PY{n}{deutsch\PYZus{}g}\PY{p}{,} \PY{n}{backend}\PY{o}{=}\PY{n}{backend\PYZus{}Q}\PY{p}{)}

\PY{c+c1}{\PYZsh{} Execute the circuit}
\PY{n}{job\PYZus{}Deutsch\PYZus{}Q} \PY{o}{=} \PY{n}{sampler\PYZus{}Q}\PY{o}{.}\PY{n}{run}\PY{p}{(}\PY{p}{[}\PY{n}{deutsch\PYZus{}f\PYZus{}t}\PY{p}{,} \PY{n}{deutsch\PYZus{}g\PYZus{}t}\PY{p}{]}\PY{p}{,} \PY{n}{shots}\PY{o}{=}\PY{l+m+mi}{3}\PY{p}{)}

\PY{c+c1}{\PYZsh{} Save job identification code for later}
\PY{n}{df}\PY{o}{.}\PY{n}{deutsch\PYZus{}id} \PY{o}{=} \PY{n}{job\PYZus{}Deutsch\PYZus{}Q}\PY{o}{.}\PY{n}{job\PYZus{}id}\PY{p}{(}\PY{p}{)}
\PY{n}{df}\PY{o}{.}\PY{n}{to\PYZus{}csv}\PY{p}{(}\PY{l+s+s1}{\PYZsq{}}\PY{l+s+s1}{session.csv}\PY{l+s+s1}{\PYZsq{}}\PY{p}{,} \PY{n}{index}\PY{o}{=}\PY{k+kc}{False}\PY{p}{)}
\end{Verbatim}
\end{tcolorbox}

    \begin{Verbatim}[commandchars=\\\{\}]
c:\textbackslash{}Users\textbackslash{}zapat\textbackslash{}Escritorio\textbackslash{}CODE\textbackslash{}PIE\_Compu\_Cuantica\textbackslash{}venv\textbackslash{}Lib\textbackslash{}site-
packages\textbackslash{}qiskit\_ibm\_runtime\textbackslash{}qiskit\_runtime\_service.py:879: UserWarning: Your
current pending jobs are estimated to consume 649.5556236231932 quantum seconds,
but you only have 588 quantum seconds left in your monthly quota; therefore, it
is likely this job will be canceled
  warnings.warn(warning\_message)
    \end{Verbatim}

    \begin{tcolorbox}[breakable, size=fbox, boxrule=1pt, pad at break*=1mm,colback=cellbackground, colframe=cellborder]
\prompt{In}{incolor}{5}{\boxspacing}
\begin{Verbatim}[commandchars=\\\{\}]
\PY{c+c1}{\PYZsh{} The following cell retrieves information about the job}
\PY{n}{job\PYZus{}Deutsch\PYZus{}Q} \PY{o}{=} \PY{n}{service}\PY{o}{.}\PY{n}{job}\PY{p}{(}\PY{n}{df}\PY{o}{.}\PY{n}{deutsch\PYZus{}id}\PY{p}{[}\PY{l+m+mi}{0}\PY{p}{]}\PY{p}{)}
\PY{n}{job\PYZus{}Deutsch\PYZus{}Q}\PY{o}{.}\PY{n}{status}\PY{p}{(}\PY{p}{)} 
\end{Verbatim}
\end{tcolorbox}

            \begin{tcolorbox}[breakable, size=fbox, boxrule=.5pt, pad at break*=1mm, opacityfill=0]
\prompt{Out}{outcolor}{5}{\boxspacing}
\begin{Verbatim}[commandchars=\\\{\}]
<JobStatus.DONE: 'job has successfully run'>
\end{Verbatim}
\end{tcolorbox}
        
    \begin{tcolorbox}[breakable, size=fbox, boxrule=1pt, pad at break*=1mm,colback=cellbackground, colframe=cellborder]
\prompt{In}{incolor}{6}{\boxspacing}
\begin{Verbatim}[commandchars=\\\{\}]
\PY{c+c1}{\PYZsh{} The next statement can put your session on hold until the job runs and returns}
\PY{n}{result\PYZus{}Deutsch\PYZus{}Q} \PY{o}{=} \PY{n}{job\PYZus{}Deutsch\PYZus{}Q}\PY{o}{.}\PY{n}{result}\PY{p}{(}\PY{p}{)}
\PY{n}{ans\PYZus{}f} \PY{o}{=} \PY{n}{result\PYZus{}Deutsch\PYZus{}Q}\PY{p}{[}\PY{l+m+mi}{0}\PY{p}{]}\PY{o}{.}\PY{n}{data}\PY{o}{.}\PY{n}{c}\PY{o}{.}\PY{n}{get\PYZus{}counts}\PY{p}{(}\PY{p}{)}
\PY{n}{ans\PYZus{}g} \PY{o}{=} \PY{n}{result\PYZus{}Deutsch\PYZus{}Q}\PY{p}{[}\PY{l+m+mi}{1}\PY{p}{]}\PY{o}{.}\PY{n}{data}\PY{o}{.}\PY{n}{c}\PY{o}{.}\PY{n}{get\PYZus{}counts}\PY{p}{(}\PY{p}{)}

\PY{c+c1}{\PYZsh{} Print the measurement results}
\PY{k}{if} \PY{n+nb}{int}\PY{p}{(}\PY{n+nb}{max}\PY{p}{(}\PY{n}{ans\PYZus{}f}\PY{p}{,} \PY{n}{key}\PY{o}{=}\PY{n}{ans\PYZus{}f}\PY{o}{.}\PY{n}{get}\PY{p}{)}\PY{p}{)} \PY{o}{==} \PY{l+m+mi}{0}\PY{p}{:}
    \PY{n+nb}{print}\PY{p}{(}\PY{l+s+s1}{\PYZsq{}}\PY{l+s+s1}{f is constant}\PY{l+s+s1}{\PYZsq{}}\PY{p}{)}
\PY{k}{else}\PY{p}{:}
    \PY{n+nb}{print}\PY{p}{(}\PY{l+s+s1}{\PYZsq{}}\PY{l+s+s1}{f is balanced}\PY{l+s+s1}{\PYZsq{}}\PY{p}{)}
\PY{k}{if} \PY{n+nb}{int}\PY{p}{(}\PY{n+nb}{max}\PY{p}{(}\PY{n}{ans\PYZus{}g}\PY{p}{,} \PY{n}{key}\PY{o}{=}\PY{n}{ans\PYZus{}g}\PY{o}{.}\PY{n}{get}\PY{p}{)}\PY{p}{)} \PY{o}{==} \PY{l+m+mi}{0}\PY{p}{:}
    \PY{n+nb}{print}\PY{p}{(}\PY{l+s+s1}{\PYZsq{}}\PY{l+s+s1}{g is constant}\PY{l+s+s1}{\PYZsq{}}\PY{p}{)}
\PY{k}{else}\PY{p}{:}
    \PY{n+nb}{print}\PY{p}{(}\PY{l+s+s1}{\PYZsq{}}\PY{l+s+s1}{g is balanced}\PY{l+s+s1}{\PYZsq{}}\PY{p}{)}
\end{Verbatim}
\end{tcolorbox}

    \begin{Verbatim}[commandchars=\\\{\}]
f is balanced
g is constant
    \end{Verbatim}

    \subsection*{\texorpdfstring{\textbf{Experimental measurement of a
Bell inequality (CHSH combination of
correlators)}}{Experimental measurement of a Bell inequality   (CHSH combination of correlators)}}\label{experimental-measurement-of-a-bell-inequality-chsh-combination-of-correlators}

If \(\{A_1, A_2\}\) and \(\{B_1,B_2\}\) are two pairs of observables
(with dichotomic/binary outcome) of two spatially separated systems, the
expected values of their products \(\langle A_i B_j \rangle\) according
to any local hidden variable model (that is, an attempt at trying to
explain away quantum features with additional classical mechanics
variables) satisfy the classical CHSH inequality

\[
|\langle A_1 B_1\rangle +\langle A_1 B_2\rangle + \langle A_2 B_1\rangle - \langle A_2 B_2\rangle|\leq 2 .
\]

Quantum theory, on the contrary, predicts that this inequality is
violated for a suitable choice of observables, obtaining the maximum
violation, in the case of two qubits, when

\[
A_1=X,\quad A_2=Y, \quad B_1=\frac{-(X+Y)}{\sqrt{2}}, \quad B_2=\frac{-(X-Y)}{\sqrt{2}}
\] or rotationally equivalent configurations. Here \(X\) and \(Y\)
denote the \(\sigma_x\) and \(\sigma_y\) Pauli matrices.

\textbf{Question 2}: Using this command,
define the four product observables that appear in the CHSH inequality:
\(A_1B_1\), \(A_2B_1\), \(A_1B_2\), and \(A_2B_2\):


\textbf{Answer:}

    \begin{tcolorbox}[breakable, size=fbox, boxrule=1pt, pad at break*=1mm,colback=cellbackground, colframe=cellborder]
\prompt{In}{incolor}{4}{\boxspacing}
\begin{Verbatim}[commandchars=\\\{\}]
\PY{c+c1}{\PYZsh{} Define pairs of observables for maximum violation of the CHSH innequality}
\PY{n}{coef} \PY{o}{=} \PY{l+m+mi}{1} \PY{o}{/} \PY{n}{np}\PY{o}{.}\PY{n}{sqrt}\PY{p}{(}\PY{l+m+mi}{2}\PY{p}{)}
\PY{n}{A1B1aux} \PY{o}{=} \PY{n}{SparsePauliOp}\PY{o}{.}\PY{n}{from\PYZus{}list}\PY{p}{(}\PY{p}{[}\PY{p}{(}\PY{l+s+s1}{\PYZsq{}}\PY{l+s+s1}{XX}\PY{l+s+s1}{\PYZsq{}}\PY{p}{,} \PY{o}{\PYZhy{}} \PY{n}{coef}\PY{p}{)}\PY{p}{,} \PY{p}{(}\PY{l+s+s1}{\PYZsq{}}\PY{l+s+s1}{XY}\PY{l+s+s1}{\PYZsq{}}\PY{p}{,} \PY{o}{\PYZhy{}} \PY{n}{coef}\PY{p}{)}\PY{p}{]}\PY{p}{)}
\PY{n}{A1B2aux} \PY{o}{=} \PY{n}{SparsePauliOp}\PY{o}{.}\PY{n}{from\PYZus{}list}\PY{p}{(}\PY{p}{[}\PY{p}{(}\PY{l+s+s1}{\PYZsq{}}\PY{l+s+s1}{XX}\PY{l+s+s1}{\PYZsq{}}\PY{p}{,} \PY{o}{\PYZhy{}} \PY{n}{coef}\PY{p}{)}\PY{p}{,} \PY{p}{(}\PY{l+s+s1}{\PYZsq{}}\PY{l+s+s1}{XY}\PY{l+s+s1}{\PYZsq{}}\PY{p}{,} \PY{o}{+} \PY{n}{coef}\PY{p}{)}\PY{p}{]}\PY{p}{)}
\PY{n}{A2B1aux} \PY{o}{=} \PY{n}{SparsePauliOp}\PY{o}{.}\PY{n}{from\PYZus{}list}\PY{p}{(}\PY{p}{[}\PY{p}{(}\PY{l+s+s1}{\PYZsq{}}\PY{l+s+s1}{YX}\PY{l+s+s1}{\PYZsq{}}\PY{p}{,} \PY{o}{\PYZhy{}} \PY{n}{coef}\PY{p}{)}\PY{p}{,} \PY{p}{(}\PY{l+s+s1}{\PYZsq{}}\PY{l+s+s1}{YY}\PY{l+s+s1}{\PYZsq{}}\PY{p}{,} \PY{o}{\PYZhy{}} \PY{n}{coef}\PY{p}{)}\PY{p}{]}\PY{p}{)}
\PY{n}{A2B2aux} \PY{o}{=} \PY{n}{SparsePauliOp}\PY{o}{.}\PY{n}{from\PYZus{}list}\PY{p}{(}\PY{p}{[}\PY{p}{(}\PY{l+s+s1}{\PYZsq{}}\PY{l+s+s1}{YX}\PY{l+s+s1}{\PYZsq{}}\PY{p}{,} \PY{o}{\PYZhy{}} \PY{n}{coef}\PY{p}{)}\PY{p}{,} \PY{p}{(}\PY{l+s+s1}{\PYZsq{}}\PY{l+s+s1}{YY}\PY{l+s+s1}{\PYZsq{}}\PY{p}{,} \PY{o}{+} \PY{n}{coef}\PY{p}{)}\PY{p}{]}\PY{p}{)}
\PY{n}{Obsaux} \PY{o}{=} \PY{p}{[}\PY{n}{A1B1aux}\PY{p}{,} \PY{n}{A1B2aux}\PY{p}{,} \PY{n}{A2B1aux}\PY{p}{,} \PY{n}{A2B2aux}\PY{p}{]}

\PY{c+c1}{\PYZsh{} We need to fill the rest of the qubits with the identity (do nothing) }
\PY{c+c1}{\PYZsh{} this is thought for the current 127\PYZhy{}qubit machines}
\PY{n}{n\PYZus{}remaining\PYZus{}qubits} \PY{o}{=} \PY{l+m+mi}{125}
\PY{n}{I125} \PY{o}{=} \PY{n}{SparsePauliOp}\PY{p}{(}\PY{l+s+s1}{\PYZsq{}}\PY{l+s+s1}{I}\PY{l+s+s1}{\PYZsq{}} \PY{o}{*} \PY{n}{n\PYZus{}remaining\PYZus{}qubits}\PY{p}{)}
\PY{n}{Obs} \PY{o}{=} \PY{p}{[}\PY{n}{I125}\PY{o}{.}\PY{n}{tensor}\PY{p}{(}\PY{n}{op}\PY{p}{)} \PY{k}{for} \PY{n}{op} \PY{o+ow}{in} \PY{n}{Obsaux}\PY{p}{]}
\end{Verbatim}
\end{tcolorbox}

\textbf{Question 1}: Define a circuit that
prepares the ``singlet state'' of two qubits
\(|\Psi^-\rangle=\tfrac{1}{\sqrt2}(|01\rangle-|10\rangle)\).


\textbf{Answer:} The following circuit specifically prepares the state
\(\tfrac{i}{\sqrt2}(|01\rangle-|10\rangle)\), which is physically
equivalent to the requested state.

    \begin{tcolorbox}[breakable, size=fbox, boxrule=1pt, pad at break*=1mm,colback=cellbackground, colframe=cellborder]
\prompt{In}{incolor}{5}{\boxspacing}
\begin{Verbatim}[commandchars=\\\{\}]
\PY{c+c1}{\PYZsh{} Prepare the input circuit:}
\PY{n}{chsh\PYZus{}circuit} \PY{o}{=} \PY{n}{QuantumCircuit}\PY{p}{(}\PY{l+m+mi}{2}\PY{p}{)}
\PY{n}{chsh\PYZus{}circuit}\PY{o}{.}\PY{n}{h}\PY{p}{(}\PY{l+m+mi}{0}\PY{p}{)} \PY{c+c1}{\PYZsh{} Hadamard on the first qubit}
\PY{n}{chsh\PYZus{}circuit}\PY{o}{.}\PY{n}{cx}\PY{p}{(}\PY{l+m+mi}{0}\PY{p}{,} \PY{l+m+mi}{1}\PY{p}{)} \PY{c+c1}{\PYZsh{} NOT controlled to the first qubit}
\PY{n}{chsh\PYZus{}circuit}\PY{o}{.}\PY{n}{y}\PY{p}{(}\PY{p}{[}\PY{l+m+mi}{0}\PY{p}{]}\PY{p}{)} \PY{c+c1}{\PYZsh{} Y\PYZhy{}Pauli gate to the first qubit}
\PY{n}{chsh\PYZus{}circuit}\PY{o}{.}\PY{n}{draw}\PY{p}{(}\PY{n}{output}\PY{o}{=}\PY{l+s+s1}{\PYZsq{}}\PY{l+s+s1}{mpl}\PY{l+s+s1}{\PYZsq{}}\PY{p}{)}
\end{Verbatim}
\end{tcolorbox}
 
            
\prompt{Out}{outcolor}{5}{}
    
    \begin{center}
    \adjustimage{max size={0.9\linewidth}{0.9\paperheight}}{ZapataFergusonM_Session3_files/ZapataFergusonM_Session3_43_0.png}
    \end{center}
    { \hspace*{\fill} \\}
    

\textbf{Question 3}: Calculate the simulated
value obtained for the CHSH inequality jointly with its error.


    \textbf{Answer:} The code we ran in the lab with UCM tokens yielded:

\texttt{The\ simulated\ result\ is\ 2.81824\ +\ 0.04:}

\texttt{does\ it\ exceed\ 2\ with\ sufficient\ statistical\ certainty?}

We can perform the calculations again ourselves.

    \begin{tcolorbox}[breakable, size=fbox, boxrule=1pt, pad at break*=1mm,colback=cellbackground, colframe=cellborder]
\prompt{In}{incolor}{8}{\boxspacing}
\begin{Verbatim}[commandchars=\\\{\}]
\PY{c+c1}{\PYZsh{} Execute the circuit}
\PY{n}{job\PYZus{}CHSH\PYZus{}S} \PY{o}{=} \PY{n}{estimator\PYZus{}S}\PY{o}{.}\PY{n}{run}\PY{p}{(}\PY{p}{[}\PY{p}{(}\PY{n}{chsh\PYZus{}circuit}\PY{p}{,} \PY{n}{Obsaux}\PY{p}{)}\PY{p}{]}\PY{p}{)}
\PY{n}{result\PYZus{}S} \PY{o}{=} \PY{n}{job\PYZus{}CHSH\PYZus{}S}\PY{o}{.}\PY{n}{result}\PY{p}{(}\PY{p}{)}

\PY{c+c1}{\PYZsh{} Statistical results}
\PY{n}{values\PYZus{}S} \PY{o}{=} \PY{n}{result\PYZus{}S}\PY{p}{[}\PY{l+m+mi}{0}\PY{p}{]}\PY{o}{.}\PY{n}{data}\PY{o}{.}\PY{n}{evs}
\PY{n}{Standard\PYZus{}errors\PYZus{}S} \PY{o}{=} \PY{n}{result\PYZus{}S}\PY{p}{[}\PY{l+m+mi}{0}\PY{p}{]}\PY{o}{.}\PY{n}{data}\PY{o}{.}\PY{n}{stds}

\PY{c+c1}{\PYZsh{} Print simulated expectation values}
\PY{n+nb}{print}\PY{p}{(}\PY{l+s+s1}{\PYZsq{}}\PY{l+s+s1}{Simulated expectation values for the four correlators:}\PY{l+s+s1}{\PYZsq{}}\PY{p}{)}
\PY{n+nb}{print}\PY{p}{(}\PY{p}{[}\PY{l+s+sa}{f}\PY{l+s+s1}{\PYZsq{}}\PY{l+s+si}{\PYZob{}}\PY{n}{v}\PY{l+s+si}{:}\PY{l+s+s1}{.3f}\PY{l+s+si}{\PYZcb{}}\PY{l+s+s1}{ ± }\PY{l+s+si}{\PYZob{}}\PY{n}{s}\PY{l+s+si}{:}\PY{l+s+s1}{.3f}\PY{l+s+si}{\PYZcb{}}\PY{l+s+s1}{\PYZsq{}} \PY{k}{for} \PY{n}{v}\PY{p}{,} \PY{n}{s} \PY{o+ow}{in} \PY{n+nb}{zip}\PY{p}{(}\PY{n}{values\PYZus{}S}\PY{p}{,} \PY{n}{Standard\PYZus{}errors\PYZus{}S}\PY{p}{)}\PY{p}{]}\PY{p}{)}
\end{Verbatim}
\end{tcolorbox}

    \begin{Verbatim}[commandchars=\\\{\}]
Simulated expectation values for the four correlators:
['0.711 ± 0.011', '0.703 ± 0.011', '0.702 ± 0.011', '-0.713 ± 0.011']
    \end{Verbatim}

    \begin{tcolorbox}[breakable, size=fbox, boxrule=1pt, pad at break*=1mm,colback=cellbackground, colframe=cellborder]
\prompt{In}{incolor}{9}{\boxspacing}
\begin{Verbatim}[commandchars=\\\{\}]
\PY{c+c1}{\PYZsh{} CHSH value}
\PY{n}{CHSH\PYZus{}mean\PYZus{}S} \PY{o}{=} \PY{n+nb}{abs}\PY{p}{(}\PY{n}{values\PYZus{}S}\PY{p}{[}\PY{l+m+mi}{0}\PY{p}{]} \PY{o}{+} \PY{n}{values\PYZus{}S}\PY{p}{[}\PY{l+m+mi}{1}\PY{p}{]} \PY{o}{+} \PY{n}{values\PYZus{}S}\PY{p}{[}\PY{l+m+mi}{2}\PY{p}{]} \PY{o}{\PYZhy{}} \PY{n}{values\PYZus{}S}\PY{p}{[}\PY{l+m+mi}{3}\PY{p}{]}\PY{p}{)}
\PY{n}{CHSH\PYZus{}uncertainty\PYZus{}S} \PY{o}{=} \PY{n+nb}{sum}\PY{p}{(}\PY{n}{Standard\PYZus{}errors\PYZus{}S}\PY{p}{)}
\PY{n+nb}{print}\PY{p}{(}\PY{l+s+sa}{f}\PY{l+s+s1}{\PYZsq{}\PYZsq{}\PYZsq{}}
\PY{l+s+s1}{The simulated result is }\PY{l+s+si}{\PYZob{}}\PY{n}{CHSH\PYZus{}mean\PYZus{}S}\PY{l+s+si}{:}\PY{l+s+s1}{.3f}\PY{l+s+si}{\PYZcb{}}\PY{l+s+s1}{ ± }\PY{l+s+si}{\PYZob{}}\PY{n}{CHSH\PYZus{}uncertainty\PYZus{}S}\PY{l+s+si}{:}\PY{l+s+s1}{.3f}\PY{l+s+si}{\PYZcb{}}\PY{l+s+s1}{:}
\PY{l+s+s1}{does it exceed 2 with sufficient statistical certainty?}
\PY{l+s+s1}{\PYZsq{}\PYZsq{}\PYZsq{}}\PY{p}{)}
\end{Verbatim}
\end{tcolorbox}

    \begin{Verbatim}[commandchars=\\\{\}]

The simulated result is 2.828 ± 0.044:
does it exceed 2 with sufficient statistical certainty?

    \end{Verbatim}

\textbf{Question 4}: Compute, with a hand
calculator or simple python commands, the experimental value obtained
for the CHSH inequality jointly with its error from the quantum data
obtained.


    \textbf{Answer:} The code we ran in the lab with UCM tokens yielded:

\texttt{The\ simulated\ result\ is\ 2.60215\ +\ 0.41467:}

\texttt{does\ it\ exceed\ 2\ with\ sufficient\ statistical\ certainty?}

Let's perform the calculations again ourselves.

    \begin{tcolorbox}[breakable, size=fbox, boxrule=1pt, pad at break*=1mm,colback=cellbackground, colframe=cellborder]
\prompt{In}{incolor}{65}{\boxspacing}
\begin{Verbatim}[commandchars=\\\{\}]
\PY{c+c1}{\PYZsh{} Transpile circuit to backend architecture}
\PY{n}{chsh\PYZus{}circuit\PYZus{}t} \PY{o}{=} \PY{n}{transpile}\PY{p}{(}\PY{n}{chsh\PYZus{}circuit}\PY{p}{,} \PY{n}{backend}\PY{o}{=}\PY{n}{backend\PYZus{}Q}\PY{p}{)}

\PY{c+c1}{\PYZsh{} Execute the circuit}
\PY{n}{job\PYZus{}CHSH\PYZus{}Q} \PY{o}{=} \PY{n}{estimator\PYZus{}Q}\PY{o}{.}\PY{n}{run}\PY{p}{(}\PY{p}{[}\PY{p}{(}\PY{n}{chsh\PYZus{}circuit\PYZus{}t}\PY{p}{,} \PY{n}{Obs}\PY{p}{)}\PY{p}{]}\PY{p}{)}

\PY{c+c1}{\PYZsh{} Save job identification code for later}
\PY{n}{df}\PY{o}{.}\PY{n}{chsh\PYZus{}id} \PY{o}{=} \PY{n}{job\PYZus{}CHSH\PYZus{}Q}\PY{o}{.}\PY{n}{job\PYZus{}id}\PY{p}{(}\PY{p}{)}
\PY{n}{df}\PY{o}{.}\PY{n}{to\PYZus{}csv}\PY{p}{(}\PY{l+s+s1}{\PYZsq{}}\PY{l+s+s1}{session.csv}\PY{l+s+s1}{\PYZsq{}}\PY{p}{,} \PY{n}{index}\PY{o}{=}\PY{k+kc}{False}\PY{p}{)}
\end{Verbatim}
\end{tcolorbox}

    \begin{Verbatim}[commandchars=\\\{\}]
c:\textbackslash{}Users\textbackslash{}zapat\textbackslash{}Escritorio\textbackslash{}CODE\textbackslash{}PIE\_Compu\_Cuantica\textbackslash{}venv\textbackslash{}Lib\textbackslash{}site-
packages\textbackslash{}qiskit\_ibm\_runtime\textbackslash{}qiskit\_runtime\_service.py:879: UserWarning: Your
current pending jobs are estimated to consume 623.1034379365619 quantum seconds,
but you only have 586 quantum seconds left in your monthly quota; therefore, it
is likely this job will be canceled
  warnings.warn(warning\_message)
    \end{Verbatim}

    \begin{tcolorbox}[breakable, size=fbox, boxrule=1pt, pad at break*=1mm,colback=cellbackground, colframe=cellborder]
\prompt{In}{incolor}{7}{\boxspacing}
\begin{Verbatim}[commandchars=\\\{\}]
\PY{c+c1}{\PYZsh{} The following cell retrieves information about the job}
\PY{n}{job\PYZus{}CHSH\PYZus{}Q} \PY{o}{=} \PY{n}{service}\PY{o}{.}\PY{n}{job}\PY{p}{(}\PY{n}{df}\PY{o}{.}\PY{n}{chsh\PYZus{}id}\PY{p}{[}\PY{l+m+mi}{0}\PY{p}{]}\PY{p}{)}
\PY{n}{job\PYZus{}CHSH\PYZus{}Q}\PY{o}{.}\PY{n}{status}\PY{p}{(}\PY{p}{)} 
\end{Verbatim}
\end{tcolorbox}

            \begin{tcolorbox}[breakable, size=fbox, boxrule=.5pt, pad at break*=1mm, opacityfill=0]
\prompt{Out}{outcolor}{7}{\boxspacing}
\begin{Verbatim}[commandchars=\\\{\}]
<JobStatus.DONE: 'job has successfully run'>
\end{Verbatim}
\end{tcolorbox}
        
    \begin{tcolorbox}[breakable, size=fbox, boxrule=1pt, pad at break*=1mm,colback=cellbackground, colframe=cellborder]
\prompt{In}{incolor}{8}{\boxspacing}
\begin{Verbatim}[commandchars=\\\{\}]
\PY{c+c1}{\PYZsh{} The next statement can put your session on hold until the job runs and returns}
\PY{n}{result\PYZus{}Q} \PY{o}{=} \PY{n}{job\PYZus{}CHSH\PYZus{}Q}\PY{o}{.}\PY{n}{result}\PY{p}{(}\PY{p}{)}

\PY{c+c1}{\PYZsh{} Statistical results}
\PY{n}{values\PYZus{}Q} \PY{o}{=} \PY{n}{result\PYZus{}Q}\PY{p}{[}\PY{l+m+mi}{0}\PY{p}{]}\PY{o}{.}\PY{n}{data}\PY{o}{.}\PY{n}{evs}
\PY{n}{Standard\PYZus{}errors\PYZus{}Q} \PY{o}{=} \PY{n}{result\PYZus{}Q}\PY{p}{[}\PY{l+m+mi}{0}\PY{p}{]}\PY{o}{.}\PY{n}{data}\PY{o}{.}\PY{n}{stds}

\PY{c+c1}{\PYZsh{} Print simulated expectation values}
\PY{n+nb}{print}\PY{p}{(}\PY{l+s+s1}{\PYZsq{}}\PY{l+s+s1}{Simulated expectation values for the four correlators:}\PY{l+s+s1}{\PYZsq{}}\PY{p}{)}
\PY{n+nb}{print}\PY{p}{(}\PY{p}{[}\PY{l+s+sa}{f}\PY{l+s+s1}{\PYZsq{}}\PY{l+s+si}{\PYZob{}}\PY{n}{v}\PY{l+s+si}{:}\PY{l+s+s1}{.3f}\PY{l+s+si}{\PYZcb{}}\PY{l+s+s1}{ ± }\PY{l+s+si}{\PYZob{}}\PY{n}{s}\PY{l+s+si}{:}\PY{l+s+s1}{.3f}\PY{l+s+si}{\PYZcb{}}\PY{l+s+s1}{\PYZsq{}} \PY{k}{for} \PY{n}{v}\PY{p}{,} \PY{n}{s} \PY{o+ow}{in} \PY{n+nb}{zip}\PY{p}{(}\PY{n}{values\PYZus{}Q}\PY{p}{,} \PY{n}{Standard\PYZus{}errors\PYZus{}Q}\PY{p}{)}\PY{p}{]}\PY{p}{)}
\end{Verbatim}
\end{tcolorbox}

    \begin{Verbatim}[commandchars=\\\{\}]
Simulated expectation values for the four correlators:
['0.689 ± 0.036', '0.801 ± 0.036', '0.744 ± 0.034', '-0.665 ± 0.034']
    \end{Verbatim}

    \begin{tcolorbox}[breakable, size=fbox, boxrule=1pt, pad at break*=1mm,colback=cellbackground, colframe=cellborder]
\prompt{In}{incolor}{9}{\boxspacing}
\begin{Verbatim}[commandchars=\\\{\}]
\PY{c+c1}{\PYZsh{} CHSH value}
\PY{n}{values\PYZus{}Q} \PY{o}{=} \PY{n}{result\PYZus{}Q}\PY{p}{[}\PY{l+m+mi}{0}\PY{p}{]}\PY{o}{.}\PY{n}{data}\PY{o}{.}\PY{n}{evs}
\PY{n}{Standard\PYZus{}errors\PYZus{}Q} \PY{o}{=} \PY{n}{result\PYZus{}Q}\PY{p}{[}\PY{l+m+mi}{0}\PY{p}{]}\PY{o}{.}\PY{n}{data}\PY{o}{.}\PY{n}{stds}
\PY{n}{CHSH\PYZus{}mean\PYZus{}Q} \PY{o}{=} \PY{n+nb}{abs}\PY{p}{(}\PY{n}{values\PYZus{}Q}\PY{p}{[}\PY{l+m+mi}{0}\PY{p}{]} \PY{o}{+} \PY{n}{values\PYZus{}Q}\PY{p}{[}\PY{l+m+mi}{1}\PY{p}{]} \PY{o}{+} \PY{n}{values\PYZus{}Q}\PY{p}{[}\PY{l+m+mi}{2}\PY{p}{]} \PY{o}{\PYZhy{}} \PY{n}{values\PYZus{}Q}\PY{p}{[}\PY{l+m+mi}{3}\PY{p}{]}\PY{p}{)}
\PY{n}{CHSH\PYZus{}uncertainty\PYZus{}Q} \PY{o}{=} \PY{n+nb}{sum}\PY{p}{(}\PY{n}{Standard\PYZus{}errors\PYZus{}Q}\PY{p}{)}
\PY{n+nb}{print}\PY{p}{(}\PY{l+s+sa}{f}\PY{l+s+s1}{\PYZsq{}\PYZsq{}\PYZsq{}}
\PY{l+s+s1}{The simulated result is }\PY{l+s+si}{\PYZob{}}\PY{n}{CHSH\PYZus{}mean\PYZus{}Q}\PY{l+s+si}{:}\PY{l+s+s1}{.3f}\PY{l+s+si}{\PYZcb{}}\PY{l+s+s1}{ ± }\PY{l+s+si}{\PYZob{}}\PY{n}{CHSH\PYZus{}uncertainty\PYZus{}Q}\PY{l+s+si}{:}\PY{l+s+s1}{.3f}\PY{l+s+si}{\PYZcb{}}\PY{l+s+s1}{:}
\PY{l+s+s1}{does it exceed 2 with sufficient statistical certainty?}
\PY{l+s+s1}{\PYZsq{}\PYZsq{}\PYZsq{}}\PY{p}{)}
\end{Verbatim}
\end{tcolorbox}

    \begin{Verbatim}[commandchars=\\\{\}]

The simulated result is 2.899 ± 0.139:
does it exceed 2 with sufficient statistical certainty?

    \end{Verbatim}

    The qubits we used for our computation exhibit quantum statistics, and
thus are properly entangled. A computation such as this serves just as
much as a proof of CHSH violation as it does as a calibration of a
quantum computer.

    \section*{\texorpdfstring{\textbf{References}}{References}}\label{references}

{[}1{]} Cleve, R., Ekert, A., Macchiavello, C., \& Mosca, M. (1998).
Quantum algorithms revisited. \emph{Proceedings of the Royal Society of
London. Series A: Mathematical, Physical and Engineering Sciences,
454(1969), 339--354}. Retrieved from
\href{https://arxiv.org/abs/quant-ph/9708016}{https://arxiv.org/abs/quant-ph/9708016}.


    % Add a bibliography block to the postdoc
    
    
    
\end{document}
